\documentclass[UTF8,aspectratio=169]{beamer}



% 基本包
\usepackage[UTF8]{ctex}
\usepackage{graphicx}
\usepackage{amsmath}
\usepackage{amsfonts}
\usepackage{amssymb}
% \usepackage{listings}  % 已替换为minted
\usepackage{xcolor}
\usepackage{hyperref}
\usepackage{booktabs}
\usepackage{multirow}
\usepackage{multicol}
\usepackage{float}
\usepackage{tikz}
\usetikzlibrary{positioning,shapes,arrows,fit,backgrounds}
\usepackage{pgfplots}
\pgfplotsset{compat=1.18}
\usepackage{minted}
\usepackage{fontspec}
\usepackage[most]{tcolorbox}

% Beamer主题设置
\usetheme{Madrid}
\usecolortheme{whale}

% 校徽设置
\logo{
  \IfFileExists{../长江大学校徽.pdf}{
    \begin{tikzpicture}[remember picture,overlay]
      \node[anchor=north east, xshift=-0.2mm, yshift=-0.2mm] at (current page.north east) {
        \includegraphics[height=1.0cm]{../长江大学校徽.pdf}
      };
    \end{tikzpicture}
  }{
    \begin{tikzpicture}[remember picture,overlay]
      \node[anchor=north east, xshift=-3mm, yshift=-3mm] at (current page.north east) {
        \textcolor{red}{\tiny [校徽文件未找到]}
      };
    \end{tikzpicture}
  }
}

% ===== 使用推荐的 font themes =====
\usefonttheme{professionalfonts}  % 允许自定义字体

% 自定义frame标题栏,也缩短长度留出logo空间
% 设置标题栏背景颜色为淡蓝色
\definecolor{frametitlebg}{RGB}{200,215,250} % 淡蓝色,可根据需要调整
\setbeamercolor{frametitle}{bg=frametitlebg, fg=ytublue!80!black}
\setbeamertemplate{frametitle}{
  \ifbeamercolorempty[bg]{frametitle}{}{\nointerlineskip}%
  \begin{tcolorbox}[
    enhanced,
    width=0.90\paperwidth,
    height=2.5ex,
    colback=frametitlebg,
    colframe=frametitlebg,
    boxrule=0pt,
    left=0pt,
    right=0pt,
    top=1pt,
    bottom=0pt,
    boxsep=0pt,
    before skip=0pt,
    after skip=0.1em,  % 减少标题栏和内容之间的间距
  ]
  \vspace{0.2ex} % 减少标题文字上方的空白
  \usebeamerfont{frametitle}\textcolor{ytublue!80!black}{\hspace{1em}\insertframetitle}
  \end{tcolorbox}
}

% ===== 设置现代字体 =====
\setsansfont{Source Sans Pro}     % 正文字体
\setmonofont{Source Code Pro}[Scale=0.9]  % 代码字体,稍微缩小一点

\setbeamertemplate{navigation symbols}{}

% 减少页面间距
\setbeamertemplate{itemize items}[circle]
\setbeamertemplate{enumerate items}[default]
\setlength{\itemsep}{0.1em}
\setlength{\parskip}{0.1em}

% 页码设置
\setbeamertemplate{footline}[frame number]

% 定义流程图样式
\tikzset{
    block/.style = {rectangle, draw, fill=blue!10,
        minimum width=6em, align=center, rounded corners, minimum height=3em},
    line/.style = {draw, -latex'}
}
% 水印设置
\setbeamertemplate{background}{
    \begin{tikzpicture}[remember picture,overlay]
        \node[rotate=-45,scale=0.8,opacity=0.1,color=gray]
             at ([xshift=0.5cm,yshift=0.5cm]current page.south west)
             {\large\textbf{WPJ}};
    \end{tikzpicture}
}

% 自定义颜色
\definecolor{qtgreen}{RGB}{41,128,185}
\definecolor{qtblue}{RGB}{52,73,94}

% 定义长江大学蓝主色调
\definecolor{ytublue}{RGB}{0,84,159}
% 统一block样式
\newtcolorbox{ytublock}[1]{
  colback=white,
  colframe=ytublue!80!black,
  colbacktitle=ytublue!20!white,
  coltitle=ytublue!80!black,
  title={#1},
  fonttitle=\bfseries,
  arc=3mm,
  boxrule=1pt,
  boxsep=1mm,
  left=2mm,
  right=2mm,
  top=0.5mm,
  bottom=0.5mm,
  before skip=3pt,
  after skip=3pt,
  enhanced,
  drop fuzzy shadow=ytublue!20!black
}

% 定义警告块样式
\newtcolorbox{ytualertblock}[1]{
  colback=white,
  colframe=red!80!black,
  colbacktitle=red!20!white,
  coltitle=red!80!black,
  title={#1},
  fonttitle=\bfseries,
  arc=3mm,
  boxrule=1.5pt,
  boxsep=1mm,
  left=2mm,
  right=2mm,
  top=0.5mm,
  bottom=0.5mm,
  before skip=3pt,
  after skip=3pt,
  enhanced,
  drop fuzzy shadow=red!20!black,
  overlay={
    \begin{tcbclipinterior}
      \fill[red!10!white] (interior.south west) rectangle (interior.north east);
    \end{tcbclipinterior}
  }
}

% cpp代码高亮设置
\setminted[cpp]{
    fontsize=\tiny,
    fontfamily=tt,             % 使用等宽字体
    linenos=true,
    frame=lines,               % 上下两条线,简洁清爽(比 tb 更现代)
    framesep=3mm,              % 内边距
    rulecolor=\color{blue!20}, % 线条颜色浅蓝,不刺眼
    bgcolor=blue!10,           % 浅蓝色背景
    baselinestretch=1.2,       % 行距稍大,更易读
    breaklines=true,
    breakautoindent=true,
    tabsize=4,
    xleftmargin=5mm,
    xrightmargin=5mm,
    numbersep=8pt,             % 行号与代码间距
    % ===== 其他美化 =====
    obeytabs=true,             % 尊重 tab 字符
    samepage=false,            % 允许跨页(重要!避免空白)
    escapeinside=||,           % 可在代码中使用 |LaTeX| 插入 LaTeX 命令
}

% 设置标题页颜色,与frame标题保持一致
\setbeamercolor{title}{bg=frametitlebg, fg=ytublue!80!black}
\setbeamercolor{subtitle}{bg=frametitlebg, fg=ytublue!70!black}
\setbeamercolor{author}{bg=frametitlebg, fg=ytublue!80!black}
\setbeamercolor{institute}{bg=frametitlebg, fg=ytublue!80!black}
\setbeamercolor{date}{bg=frametitlebg, fg=ytublue!80!black}

% 自定义标题页样式,全部内容同一个tcolorbox,居中排版,字体和间距区分
\setbeamertemplate{title page}{
  \vbox{}
  \begingroup
    \centering
    \begin{tcolorbox}[
      enhanced,
      width=0.92\paperwidth,
      colback=frametitlebg,
      colframe=frametitlebg,
      boxrule=0pt,
      left=0pt,
      right=0pt,
      top=4mm,
      bottom=4mm,
      boxsep=0pt,
      before skip=0pt,
      after skip=1.2em,
    ]
    % 标题
    {\centering
      {\fontsize{24pt}{27pt}\selectfont\textcolor{ytublue!80!black}{\inserttitle}\par}
      \vspace{1.2em}
      % 副标题
      {\fontsize{21pt}{24pt}\selectfont\textcolor{ytublue!70!black}{\insertsubtitle}\par}
      \vspace{2.0em}
      % 作者
      {\fontsize{12pt}{15pt}\selectfont\insertauthor\par}
      \vspace{0.7em}
      % 单位
      {\fontsize{12pt}{15pt}\selectfont\insertinstitute\par}
      \vspace{0.7em}
      % 日期
      {\fontsize{12pt}{15pt}\selectfont\insertdate\par}
    }
    \vspace{0.5em}
    \vfill
    \end{tcolorbox}
  \endgroup
}


% 文档信息
\title{高等程序设计 - Qt/C++}
\subtitle{第4章:编程范式详解}
\author{王培杰}
\institute{长江大学地球物理与石油资源学院}
\date{\today}

\begin{document}

% 标题页
\begin{frame}
    \titlepage
\end{frame}

% 目录页
\begin{frame}{目录}
    \begin{multicols}{2}
        \tableofcontents[]
    \end{multicols}
\end{frame}

\section{编程范式概述}
\begin{frame}{目录}
    \begin{multicols}{2}
        \tableofcontents[currentsection]
    \end{multicols}
\end{frame}

\begin{frame}{编程范式概念}
    \begin{ytublock}{什么是编程范式?}
        \begin{itemize}
            \item 编程的基本风格和方法论
            \item 解决问题的思维方式和工具
            \item 代码组织和结构的方法
            \item 不同范式的组合使用
        \end{itemize}
    \end{ytublock}

    \begin{ytublock}{主要编程范式}
        \begin{itemize}
            \item 过程式编程(Procedural Programming)
            \item 面向对象编程(Object-Oriented Programming)
            \item 泛型编程(Generic Programming)
            \item 函数式编程(Functional Programming)
            \item 事件驱动编程(Event-Driven Programming)
            \item 声明式编程(Declarative Programming)
            \item 组件式编程(Component-Based Programming)
        \end{itemize}
    \end{ytublock}
\end{frame}

\section{过程式编程}
\begin{frame}{目录}
    \begin{multicols}{2}
        \tableofcontents[currentsection]
    \end{multicols}
\end{frame}

\begin{frame}{过程式编程概述}
    \begin{columns}
        \begin{column}{0.48\textwidth}
            \begin{ytublock}{过程式编程特点}
                \begin{itemize}
                    \item 以过程/函数为中心
                    \item 数据与操作分离
                    \item 顺序执行
                    \item 模块化设计
                    \item 易于理解和调试
                \end{itemize}
            \end{ytublock}
        \end{column}
        \begin{column}{0.48\textwidth}
            \begin{ytublock}{适用场景}
                \begin{itemize}
                    \item 算法实现
                    \item 数据处理
                    \item 工具函数
                    \item 系统编程
                    \item 性能关键代码
                \end{itemize}
            \end{ytublock}
        \end{column}
    \end{columns}
    \begin{ytublock}{过程式编程示意图}
        \begin{center}
        \begin{tikzpicture}[node distance=2.2cm, every node/.style={font=\small}]
            % 节点(横向排列)
            \node[ytublock, minimum width= 1.8cm, minimum height=1.2cm, fill=blue!10] (start) {开始};
            \node[ytublock, right of=start, fill=blue!10, minimum width=1.8cm, minimum height=1.2cm, xshift=0.2cm] (input) {输入数据};
            \node[ytublock, right of=input, fill=green!10, minimum width=1.8cm, minimum height=1.2cm, xshift=0.2cm] (process) {过程/函数};
            \node[ytublock, right of=process, fill=orange!10, minimum width=1.8cm, minimum height=1.2cm, xshift=0.2cm] (output) {输出结果};
            \node[ytublock, right of=output, fill=red!10, minimum width=1.8cm, minimum height=1.2cm, xshift=0.2cm] (end) {结束};
            % 箭头
            \draw[->, thick] (start) -- (input);
            \draw[->, thick] (input) -- (process);
            \draw[->, thick] (process) -- (output);
            \draw[->, thick] (output) -- (end);
        \end{tikzpicture}
        \end{center}
    \end{ytublock}
\end{frame}

\begin{frame}[fragile]{过程式编程示例}
    \begin{columns}
        \begin{column}{0.48\textwidth}
            \inputminted[firstline=1,lastline=19]{cpp}{code/process_example.cpp}
        \end{column}
        \begin{column}{0.48\textwidth}
            \inputminted[firstline=20,lastline=38]{cpp}{code/process_example.cpp}
        \end{column}
    \end{columns}
\end{frame}

\begin{frame}[fragile]{过程式编程示例-算法实现}
    \begin{columns}
        \begin{column}{0.48\textwidth}
            \inputminted[firstline=1,lastline=19]{cpp}{code/process_example_algorithm.cpp}
        \end{column}
        \begin{column}{0.48\textwidth}
            \inputminted[firstline=20,lastline=39]{cpp}{code/process_example_algorithm.cpp}
        \end{column}
    \end{columns}
\end{frame}

\begin{frame}[fragile]{过程式编程示例-数据处理}
    \begin{columns}
        \begin{column}{0.48\textwidth}
            \inputminted[firstline=1,lastline=17]{cpp}{code/process_example_data.cpp}
        \end{column}
        \begin{column}{0.48\textwidth}
            \inputminted[firstline=18,lastline=36]{cpp}{code/process_example_data.cpp}
        \end{column}
    \end{columns}
\end{frame}

\begin{frame}[fragile]{过程式编程示例-工具函数}
    \begin{columns}
        \begin{column}{0.48\textwidth}
            \inputminted[firstline=1,lastline=16]{cpp}{code/process_example_tool_function.cpp}
        \end{column}
        \begin{column}{0.48\textwidth}
            \inputminted[firstline=17,lastline=36]{cpp}{code/process_example_tool_function.cpp}
        \end{column}
    \end{columns}
\end{frame}

\begin{frame}[fragile]{过程式编程示例-系统编程}
    \begin{columns}
        \begin{column}{0.48\textwidth}
            \inputminted[firstline=1,lastline=17]{cpp}{code/process_example_system_function.cpp}
        \end{column}
        \begin{column}{0.48\textwidth}
            \inputminted[firstline=18,lastline=34]{cpp}{code/process_example_system_function.cpp}
        \end{column}
    \end{columns}
\end{frame}

\begin{frame}[fragile]{过程式编程示例-性能关键代码}
    \begin{columns}
        \begin{column}{0.48\textwidth}
            \inputminted[firstline=1,lastline=18]{cpp}{code/process_example_performance_critical.cpp}
        \end{column}
        \begin{column}{0.48\textwidth}
            \inputminted[firstline=19,lastline=36]{cpp}{code/process_example_performance_critical.cpp}
        \end{column}
    \end{columns}
\end{frame}

\section{面向对象编程}
\begin{frame}{目录}
    \begin{multicols}{2}
        \tableofcontents[currentsection]
    \end{multicols}
\end{frame}

\begin{frame}{面向对象编程概述}
    \begin{columns}
        \begin{column}{0.48\textwidth}
            \begin{ytublock}{OOP核心概念}
                \begin{itemize}
                    \item 封装(Encapsulation)
                    \item 继承(Inheritance)
                    \item 多态(Polymorphism)
                    \item 抽象(Abstraction)
                \end{itemize}
            \end{ytublock}
            \begin{ytublock}{Qt中的OOP应用}
                \begin{itemize}
                    \item QWidget继承体系
                    \item 信号槽机制
                    \item 属性系统
                    \item 事件处理
                \end{itemize}
            \end{ytublock}
        \end{column}
        \begin{column}{0.48\textwidth}
            \begin{ytublock}{面向对象编程示意图}
                \begin{center}
                    \begin{tikzpicture}[scale=1.0, transform shape, node distance=1.3cm, auto, thick,
                        every node/.style={rectangle, draw, minimum height=0.6cm, font=\small},
                        process/.style={draw, rectangle, rounded corners, fill=blue!30},
                        decision/.style={diamond, draw, fill=yellow!30, aspect=2},
                        arrow/.style={thick, -{Latex[length=1.5mm,width=1.5mm]}}]

                        % 节点定义
                        \node (start) [process] {开始};
                        \node (instantiateAnimal) [process, below of=start] {实例化 Animal};
                        \node (createDog) [process, below of=instantiateAnimal] {创建 Dog 对象};
                        \node (createCat) [process, right of=createDog, xshift=2.5cm] {创建 Cat 对象};
                        \node (callBark) [process, below of=createDog] {调用 bark() 方法};
                        \node (callMeow) [process, below of=createCat] {调用 meow() 方法};
                        \node (end) [process, below of=callBark, yshift=-0.7cm] {结束};

                        % 连接线
                        \draw [arrow] (start) -- (instantiateAnimal);
                        \draw [arrow] (instantiateAnimal) -- (createDog);
                        \draw [arrow] (instantiateAnimal) -| (createCat);
                        \draw [arrow] (createDog) -- (callBark);
                        \draw [arrow] (createCat) -- (callMeow);
                        \draw [arrow] (callBark) -- ++(0,-1) -| (end);
                        \draw [arrow] (callMeow) -- ++(0,-1) -| (end);

                    \end{tikzpicture}
                \end{center}
            \end{ytublock}
        \end{column}
    \end{columns}
\end{frame}

\begin{frame}[fragile]{面向对象编程示例}
    \begin{columns}
        \begin{column}{0.48\textwidth}
            \inputminted[firstline=1,lastline=19]{cpp}{code/oop_example.cpp}
        \end{column}
        \begin{column}{0.48\textwidth}
            \inputminted[firstline=20,lastline=38]{cpp}{code/oop_example.cpp}
        \end{column}
    \end{columns}
\end{frame}

\begin{frame}[fragile]{面向对象的适用场景}
    \begin{itemize}
        \item 代码重用 : 通过继承和多态实现代码重用。
        \item GUI应用程序 : 通过面向对象设计实现GUI应用程序。
        \item 模拟现实世界 : 在游戏开发中创建角色、车辆等对象。
        \item 复杂系统 : 通过封装和抽象实现复杂系统。
        \item 大型项目 : 通过模块化设计实现大型项目。
        \item 团队协作 : 通过面向对象设计实现团队协作。
        \item 框架开发 : 通过面向对象设计实现框架开发。
    \end{itemize}
\end{frame}

\begin{frame}[fragile]{面向对象编程示例-代码重用(封装)}
    \begin{columns}
        \begin{column}{0.48\textwidth}
            \inputminted[firstline=1,lastline=19]{cpp}{code/oop_example_code_reuse_encapsulation.cpp}
        \end{column}
        \begin{column}{0.48\textwidth}
            \inputminted[firstline=20,lastline=38]{cpp}{code/oop_example_code_reuse_encapsulation.cpp}
        \end{column}
    \end{columns}
\end{frame}

\begin{frame}[fragile]{面向对象编程示例-代码重用(继承)}
    \begin{columns}
        \begin{column}{0.48\textwidth}
            \inputminted[firstline=1,lastline=19]{cpp}{code/oop_example_code_reuse_inheritance.cpp}
        \end{column}
        \begin{column}{0.48\textwidth}
            \inputminted[firstline=20,lastline=39]{cpp}{code/oop_example_code_reuse_inheritance.cpp}
        \end{column}
    \end{columns}
\end{frame}

\begin{frame}[fragile]{面向对象编程示例-代码重用(多态)}
    \begin{columns}
        \begin{column}{0.48\textwidth}
            \inputminted[firstline=1,lastline=16]{cpp}{code/oop_example_code_reuse_polymorphism.cpp}
        \end{column}
        \begin{column}{0.48\textwidth}
            \inputminted[firstline=17,lastline=32]{cpp}{code/oop_example_code_reuse_polymorphism.cpp}
        \end{column}
    \end{columns}
\end{frame}

\begin{frame}[fragile]{面向对象编程示例-GUI}
    \begin{columns}
        \begin{column}{0.48\textwidth}
            \inputminted[firstline=1,lastline=16]{cpp}{code/oop_example_gui.cpp}
        \end{column}
        \begin{column}{0.48\textwidth}
            \inputminted[firstline=17,lastline=32]{cpp}{code/oop_example_gui.cpp}
        \end{column}
    \end{columns}
\end{frame}

\begin{frame}[fragile]{面向对象编程示例-模拟现实世界(图书馆-书籍-读者关系)}
    \begin{columns}
        \begin{column}{0.48\textwidth}
            \inputminted[firstline=1,lastline=18]{cpp}{code/oop_example_real_world.cpp}
        \end{column}
        \begin{column}{0.48\textwidth}
            \inputminted[firstline=19,lastline=36]{cpp}{code/oop_example_real_world.cpp}
        \end{column}
    \end{columns}
\end{frame}

\begin{frame}[fragile]{面向对象编程示例-模拟现实世界(图书馆-书籍-读者关系)}
    \begin{columns}
        \begin{column}{0.48\textwidth}
            \inputminted[firstline=37,lastline=52]{cpp}{code/oop_example_real_world.cpp}
        \end{column}
        \begin{column}{0.48\textwidth}
            \inputminted[firstline=53,lastline=72]{cpp}{code/oop_example_real_world.cpp}
        \end{column}
    \end{columns}
\end{frame}

\begin{frame}[fragile]{面向对象编程示例-模拟现实世界(图书馆-书籍-读者关系)}
    \begin{columns}
        \begin{column}{0.48\textwidth}
            \inputminted[firstline=73,lastline=90]{cpp}{code/oop_example_real_world.cpp}
        \end{column}
        \begin{column}{0.48\textwidth}
            \inputminted[firstline=91,lastline=110]{cpp}{code/oop_example_real_world.cpp}
        \end{column}
    \end{columns}
\end{frame}

\begin{frame}[fragile]{面向对象编程示例-复杂系统(机器人)}
    \begin{columns}
        \begin{column}{0.48\textwidth}
            \inputminted[firstline=1,lastline=18]{cpp}{code/oop_example_complex_system.cpp}
        \end{column}
        \begin{column}{0.48\textwidth}
            \inputminted[firstline=19,lastline=36]{cpp}{code/oop_example_complex_system.cpp}
        \end{column}
    \end{columns}
\end{frame}

\begin{frame}[fragile]{面向对象编程示例-复杂系统(机器人)}
    \begin{columns}
        \begin{column}{0.48\textwidth}
            \inputminted[firstline=37,lastline=54]{cpp}{code/oop_example_complex_system.cpp}
        \end{column}
        \begin{column}{0.48\textwidth}
            \inputminted[firstline=55,lastline=72]{cpp}{code/oop_example_complex_system.cpp}
        \end{column}
    \end{columns}
\end{frame}

\begin{frame}[fragile]{面向对象编程示例-复杂系统(机器人)}
    \begin{columns}
        \begin{column}{0.48\textwidth}
            \inputminted[firstline=73,lastline=91]{cpp}{code/oop_example_complex_system.cpp}
        \end{column}
        \begin{column}{0.48\textwidth}
            \inputminted[firstline=92,lastline=112]{cpp}{code/oop_example_complex_system.cpp}
        \end{column}
    \end{columns}
\end{frame}

\begin{frame}[fragile]{面向对象编程示例-大型项目(游戏引擎)}
    \begin{columns}
        \begin{column}{0.48\textwidth}
            \inputminted[firstline=1,lastline=20]{cpp}{code/oop_example_large_project.cpp}
        \end{column}
        \begin{column}{0.48\textwidth}
            \inputminted[firstline=21,lastline=35]{cpp}{code/oop_example_large_project.cpp}
        \end{column}
    \end{columns}
\end{frame}

\begin{frame}[fragile]{面向对象编程示例-大型项目(游戏引擎)}
    \begin{columns}
        \begin{column}{0.48\textwidth}
            \inputminted[firstline=36,lastline=57]{cpp}{code/oop_example_large_project.cpp}
        \end{column}
        \begin{column}{0.48\textwidth}
            \inputminted[firstline=58,lastline=78]{cpp}{code/oop_example_large_project.cpp}
        \end{column}
    \end{columns}
\end{frame}

\tikzstyle{startstop} = [ellipse, minimum width=2.5cm, minimum height=1cm,text centered, draw=black, fill=red!20]
\tikzstyle{process} = [rectangle, minimum width=3cm, minimum height=1cm,text centered, draw=black, fill=blue!20]
\tikzstyle{arrow} = [thick,->,>=stealth]

\begin{frame}[fragile]{面向对象编程示例-大型项目(游戏引擎)}
    \begin{columns}
        \begin{column}{0.48\textwidth}
            \inputminted[firstline=80,lastline=90]{cpp}{code/oop_example_large_project.cpp}
        \end{column}
        \begin{column}{0.48\textwidth}
            \begin{center}
                \begin{tikzpicture}[node distance=0.40cm, scale=0.7, transform shape]

                % 节点
                \node (start) [startstop] {main() 启动};
                \node (createscene) [process, below=of start] {创建 Scene 对象};
                \node (addobjects) [process, below=of createscene] {向 Scene 添加 Player / Enemy};
                \node (loadscene) [process, below=of addobjects] {加载 Scene 到 GameEngine};
                \node (run) [process, below=of loadscene] {engine.run()};
                \node (update) [process, below=of run] {Scene.update() 遍历更新对象};
                \node (render) [process, below=of update] {Scene.render() 遍历渲染对象};
                \node (close) [startstop, below=of render] {引擎关闭并释放资源};

                % 箭头
                \draw [arrow] (start) -- (createscene);
                \draw [arrow] (createscene) -- (addobjects);
                \draw [arrow] (addobjects) -- (loadscene);
                \draw [arrow] (loadscene) -- (run);
                \draw [arrow] (run) -- (update);
                \draw [arrow] (update) -- (render);
                \draw [arrow] (render) -- (close);

                \end{tikzpicture}
            \end{center}
        \end{column}
    \end{columns}
\end{frame}

\section{泛型编程}
\begin{frame}{目录}
    \begin{multicols}{2}
        \tableofcontents[currentsection]
    \end{multicols}
\end{frame}

\begin{frame}{泛型编程概述}
    \begin{ytublock}{什么是泛型编程}
        \begin{itemize}
            \item 泛型编程是一种编程范式,它允许你编写与类型无关的代码。例如:与类型无关的算法。
            \item 泛型编程的目的是提高代码的复用性和可维护性。
            \item 泛型编程的实现方式是使用模板。
        \end{itemize}
    \end{ytublock}
    \begin{columns}
        \begin{column}{0.48\textwidth}
            \begin{ytublock}{泛型编程特点}
                \begin{itemize}
                    \item 类型无关的算法
                    \item 编译时多态
                    \item 代码复用
                    \item 性能优化
                    \item 类型安全
                \end{itemize}
            \end{ytublock}
        \end{column}
        \begin{column}{0.48\textwidth}
            \begin{ytublock}{Qt中的泛型应用}
                \begin{itemize}
                    \item QList等容器类
                    \item 算法模板
                    \item 智能指针
                    \item 类型推导
                \end{itemize}
            \end{ytublock}
        \end{column}
    \end{columns}
\end{frame}

\begin{frame}{模板——编译时多态}
    \begin{ytublock}{模板的编译时多态}
        \begin{itemize}
            \item 模板在编译阶段实例化,生成针对不同类型的高效代码,无运行时开销。
            \item 编译器会对模板参数进行类型检查,保证类型安全。
            \item 支持类型自动推导,简化模板的使用。
            \item 通过模板特化和重载,实现灵活的类型适配。
            \item C++20引入concepts(概念),可对模板参数类型进行约束,提升可读性和错误提示。
        \end{itemize}
    \end{ytublock}
\end{frame}

\begin{frame}[fragile]{泛型编程示例-函数模板-单参数模板}
    \begin{columns}
        \begin{column}{0.48\textwidth}
            \inputminted[firstline=1,lastline=18]{cpp}{code/gp_function_template_1.cpp}
        \end{column}
        \begin{column}{0.48\textwidth}
            \inputminted[firstline=20,lastline=40]{cpp}{code/gp_function_template_1.cpp}
        \end{column}
    \end{columns}
\end{frame}

\begin{frame}[fragile]{泛型编程示例-函数模板-多参数模板}
    \begin{columns}
        \begin{column}{0.48\textwidth}
            \inputminted[firstline=1,lastline=12]{cpp}{code/gp_function_template_2.cpp}
        \end{column}
        \begin{column}{0.48\textwidth}
            \inputminted[firstline=46,lastline=62]{cpp}{code/gp_function_template_2.cpp}
        \end{column}
    \end{columns}
\end{frame}

\begin{frame}[fragile]{泛型编程示例-函数模板-模板特化}
    \begin{columns}
        \begin{column}{0.48\textwidth}
            \inputminted[firstline=1,lastline=13]{cpp}{code/gp_function_template_3.cpp}
        \end{column}
        \begin{column}{0.48\textwidth}
            \inputminted[firstline=15,lastline=25]{cpp}{code/gp_function_template_3.cpp}
        \end{column}
    \end{columns}
\end{frame}

\begin{frame}[fragile]{泛型编程示例-函数模板-可变参数模板}
    \begin{columns}
        \begin{column}{0.48\textwidth}
            \inputminted[firstline=1,lastline=13]{cpp}{code/gp_function_template_4.cpp}
        \end{column}
        \begin{column}{0.48\textwidth}
            \inputminted[firstline=15,lastline=25]{cpp}{code/gp_function_template_4.cpp}
        \end{column}
    \end{columns}
\end{frame}

\begin{frame}[fragile]{泛型编程示例-函数模板-概念 (Concepts) - C++20}
    \begin{columns}
        \begin{column}{0.48\textwidth}
            \inputminted[firstline=1,lastline=13]{cpp}{code/gp_function_template_5.cpp}
        \end{column}
        \begin{column}{0.48\textwidth}
            \inputminted[firstline=15,lastline=32]{cpp}{code/gp_function_template_5.cpp}
        \end{column}
    \end{columns}
\end{frame}

\begin{frame}[fragile]{泛型编程示例-类模板-单参数模板}
    \begin{columns}
        \begin{column}{0.48\textwidth}
            \inputminted[firstline=1,lastline=20]{cpp}{code/gp_class_template_1.cpp}
        \end{column}
        \begin{column}{0.48\textwidth}
            \inputminted[firstline=22,lastline=40]{cpp}{code/gp_class_template_1.cpp}
        \end{column}
    \end{columns}
\end{frame}

\begin{frame}[fragile]{泛型编程示例-类模板-多参数模板}
    \begin{columns}
        \begin{column}{0.48\textwidth}
            \inputminted[firstline=1,lastline=21]{cpp}{code/gp_class_template_2.cpp}
        \end{column}
        \begin{column}{0.48\textwidth}
            \inputminted[firstline=23,lastline=40]{cpp}{code/gp_class_template_2.cpp}
        \end{column}
    \end{columns}
\end{frame}

\begin{frame}[fragile]{泛型编程示例-类模板-非类型模板参数(常量)}
    \begin{columns}
        \begin{column}{0.48\textwidth}
            \inputminted[firstline=1,lastline=15]{cpp}{code/gp_class_template_3.cpp}
        \end{column}
        \begin{column}{0.48\textwidth}
            \inputminted[firstline=17,lastline=26]{cpp}{code/gp_class_template_3.cpp}
        \end{column}
    \end{columns}
\end{frame}

\begin{frame}[fragile]{泛型编程示例-类模板-非类型模板参数(指针和引用)}
    \begin{columns}
        \begin{column}{0.48\textwidth}
            \inputminted[firstline=1,lastline=20]{cpp}{code/gp_class_template_4.cpp}
        \end{column}
        \begin{column}{0.48\textwidth}
            \inputminted[firstline=21,lastline=40]{cpp}{code/gp_class_template_4.cpp}
        \end{column}
    \end{columns}
\end{frame}

\begin{frame}[fragile]{泛型编程示例-类模板-模板模板参数(容器)}
    \begin{columns}
        \begin{column}{0.48\textwidth}
            \inputminted[firstline=1,lastline=20]{cpp}{code/gp_class_template_5.cpp}
        \end{column}
        \begin{column}{0.48\textwidth}
            \inputminted[firstline=21,lastline=40]{cpp}{code/gp_class_template_5.cpp}
        \end{column}
    \end{columns}
\end{frame}

\begin{frame}[fragile]{泛型编程示例-类模板-默认模板参数}
    \begin{columns}
        \begin{column}{0.48\textwidth}
            \inputminted[firstline=1,lastline=16]{cpp}{code/gp_class_template_6.cpp}
        \end{column}
        \begin{column}{0.48\textwidth}
            \inputminted[firstline=17,lastline=32]{cpp}{code/gp_class_template_6.cpp}
        \end{column}
    \end{columns}
\end{frame}

\begin{frame}[fragile]{泛型编程示例-类模板-成员模板}
    \begin{columns}
        \begin{column}{0.48\textwidth}
            \inputminted[firstline=1,lastline=16]{cpp}{code/gp_class_template_7.cpp}
        \end{column}
        \begin{column}{0.48\textwidth}
            \inputminted[firstline=18,lastline=36]{cpp}{code/gp_class_template_7.cpp}
        \end{column}
    \end{columns}
\end{frame}

\begin{frame}[fragile]{泛型编程示例-类模板-继承模板}
    \begin{columns}
        \begin{column}{0.48\textwidth}
            \inputminted[firstline=1,lastline=18]{cpp}{code/gp_class_template_8.cpp}
        \end{column}
        \begin{column}{0.48\textwidth}
            \inputminted[firstline=19,lastline=35]{cpp}{code/gp_class_template_8.cpp}
        \end{column}
    \end{columns}
\end{frame}

\begin{frame}[fragile]{泛型编程示例-类模板-SFINAE (Substitution Failure Is Not An Error)}
    \begin{ytublock}{SFINAE (Substitution Failure Is Not An Error)}
        在编译时,如果模板参数不满足条件,编译器会尝试其他可能的模板参数。如果所有可能的模板参数都失败,编译器会报错。
    \end{ytublock}
    \begin{columns}
        \begin{column}{0.48\textwidth}
            \inputminted[firstline=1,lastline=13]{cpp}{code/gp_class_template_9.cpp}
        \end{column}
        \begin{column}{0.48\textwidth}
            \inputminted[firstline=14,lastline=26]{cpp}{code/gp_class_template_9.cpp}
        \end{column}
    \end{columns}
\end{frame}

\begin{frame}[fragile]{泛型编程示例-类模板-变长模板参数(C++20)}
    \begin{columns}
        \begin{column}{0.48\textwidth}
            \inputminted[firstline=1,lastline=18]{cpp}{code/gp_class_template_10.cpp}
        \end{column}
        \begin{column}{0.48\textwidth}
            \inputminted[firstline=19,lastline=40]{cpp}{code/gp_class_template_10.cpp}
        \end{column}
    \end{columns}
\end{frame}

\begin{frame}{模板元编程概述}
    \begin{ytublock}{模板元编程(Template Metaprogramming)}
        \begin{itemize}
            \item 利用C++模板机制,在\textbf{编译期}进行类型和常量的计算与推导。
            \item 代码在编译阶段生成,\textbf{无运行时开销},提升性能。
            \item 可实现\textbf{编译时多态},如类型选择、特化、SFINAE等。
            \item 支持\textbf{递归}与\textbf{条件分支},可实现复杂的编译期逻辑。
            \item 常用于类型萃取(type traits)、静态断言、静态循环、类型变换等高级用法。
            \item 现代C++(C++11及以后)引入了\texttt{constexpr}、\texttt{if constexpr}、\texttt{std::integral\_constant}等工具,极大简化了模板元编程。
        \end{itemize}
    \end{ytublock}
    \begin{ytublock}{典型应用场景}
        \begin{itemize}
            \item 静态断言与类型检查(如\texttt{static\_assert}、SFINAE、concepts)
            \item 编译期常量计算(如阶乘、斐波那契数列等)
            \item 类型萃取与类型变换(如\texttt{std::remove\_const}、\texttt{std::is\_same}等)
            \item 优化泛型代码的分支与选择(如根据类型选择不同实现)
        \end{itemize}
    \end{ytublock}
\end{frame}

\begin{frame}[fragile]{泛型编程示例-模板元编程-类型萃取}
    \begin{columns}
        \begin{column}{0.48\textwidth}
            \inputminted[firstline=1,lastline=14]{cpp}{code/gp_template_metaprogramming_1.cpp}
        \end{column}
        \begin{column}{0.48\textwidth}
            \inputminted[firstline=16,lastline=28]{cpp}{code/gp_template_metaprogramming_1.cpp}
        \end{column}
    \end{columns}
\end{frame}

\begin{frame}[fragile]{泛型编程示例-模板元编程-静态断言}
    \begin{columns}
        \begin{column}{0.48\textwidth}
            \inputminted[firstline=1,lastline=19]{cpp}{code/gp_template_metaprogramming_2.cpp}
        \end{column}
        \begin{column}{0.48\textwidth}
            \inputminted[firstline=21,lastline=38]{cpp}{code/gp_template_metaprogramming_2.cpp}
        \end{column}
    \end{columns}
\end{frame}

\begin{frame}[fragile]{泛型编程示例-模板元编程-编译期常量计算}
    \begin{columns}
        \begin{column}{0.48\textwidth}
            \inputminted[firstline=1,lastline=16]{cpp}{code/gp_template_metaprogramming_3.cpp}
        \end{column}
        \begin{column}{0.48\textwidth}
            \inputminted[firstline=18,lastline=34]{cpp}{code/gp_template_metaprogramming_3.cpp}
        \end{column}
    \end{columns}
\end{frame}

\begin{frame}[fragile]{泛型编程示例-模板元编程-类型变换}
    \begin{columns}
        \begin{column}{0.48\textwidth}
            \inputminted[firstline=1,lastline=12]{cpp}{code/gp_template_metaprogramming_4.cpp}
        \end{column}
        \begin{column}{0.48\textwidth}
            \inputminted[firstline=13,lastline=24]{cpp}{code/gp_template_metaprogramming_4.cpp}
        \end{column}
    \end{columns}
\end{frame}

\begin{frame}[fragile]{泛型编程示例-模板元编程-编译期循环}
    \begin{columns}
        \begin{column}{0.48\textwidth}
            \inputminted[firstline=1,lastline=17]{cpp}{code/gp_template_metaprogramming_5.cpp}
        \end{column}
        \begin{column}{0.48\textwidth}
            \inputminted[firstline=19,lastline=35]{cpp}{code/gp_template_metaprogramming_5.cpp}
        \end{column}
    \end{columns}
\end{frame}

\begin{frame}[fragile]{泛型编程示例-模板元编程-编译时多态}
    \begin{columns}
        \begin{column}{0.48\textwidth}
            \inputminted[firstline=1,lastline=18]{cpp}{code/gp_template_metaprogramming_6.cpp}
        \end{column}
        \begin{column}{0.48\textwidth}
            \inputminted[firstline=20,lastline=32]{cpp}{code/gp_template_metaprogramming_6.cpp}
        \end{column}
    \end{columns}
\end{frame}

\begin{frame}[fragile]{泛型编程示例-模板元编程-constexpr}
    \begin{columns}
        \begin{column}{0.48\textwidth}
            \inputminted[firstline=1,lastline=16]{cpp}{code/gp_template_metaprogramming_7.cpp}
        \end{column}
        \begin{column}{0.48\textwidth}
            \inputminted[firstline=18,lastline=36]{cpp}{code/gp_template_metaprogramming_7.cpp}
        \end{column}
    \end{columns}
\end{frame}

\begin{frame}[fragile]{泛型编程示例-模板元编程-if constexpr}
    \begin{columns}
        \begin{column}{0.48\textwidth}
            \inputminted[firstline=1,lastline=15]{cpp}{code/gp_template_metaprogramming_8.cpp}
        \end{column}
        \begin{column}{0.48\textwidth}
            \inputminted[firstline=16,lastline=35]{cpp}{code/gp_template_metaprogramming_8.cpp}
        \end{column}
    \end{columns}
\end{frame}

\begin{frame}[fragile]{泛型编程示例-模板元编程-std::integral\_constant}
    \begin{columns}
        \begin{column}{0.48\textwidth}
            \inputminted[firstline=1,lastline=16]{cpp}{code/gp_template_metaprogramming_9.cpp}
        \end{column}
        \begin{column}{0.48\textwidth}
            \inputminted[firstline=17,lastline=32]{cpp}{code/gp_template_metaprogramming_9.cpp}
        \end{column}
    \end{columns}
\end{frame}

\begin{frame}{泛型编程小结}
    \begin{itemize}
        \item 泛型编程(Generic Programming)是一种以抽象和复用为核心思想的高级编程范式,强调算法与数据结构的分离,使代码能够适用于多种类型。
        \item C++通过模板(Template)机制实现泛型编程,支持函数模板、类模板、变量模板,以及模板特化、偏特化等高级特性。
        \item 模板不仅支持类型参数,还可结合非类型模板参数、模板模板参数,实现更高层次的抽象与灵活性。
        \item 泛型编程推动了C++标准库(如STL容器、算法、迭代器等)的设计,极大提升了代码复用性和可维护性。
        \item 现代C++泛型编程还包括模板元编程(Template Metaprogramming)、类型萃取(Type Traits)、SFINAE、constexpr等技术,实现编译期计算与类型推导,进一步提升类型安全与性能。
        \item 泛型编程的优势:\textbf{代码复用}、\textbf{类型安全}、\textbf{零开销抽象}、\textbf{性能优化},但也带来编译错误复杂、可读性下降等挑战。
    \end{itemize}
\end{frame}

\begin{frame}{泛型编程的典型应用场景}
    \begin{itemize}
        \item \textbf{泛型排序算法}:如 \texttt{std::sort}、HeapSort、QuickSort 等,支持任意可比较类型。
        \item \textbf{泛型容器与数据结构}:如 \texttt{std::vector}、\texttt{std::map}、\texttt{std::set} 等,适用于多种类型的数据存储与管理。
        \item \textbf{泛型数值与科学计算}:如矩阵运算、数值积分、线性代数库等,支持不同数值类型。
        \item \textbf{泛型算法库}:如 \texttt{STL} 算法(查找、变换、归约等),可作用于任意容器和类型。
        \item \textbf{泛型AI与优化算法}:如神经网络、支持向量机、遗传算法、粒子群优化等,算法结构与数据类型解耦。
        \item \textbf{泛型图像与信号处理}:如图像滤波、特征提取、信号变换等,支持不同像素/信号类型。
    \end{itemize}
\end{frame}

\section{函数式编程}
\begin{frame}{目录}
    \begin{multicols}{2}
        \tableofcontents[currentsection]
    \end{multicols}
\end{frame}

\begin{frame}{函数式编程概述}
    \begin{ytublock}{什么是函数式编程}
        \begin{itemize}
            \item 函数式编程(Functional Programming)是一种编程范式,它将计算视为数学函数的求值过程。
            \item 函数式编程的代码通常是不可变的(Immutable),避免副作用(Side Effect),并且函数本身也是第一类对象。
            \item 函数式编程强调函数组合(Function Composition)和纯函数(Pure Function),避免状态变化和可变共享状态。
        \end{itemize}
    \end{ytublock}
    \begin{columns}
        \begin{column}{0.48\textwidth}
            \begin{ytublock}{函数式编程特点}
                \begin{itemize}
                    \item 函数是一等公民
                    \item 不可变性(Immutability)
                    \item 无副作用
                    \item 高阶函数
                    \item 递归和组合
                \end{itemize}
            \end{ytublock}
        \end{column}
        \begin{column}{0.48\textwidth}
            \begin{ytublock}{C++中的函数式特性}
                \begin{itemize}
                    \item Lambda表达式
                    \item std::function
                    \item 函数对象
                    \item 算法库
                    \item 智能指针
                \end{itemize}
            \end{ytublock}
        \end{column}
    \end{columns}
\end{frame}

\begin{frame}[fragile]{函数式编程示例-纯函数}
    \begin{columns}
        \begin{column}{0.48\textwidth}
            \inputminted[firstline=1,lastline=11]{cpp}{code/functional_programming_0.cpp}
        \end{column}
        \begin{column}{0.48\textwidth}
            \inputminted[firstline=13,lastline=25]{cpp}{code/functional_programming_0.cpp}
        \end{column}
    \end{columns}
\end{frame}

\begin{frame}[fragile]{函数式编程示例-Lambda表达式}
    \begin{columns}
        \begin{column}{0.48\textwidth}
            \inputminted[firstline=1,lastline=19]{cpp}{code/functional_programming_1.cpp}
        \end{column}
        \begin{column}{0.48\textwidth}
            \inputminted[firstline=20,lastline=37]{cpp}{code/functional_programming_1.cpp}
        \end{column}
    \end{columns}
\end{frame}

\begin{frame}[fragile]{函数式编程示例-高阶函数-函数作为参数}
    \begin{columns}
        \begin{column}{0.48\textwidth}
            \inputminted[firstline=1,lastline=17]{cpp}{code/functional_programming_2.cpp}
        \end{column}
        \begin{column}{0.48\textwidth}
            \inputminted[firstline=19,lastline=39]{cpp}{code/functional_programming_2.cpp}
        \end{column}
    \end{columns}
\end{frame}

\begin{frame}[fragile]{函数式编程示例-高阶函数-函数作为返回值}
    \begin{columns}
        \begin{column}{0.48\textwidth}
            \inputminted[firstline=1,lastline=17]{cpp}{code/functional_programming_3.cpp}
        \end{column}
        \begin{column}{0.48\textwidth}
            \inputminted[firstline=19,lastline=35]{cpp}{code/functional_programming_3.cpp}
        \end{column}
    \end{columns}
\end{frame}

\begin{frame}[fragile]{函数式编程示例-递归与组合(一)}
    \begin{columns}
        \begin{column}{0.48\textwidth}
            \inputminted[firstline=1,lastline=17]{cpp}{code/functional_programming_4.cpp}
        \end{column}
        \begin{column}{0.48\textwidth}
            \inputminted[firstline=18,lastline=33]{cpp}{code/functional_programming_4.cpp}
        \end{column}
    \end{columns}
\end{frame}

\begin{frame}[fragile]{函数式编程示例-递归与组合(二)}
    \begin{columns}
        \begin{column}{0.48\textwidth}
            \inputminted[firstline=35,lastline=50]{cpp}{code/functional_programming_4.cpp}
        \end{column}
        \begin{column}{0.48\textwidth}
            \inputminted[firstline=51,lastline=63]{cpp}{code/functional_programming_4.cpp}
        \end{column}
    \end{columns}
\end{frame}

\begin{frame}[fragile]{函数式编程示例-递归与组合(三)}
    \begin{columns}
        \begin{column}{0.48\textwidth}
            \inputminted[firstline=1,lastline=16]{cpp}{code/functional_programming_7.cpp}
        \end{column}
        \begin{column}{0.48\textwidth}
            \inputminted[firstline=18,lastline=36]{cpp}{code/functional_programming_7.cpp}
        \end{column}
    \end{columns}
\end{frame}

\begin{frame}[fragile]{函数式编程示例-函数对象 (Functors)-函数作为对象}
    \begin{columns}
        \begin{column}{0.48\textwidth}
            \inputminted[firstline=1,lastline=17]{cpp}{code/functional_programming_5.cpp}
        \end{column}
        \begin{column}{0.48\textwidth}
            \inputminted[firstline=19,lastline=38]{cpp}{code/functional_programming_5.cpp}
        \end{column}
    \end{columns}
\end{frame}

\begin{frame}[fragile]{函数式编程示例-标准库算法(一)}
    \begin{columns}
        \begin{column}{0.48\textwidth}
            \inputminted[firstline=1,lastline=16]{cpp}{code/functional_programming_6.cpp}
        \end{column}
        \begin{column}{0.48\textwidth}
            \inputminted[firstline=18,lastline=31]{cpp}{code/functional_programming_6.cpp}
        \end{column}
    \end{columns}
\end{frame}

\begin{frame}[fragile]{函数式编程示例-标准库算法(二)}
    \begin{columns}
        \begin{column}{0.48\textwidth}
            \inputminted[firstline=33,lastline=47]{cpp}{code/functional_programming_6.cpp}
        \end{column}
        \begin{column}{0.48\textwidth}
            \inputminted[firstline=49,lastline=60]{cpp}{code/functional_programming_6.cpp}
        \end{column}
    \end{columns}
\end{frame}

\begin{frame}[fragile]{函数式编程示例-标准库算法(三)}
    \begin{columns}
        \begin{column}{0.48\textwidth}
            \inputminted[firstline=62,lastline=76]{cpp}{code/functional_programming_6.cpp}
        \end{column}
        \begin{column}{0.48\textwidth}
            \inputminted[firstline=78,lastline=91]{cpp}{code/functional_programming_6.cpp}
        \end{column}
    \end{columns}
\end{frame}

\begin{frame}[fragile]{函数式编程示例-标准库算法(四)}
    \begin{columns}
        \begin{column}{0.48\textwidth}
            \inputminted[firstline=93,lastline=107]{cpp}{code/functional_programming_6.cpp}
        \end{column}
        \begin{column}{0.48\textwidth}
            \inputminted[firstline=109,lastline=122]{cpp}{code/functional_programming_6.cpp}
        \end{column}
    \end{columns}
\end{frame}

\begin{frame}[fragile]{函数式编程示例-标准库算法(五)}
    \begin{columns}
        \begin{column}{0.48\textwidth}
            \inputminted[firstline=124,lastline=143]{cpp}{code/functional_programming_6.cpp}
        \end{column}
        \begin{column}{0.48\textwidth}
            \inputminted[firstline=145,lastline=159]{cpp}{code/functional_programming_6.cpp}
        \end{column}
    \end{columns}
\end{frame}

\begin{frame}[fragile]{函数式编程示例-标准库算法(六)}
    \begin{columns}
        \begin{column}{0.48\textwidth}
            \inputminted[firstline=161,lastline=174]{cpp}{code/functional_programming_6.cpp}
        \end{column}
        \begin{column}{0.48\textwidth}
            \inputminted[firstline=176,lastline=190]{cpp}{code/functional_programming_6.cpp}
        \end{column}
    \end{columns}
\end{frame}

\begin{frame}[fragile]{函数式编程示例-标准库算法(七)}
    \begin{columns}
        \begin{column}{0.48\textwidth}
            \inputminted[firstline=192,lastline=204]{cpp}{code/functional_programming_6.cpp}
        \end{column}
        \begin{column}{0.48\textwidth}
            \inputminted[firstline=206,lastline=213]{cpp}{code/functional_programming_6.cpp}
        \end{column}
    \end{columns}
\end{frame}

\begin{frame}[fragile]{函数式编程示例-标准库算法(八)}
    \begin{columns}
        \begin{column}{0.48\textwidth}
            \inputminted[firstline=215,lastline=223]{cpp}{code/functional_programming_6.cpp}
        \end{column}
        \begin{column}{0.48\textwidth}
            \inputminted[firstline=225,lastline=233]{cpp}{code/functional_programming_6.cpp}
        \end{column}
    \end{columns}
\end{frame}

\begin{frame}[fragile]{函数式编程示例-模式匹配}
    \begin{columns}
        \begin{column}{0.48\textwidth}
            \inputminted[firstline=1,lastline=18]{cpp}{code/functional_programming_8.cpp}
        \end{column}
        \begin{column}{0.48\textwidth}
            \inputminted[firstline=19,lastline=33]{cpp}{code/functional_programming_8.cpp}
        \end{column}
    \end{columns}
\end{frame}

\begin{frame}[fragile]{函数式编程示例-惰性求值}
    \begin{columns}
        \begin{column}{0.48\textwidth}
            \inputminted[firstline=1,lastline=21]{cpp}{code/functional_programming_9.cpp}
        \end{column}
        \begin{column}{0.48\textwidth}
            \inputminted[firstline=22,lastline=37]{cpp}{code/functional_programming_9.cpp}
        \end{column}
    \end{columns}
\end{frame}

\begin{frame}{函数式编程小结}
    \begin{itemize}
        \item 函数式编程是一种以数学函数为核心思想的编程范式,将计算过程抽象为函数的组合与变换。
        \item 代码强调不可变性(Immutable),尽量避免副作用(Side Effect),函数作为“第一类对象”可以像变量一样传递和返回。
        \item 推崇纯函数(Pure Function)和函数组合(Function Composition),减少状态变化和共享可变状态,使程序更易于推理和测试。
        \item 常见特性包括高阶函数、递归、惰性求值、模式匹配等,提升代码的表达力和可维护性。
    \end{itemize}
    \begin{ytublock}{典型应用场景}
        \begin{columns}
            \begin{column}{0.48\textwidth}
                \begin{itemize}
                    \item 数学计算和科学计算
                    \item 数据处理和转换
                    \item 并行和并发编程
                    \item 事件驱动编程
                \end{itemize}
            \end{column}
            \begin{column}{0.48\textwidth}
                \begin{itemize}
                    \item 状态机和工作流
                    \item 编译器和解释器
                    \item 函数组合和管道
                    \item 测试和验证
                \end{itemize}
            \end{column}
        \end{columns}
    \end{ytublock}
\end{frame}

\section{其他编程范式}
\begin{frame}{目录}
    \begin{multicols}{2}
        \tableofcontents[currentsection]
    \end{multicols}
\end{frame}

\begin{frame}{事件驱动编程}
    \begin{ytublock}{什么是事件驱动编程}
        \begin{itemize}
            \item 事件驱动编程(Event-Driven Programming)是一种编程范式,它将程序的执行过程看作是事件的响应和处理。
            \item 事件驱动编程的代码通常是异步的(Asynchronous),事件的触发和处理是分离的。
            \item 事件驱动编程强调事件的响应和处理,而不是按照顺序执行。
        \end{itemize}
    \end{ytublock}
    \begin{columns}
        \begin{column}{0.48\textwidth}
            \begin{ytublock}{事件驱动编程特点}
                \begin{itemize}
                    \item 以事件为中心
                    \item 异步处理
                    \item 松耦合设计
                    \item 响应式编程
                    \item 用户交互驱动
                \end{itemize}
            \end{ytublock}
        \end{column}
        \begin{column}{0.48\textwidth}
            \begin{ytublock}{Qt中的事件驱动}
                \begin{itemize}
                    \item 信号槽机制
                    \item 事件循环
                    \item 事件过滤器
                    \item 定时器事件
                \end{itemize}
            \end{ytublock}
        \end{column}
    \end{columns}
\end{frame}

\begin{frame}[fragile]{事件驱动编程示例}
    \begin{columns}
        \begin{column}{0.48\textwidth}
            \inputminted[firstline=1,lastline=18]{cpp}{code/event_driven_programming.cpp}
        \end{column}
        \begin{column}{0.48\textwidth}
            \inputminted[firstline=19,lastline=32]{cpp}{code/event_driven_programming.cpp}
        \end{column}
    \end{columns}
\end{frame}

\begin{frame}{声明式编程}
    \begin{ytublock}{什么是声明式编程}
        \begin{itemize}
            \item 声明式编程(Declarative Programming)是一种编程范式,它将程序的执行过程看作是声明的执行。
            \item 声明式编程的代码通常是声明式的,而不是命令式的。
            \item 声明式编程强调声明的执行,而不是按照顺序执行。
        \end{itemize}
    \end{ytublock}
    \begin{columns}
        \begin{column}{0.48\textwidth}
            \begin{ytublock}{声明式编程特点}
                \begin{itemize}
                    \item 描述"做什么"而非"怎么做"
                    \item 关注结果而非过程
                    \item 更接近自然语言
                    \item 减少副作用
                    \item 提高可读性
                \end{itemize}
            \end{ytublock}
        \end{column}
        \begin{column}{0.48\textwidth}
            \begin{ytublock}{Qt中的声明式编程}
                \begin{itemize}
                    \item QML语言
                    \item 属性绑定
                    \item 状态机
                    \item 样式表
                \end{itemize}
            \end{ytublock}
        \end{column}
    \end{columns}
\end{frame}

\begin{frame}[fragile]{声明式编程示例}
    \begin{columns}
        \begin{column}{0.48\textwidth}
            \inputminted[firstline=1,lastline=20]{cpp}{code/declarative_programming.cpp}
        \end{column}
        \begin{column}{0.48\textwidth}
            \inputminted[firstline=21,lastline=35]{cpp}{code/declarative_programming.cpp}
        \end{column}
    \end{columns}
\end{frame}

\begin{frame}{组件式编程}
    \begin{ytublock}{什么是组件式编程}
        \begin{itemize}
            \item 组件式编程(Component-Based Programming)是一种编程范式,它将程序的执行过程看作是组件的组合和交互。
            \item 组件式编程的代码通常是模块化的,组件之间是松耦合的。
            \item 组件式编程强调组件的复用和组合,而不是按照顺序执行。
        \end{itemize}
    \end{ytublock}
    \begin{columns}
        \begin{column}{0.48\textwidth}
            \begin{ytublock}{组件式编程特点}
                \begin{itemize}
                    \item 模块化设计
                    \item 可重用组件
                    \item 松耦合架构
                    \item 标准化接口
                    \item 组合优于继承
                \end{itemize}
            \end{ytublock}
        \end{column}
        \begin{column}{0.48\textwidth}
            \begin{ytublock}{Qt中的组件系统}
                \begin{itemize}
                    \item Qt Widgets
                    \item Qt Quick Components
                    \item 插件系统
                    \item 自定义组件
                \end{itemize}
            \end{ytublock}
        \end{column}
    \end{columns}
\end{frame}

\begin{frame}[fragile]{组件式编程示例-组件接口与实现}
    \begin{columns}
        \begin{column}{0.48\textwidth}
            \inputminted[firstline=1,lastline=17]{cpp}{code/component_programming_1.cpp}
        \end{column}
        \begin{column}{0.48\textwidth}
            \inputminted[firstline=18,lastline=33]{cpp}{code/component_programming_1.cpp}
        \end{column}
    \end{columns}
\end{frame}

\begin{frame}[fragile]{组件式编程示例-组件管理器}
    \begin{columns}
        \begin{column}{0.48\textwidth}
            \inputminted[firstline=24,lastline=40]{cpp}{code/component_programming_2.cpp}
        \end{column}
        \begin{column}{0.48\textwidth}
            \inputminted[firstline=41,lastline=55]{cpp}{code/component_programming_2.cpp}
        \end{column}
    \end{columns}
\end{frame}

\begin{frame}[fragile]{组件式编程示例-Qt组件系统}
    \begin{columns}
        \begin{column}{0.48\textwidth}
            \inputminted[firstline=7,lastline=23]{cpp}{code/component_programming_3.cpp}
        \end{column}
        \begin{column}{0.48\textwidth}
            \inputminted[firstline=24,lastline=43]{cpp}{code/component_programming_3.cpp}
        \end{column}
    \end{columns}
\end{frame}

\section{总结}
\begin{frame}{目录}
    \begin{multicols}{2}
        \tableofcontents[currentsection]
    \end{multicols}
\end{frame}

\begin{frame}{总结}
    \begin{columns}
        \begin{column}{0.48\textwidth}
            \begin{ytublock}{本章要点}
                \begin{itemize}
                    \item 理解各种编程范式的特点和应用
                    \item 掌握过程式编程的基本方法
                    \item 学会面向对象编程的设计模式
                    \item 理解泛型编程的优势和使用
                    \item 掌握函数式编程的核心概念
                    \item 学会事件驱动编程的实现
                    \item 理解声明式编程的思维方式
                    \item 掌握组件式编程的架构设计
                \end{itemize}
            \end{ytublock}
        \end{column}
        \begin{column}{0.48\textwidth}
            \begin{ytublock}{实践建议}
                \begin{itemize}
                    \item 根据问题选择合适的编程范式
                    \item 组合使用多种范式
                    \item 注重代码的可读性和维护性
                    \item 在实践中不断改进和优化
                \end{itemize}
            \end{ytublock}
        \end{column}
    \end{columns}
\end{frame}

\end{document}