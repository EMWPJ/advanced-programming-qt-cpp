\documentclass[UTF8,aspectratio=169]{beamer}

% 基本包
\usepackage[UTF8]{ctex}
\usepackage{graphicx}
\usepackage{amsmath}
\usepackage{amsfonts}
\usepackage{amssymb}
\usepackage{listings}
\usepackage{xcolor}
\usepackage{hyperref}
\usepackage{booktabs}
\usepackage{multirow}
\usepackage{float}
\usepackage{tikz}
\usepackage{enumitem}

% Beamer主题设置
\usetheme{Madrid}
\usecolortheme{default}
\setbeamertemplate{navigation symbols}{}

% 减少页面间距
\setbeamertemplate{itemize items}[circle]
\setbeamertemplate{enumerate items}[default]
\setlength{\itemsep}{0.5em}
\setlength{\parskip}{0.5em}

% 页码设置
\setbeamertemplate{footline}[frame number]

% 水印设置
\setbeamertemplate{background}{
    \begin{tikzpicture}[remember picture,overlay]
        \node[rotate=-45,scale=0.8,opacity=0.1,color=gray]
             at ([xshift=0.5cm,yshift=0.5cm]current page.south west)
             {\large\textbf{WPJ}};
    \end{tikzpicture}
}

% 代码高亮设置
\lstset{
    language=C++,
    basicstyle=\ttfamily\tiny,
    keywordstyle=\color{blue},
    commentstyle=\color{green!60!black},
    stringstyle=\color{red},
    numbers=left,
    numberstyle=\tiny,
    numbersep=5pt,
    frame=single,
    breaklines=true,
    showstringspaces=false,
    escapechar=@,
    linewidth=\textwidth,
    xleftmargin=5pt,
    xrightmargin=5pt
}

% 自定义颜色
\definecolor{qtgreen}{RGB}{41,128,185}
\definecolor{qtblue}{RGB}{52,73,94}

% 文档信息
\title{第6章:Qt高级特性与实战}
\subtitle{高等程序设计 - Qt/C++}
\author{课程讲义}
\institute{高等程序设计课程}
\date{\today}

\begin{document}

% 标题页
\begin{frame}
    \titlepage
\end{frame}

% 目录页
\begin{frame}{目录}
    \tableofcontents
\end{frame}

\section{Qt多线程编程}

\begin{frame}{Qt多线程概述}
    \begin{block}{Qt多线程特点}
        \begin{itemize}
            \item 基于QThread类
            \item 信号槽跨线程通信
            \item 线程安全的事件循环
            \item 自动内存管理
            \item 与Qt事件系统集成
        \end{itemize}
    \end{block}

    \begin{block}{多线程应用场景}
        \begin{itemize}
            \item 耗时计算
            \item 网络请求
            \item 文件I/O操作
            \item 数据处理
            \item 实时更新UI
        \end{itemize}
    \end{block}
\end{frame}

\begin{frame}[fragile]{Qt多线程示例}
    \begin{lstlisting}
#include <QThread>
#include <QObject>
#include <QDebug>

// worker thread class
class Worker : public QObject
{
    Q_OBJECT

public slots:
    void doWork() {
        qDebug() << "Worker thread:" << QThread::currentThread();

        // simulate time-consuming work
        for (int i = 0; i < 10; ++i) {
            QThread::msleep(100);
            emit progressUpdated(i * 10);
        }

        emit workFinished();
    }

signals:
    void progressUpdated(int value);
    void workFinished();
};

// main window class
class MainWindow : public QMainWindow
{
    Q_OBJECT

private slots:
    void startWork() {
        m\_worker->doWork();
    }

    void onProgressUpdated(int value) {
        m\_progressBar->setValue(value);
    }

    void onWorkFinished() {
        qDebug() << "Work finished";
    }

private:
    void setupThreading() {
        // create worker thread
        m\_thread = new QThread(this);
        m\_worker = new Worker();
        m\_worker->moveToThread(m\_thread);

        // connect signal slot
        connect(m\_worker, &Worker::progressUpdated,
                this, &MainWindow::onProgressUpdated);
        connect(m\_worker, &Worker::workFinished,
                this, &MainWindow::onWorkFinished);

        // start thread
        m\_thread->start();
    }

private:
    QThread *m\_thread;
    Worker *m\_worker;
    QProgressBar *m\_progressBar;
};
    \end{lstlisting}
\end{frame}

\section{Qt绘图系统}

\begin{frame}{Qt绘图系统概述}
    \begin{block}{Qt绘图组件}
        \begin{itemize}
            \item QPainter - 绘图引擎
            \item QPaintDevice - 绘图设备
            \item QPen - 画笔
            \item QBrush - 画刷
            \item QFont - 字体
            \item QPixmap - 位图
        \end{itemize}
    \end{block}

    \begin{block}{绘图应用}
        \begin{itemize}
            \item 自定义控件绘制
            \item 图表和图形
            \item 图像处理
            \item 动画效果
            \item 游戏开发
        \end{itemize}
    \end{block}
\end{frame}

\begin{frame}[fragile]{Qt绘图示例}
    \begin{lstlisting}
#include <QWidget>
#include <QPainter>
#include <QPen>
#include <QBrush>

// custom drawing widget
class DrawingWidget : public QWidget
{
    Q_OBJECT

protected:
    void paintEvent(QPaintEvent *event) override {
        QPainter painter(this);
        painter.setRenderHint(QPainter::Antialiasing);

        // set background
        painter.fillRect(rect(), QColor(240, 240, 240));

        // draw rectangle
        QPen pen(Qt::black, 2);
        QBrush brush(QColor(100, 150, 200));
        painter.setPen(pen);
        painter.setBrush(brush);
        painter.drawRect(10, 10, 80, 60);

        // draw ellipse
        pen.setColor(Qt::red);
        brush.setColor(QColor(200, 100, 100));
        painter.setPen(pen);
        painter.setBrush(brush);
        painter.drawEllipse(120, 10, 80, 60);

        // draw text
        QFont font("Arial", 16, QFont::Bold);
        painter.setFont(font);
        painter.setPen(Qt::blue);
        painter.drawText(10, 100, "Qt drawing system example");
    }
};
    \end{lstlisting}
\end{frame}

\section{QtCharts}

\begin{frame}{QtCharts概述}
    \begin{block}{QtCharts特性}
        \begin{itemize}
            \item 丰富的图表类型
            \item 交互式图表
            \item 实时数据更新
            \item 自定义样式
            \item 导出功能
        \end{itemize}
    \end{block}

    \begin{block}{图表类型}
        \begin{itemize}
            \item 折线图(Line Chart)
            \item 柱状图(Bar Chart)
            \item 饼图(Pie Chart)
            \item 散点图(Scatter Chart)
            \item 面积图(Area Chart)
        \end{itemize}
    \end{block}
\end{frame}

\begin{frame}[fragile]{QtCharts示例}
    \begin{lstlisting}
#include <QtCharts/QChartView>
#include <QtCharts/QLineSeries>
#include <QtCharts/QBarSeries>
#include <QtCharts/QBarSet>

QT_CHARTS_USE_NAMESPACE

class ChartDemo : public QMainWindow
{
    Q_OBJECT

public:
    ChartDemo(QWidget *parent = nullptr) : QMainWindow(parent) {
        setupCharts();
    }

private:
    void setupCharts() {
        // create chart
        m\_chart = new QChart();
        m\_chartView = new QChartView(m\_chart);
        m\_chartView->setRenderHint(QPainter::Antialiasing);

        // create data series
        m\_lineSeries = new QLineSeries();

        // add data
        for (int i = 0; i < 10; ++i) {
            m\_lineSeries->append(i, qrand() % 100);
        }

        // show chart
        m\_chart->addSeries(m\_lineSeries);
        m\_chart->setTitle("QtCharts example");
        m\_chart->createDefaultAxes();

        setCentralWidget(m\_chartView);
    }

private:
    QChartView *m\_chartView;
    QChart *m\_chart;
    QLineSeries *m\_lineSeries;
};
    \end{lstlisting}
\end{frame}

\section{Qt网络编程}

\begin{frame}{Qt网络编程概述}
    \begin{block}{Qt网络模块}
        \begin{itemize}
            \item QNetworkAccessManager - 网络访问管理器
            \item QNetworkRequest - 网络请求
            \item QNetworkReply - 网络响应
            \item QNetworkProxy - 网络代理
            \item QSslSocket - SSL套接字
        \end{itemize}
    \end{block}

    \begin{block}{支持的网络协议}
        \begin{itemize}
            \item HTTP/HTTPS
            \item FTP
            \item WebSocket
            \item TCP/UDP
            \item SSL/TLS
        \end{itemize}
    \end{block}
\end{frame}

\begin{frame}[fragile]{Qt网络编程示例}
    \begin{lstlisting}
#include <QNetworkAccessManager>
#include <QNetworkRequest>
#include <QNetworkReply>
#include <QJsonDocument>
#include <QJsonObject>

class NetworkDemo : public QMainWindow
{
    Q_OBJECT

private slots:
    void sendGetRequest() {
        QString url = "https://httpbin.org/get";

        QNetworkRequest request(QUrl(url));
        request.setHeader(QNetworkRequest::UserAgentHeader,
                         "Qt Network Demo/1.0");

        QNetworkReply *reply = m\_networkManager->get(request);

        connect(reply, &QNetworkReply::finished, [this, reply]() {
            if (reply->error() == QNetworkReply::NoError) {
                QString response = QString::fromUtf8(reply->readAll());
                qDebug() << "Response:" << response;
            } else {
                qDebug() << "Error:" << reply->errorString();
            }
            reply->deleteLater();
        });
    }

    void sendPostRequest() {
        QString url = "https://httpbin.org/post";

        QNetworkRequest request(QUrl(url));
        request.setHeader(QNetworkRequest::ContentTypeHeader,
                         "application/json");

        // create JSON data
        QJsonObject jsonData;
        jsonData["name"] = "Test user";
        jsonData["age"] = 25;

        QJsonDocument doc(jsonData);
        QByteArray data = doc.toJson();

        QNetworkReply *reply = m\_networkManager->post(request, data);

        connect(reply, &QNetworkReply::finished, [this, reply]() {
            if (reply->error() == QNetworkReply::NoError) {
                QString response = QString::fromUtf8(reply->readAll());
                qDebug() << "Response:" << response;
            } else {
                qDebug() << "Error:" << reply->errorString();
            }
            reply->deleteLater();
        });
    }

private:
    void setupNetwork() {
        m\_networkManager = new QNetworkAccessManager(this);
    }

private:
    QNetworkAccessManager *m\_networkManager;
};
    \end{lstlisting}
\end{frame}

\section{Qt数据库编程}

\begin{frame}{Qt数据库编程概述}
    \begin{block}{Qt数据库支持}
        \begin{itemize}
            \item QSqlDatabase - 数据库连接
            \item QSqlQuery - SQL查询
            \item QSqlTableModel - 表格模型
            \item QSqlRelationalTableModel - 关系表格模型
            \item QSqlQueryModel - 查询模型
        \end{itemize}
    \end{block}

    \begin{block}{支持的数据库}
        \begin{itemize}
            \item SQLite
            \item MySQL
            \item PostgreSQL
            \item Oracle
            \item Microsoft SQL Server
        \end{itemize}
    \end{block}
\end{frame}

\begin{frame}[fragile]{Qt数据库编程示例}
    \begin{lstlisting}
#include <QSqlDatabase>
#include <QSqlQuery>
#include <QSqlError>
#include <QSqlTableModel>
#include <QTableView>

class DatabaseDemo : public QMainWindow
{
    Q_OBJECT

public:
    DatabaseDemo(QWidget *parent = nullptr) : QMainWindow(parent) {
        setupDatabase();
        setupModel();
    }

private slots:
    void addRecord() {
        QString name = "New user";
        QString email = "newuser@example.com";
        int age = 25;

        QSqlQuery query;
        query.prepare("INSERT INTO users (name, email, age) VALUES (?, ?, ?)");
        query.addBindValue(name);
        query.addBindValue(email);
        query.addBindValue(age);

        if (query.exec()) {
            m\_model->select(); // refresh model
            qDebug() << "Record added successfully";
        } else {
            qDebug() << "Add failed:" << query.lastError().text();
        }
    }

private:
    void setupDatabase() {
        // create SQLite database connection
        QSqlDatabase db = QSqlDatabase::addDatabase("QSQLITE");
        db.setDatabaseName("users.db");

        if (!db.open()) {
            qDebug() << "Cannot open database:" << db.lastError().text();
            return;
        }

        // create table
        QSqlQuery query;
        query.exec("CREATE TABLE IF NOT EXISTS users ("
                  "id INTEGER PRIMARY KEY AUTOINCREMENT,"
                  "name TEXT NOT NULL,"
                  "email TEXT NOT NULL,"
                  "age INTEGER)");

        // insert example data
        query.exec("INSERT OR IGNORE INTO users (name, email, age) VALUES "
                  "('Zhang San', 'zhangsan@example.com', 25),"
                  "('Li Si', 'lisi@example.com', 30)");
    }

    void setupModel() {
        m\_model = new QSqlTableModel(this);
        m\_model->setTable("users");
        m\_model->select();

        // set table header
        m\_model->setHeaderData(0, Qt::Horizontal, "ID");
        m\_model->setHeaderData(1, Qt::Horizontal, "Name");
        m\_model->setHeaderData(2, Qt::Horizontal, "Email");
        m\_model->setHeaderData(3, Qt::Horizontal, "Age");

        m\_tableView = new QTableView(this);
        m\_tableView->setModel(m\_model);
        setCentralWidget(m\_tableView);
    }

private:
    QSqlTableModel *m\_model;
    QTableView *m\_tableView;
};
    \end{lstlisting}
\end{frame}

\section{总结}

\begin{frame}{总结}
    \begin{block}{本章要点}
        \begin{itemize}
            \item 掌握Qt多线程编程
            \item 学会使用Qt绘图系统
            \item 理解QtCharts的使用
            \item 掌握Qt网络编程
            \item 学会Qt数据库编程
            \item 理解Qt高级特性的应用
            \item 能够开发完整的Qt应用程序
        \end{itemize}
    \end{block}

    \begin{block}{实践建议}
        \begin{itemize}
            \item 多练习实际项目开发
            \item 关注性能优化
            \item 学习最佳实践
            \item 参与开源项目
            \item 持续学习新技术
        \end{itemize}
    \end{block}
\end{frame}

\end{document}