\documentclass[UTF8,aspectratio=169]{beamer}



% 基本包
\usepackage[UTF8]{ctex}
\usepackage{graphicx}
\usepackage{amsmath}
\usepackage{amsfonts}
\usepackage{amssymb}
% \usepackage{listings}  % 已替换为minted
\usepackage{xcolor}
\usepackage{hyperref}
\usepackage{booktabs}
\usepackage{multirow}
\usepackage{multicol}
\usepackage{float}
\usepackage{tikz}
\usetikzlibrary{positioning,shapes,arrows,fit,backgrounds}
\usepackage{pgfplots}
\pgfplotsset{compat=1.18}
\usepackage{minted}
\usepackage{fontspec}
\usepackage[most]{tcolorbox}

% Beamer主题设置
\usetheme{Madrid}
\usecolortheme{whale}

% 校徽设置
\logo{
  \IfFileExists{../长江大学校徽.pdf}{
    \begin{tikzpicture}[remember picture,overlay]
      \node[anchor=north east, xshift=-0.2mm, yshift=-0.2mm] at (current page.north east) {
        \includegraphics[height=1.0cm]{../长江大学校徽.pdf}
      };
    \end{tikzpicture}
  }{
    \begin{tikzpicture}[remember picture,overlay]
      \node[anchor=north east, xshift=-3mm, yshift=-3mm] at (current page.north east) {
        \textcolor{red}{\tiny [校徽文件未找到]}
      };
    \end{tikzpicture}
  }
}

% ===== 使用推荐的 font themes =====
\usefonttheme{professionalfonts}  % 允许自定义字体

% 自定义frame标题栏,也缩短长度留出logo空间
% 设置标题栏背景颜色为淡蓝色
\definecolor{frametitlebg}{RGB}{200,215,250} % 淡蓝色,可根据需要调整
\setbeamercolor{frametitle}{bg=frametitlebg, fg=ytublue!80!black}
\setbeamertemplate{frametitle}{
  \ifbeamercolorempty[bg]{frametitle}{}{\nointerlineskip}%
  \begin{tcolorbox}[
    enhanced,
    width=0.90\paperwidth,
    height=2.5ex,
    colback=frametitlebg,
    colframe=frametitlebg,
    boxrule=0pt,
    left=0pt,
    right=0pt,
    top=1pt,
    bottom=0pt,
    boxsep=0pt,
    before skip=0pt,
    after skip=0.1em,  % 减少标题栏和内容之间的间距
  ]
  \vspace{0.2ex} % 减少标题文字上方的空白
  \usebeamerfont{frametitle}\textcolor{ytublue!80!black}{\hspace{1em}\insertframetitle}
  \end{tcolorbox}
}

% ===== 设置现代字体 =====
\setsansfont{Source Sans Pro}     % 正文字体
\setmonofont{Source Code Pro}[Scale=0.9]  % 代码字体,稍微缩小一点

\setbeamertemplate{navigation symbols}{}

% 减少页面间距
\setbeamertemplate{itemize items}[circle]
\setbeamertemplate{enumerate items}[default]
\setlength{\itemsep}{0.1em}
\setlength{\parskip}{0.1em}

% 页码设置
\setbeamertemplate{footline}[frame number]

% 定义流程图样式
\tikzset{
    block/.style = {rectangle, draw, fill=blue!10,
        minimum width=6em, align=center, rounded corners, minimum height=3em},
    line/.style = {draw, -latex'}
}
% 水印设置
\setbeamertemplate{background}{
    \begin{tikzpicture}[remember picture,overlay]
        \node[rotate=-45,scale=0.8,opacity=0.1,color=gray]
             at ([xshift=0.5cm,yshift=0.5cm]current page.south west)
             {\large\textbf{WPJ}};
    \end{tikzpicture}
}

% 自定义颜色
\definecolor{qtgreen}{RGB}{41,128,185}
\definecolor{qtblue}{RGB}{52,73,94}

% 定义长江大学蓝主色调
\definecolor{ytublue}{RGB}{0,84,159}
% 统一block样式
\newtcolorbox{ytublock}[1]{
  colback=white,
  colframe=ytublue!80!black,
  colbacktitle=ytublue!20!white,
  coltitle=ytublue!80!black,
  title={#1},
  fonttitle=\bfseries,
  arc=3mm,
  boxrule=1pt,
  boxsep=1mm,
  left=2mm,
  right=2mm,
  top=0.5mm,
  bottom=0.5mm,
  before skip=3pt,
  after skip=3pt,
  enhanced,
  drop fuzzy shadow=ytublue!20!black
}

% 定义警告块样式
\newtcolorbox{ytualertblock}[1]{
  colback=white,
  colframe=red!80!black,
  colbacktitle=red!20!white,
  coltitle=red!80!black,
  title={#1},
  fonttitle=\bfseries,
  arc=3mm,
  boxrule=1.5pt,
  boxsep=1mm,
  left=2mm,
  right=2mm,
  top=0.5mm,
  bottom=0.5mm,
  before skip=3pt,
  after skip=3pt,
  enhanced,
  drop fuzzy shadow=red!20!black,
  overlay={
    \begin{tcbclipinterior}
      \fill[red!10!white] (interior.south west) rectangle (interior.north east);
    \end{tcbclipinterior}
  }
}

% cpp代码高亮设置
\setminted[cpp]{
    fontsize=\tiny,
    fontfamily=tt,             % 使用等宽字体
    linenos=true,
    frame=lines,               % 上下两条线,简洁清爽(比 tb 更现代)
    framesep=3mm,              % 内边距
    rulecolor=\color{blue!20}, % 线条颜色浅蓝,不刺眼
    bgcolor=blue!10,           % 浅蓝色背景
    baselinestretch=1.2,       % 行距稍大,更易读
    breaklines=true,
    breakautoindent=true,
    tabsize=4,
    xleftmargin=5mm,
    xrightmargin=5mm,
    numbersep=8pt,             % 行号与代码间距
    % ===== 其他美化 =====
    obeytabs=true,             % 尊重 tab 字符
    samepage=false,            % 允许跨页(重要!避免空白)
    escapeinside=||,           % 可在代码中使用 |LaTeX| 插入 LaTeX 命令
}

% 设置标题页颜色,与frame标题保持一致
\setbeamercolor{title}{bg=frametitlebg, fg=ytublue!80!black}
\setbeamercolor{subtitle}{bg=frametitlebg, fg=ytublue!70!black}
\setbeamercolor{author}{bg=frametitlebg, fg=ytublue!80!black}
\setbeamercolor{institute}{bg=frametitlebg, fg=ytublue!80!black}
\setbeamercolor{date}{bg=frametitlebg, fg=ytublue!80!black}

% 自定义标题页样式,全部内容同一个tcolorbox,居中排版,字体和间距区分
\setbeamertemplate{title page}{
  \vbox{}
  \begingroup
    \centering
    \begin{tcolorbox}[
      enhanced,
      width=0.92\paperwidth,
      colback=frametitlebg,
      colframe=frametitlebg,
      boxrule=0pt,
      left=0pt,
      right=0pt,
      top=4mm,
      bottom=4mm,
      boxsep=0pt,
      before skip=0pt,
      after skip=1.2em,
    ]
    % 标题
    {\centering
      {\fontsize{24pt}{27pt}\selectfont\textcolor{ytublue!80!black}{\inserttitle}\par}
      \vspace{1.2em}
      % 副标题
      {\fontsize{21pt}{24pt}\selectfont\textcolor{ytublue!70!black}{\insertsubtitle}\par}
      \vspace{2.0em}
      % 作者
      {\fontsize{12pt}{15pt}\selectfont\insertauthor\par}
      \vspace{0.7em}
      % 单位
      {\fontsize{12pt}{15pt}\selectfont\insertinstitute\par}
      \vspace{0.7em}
      % 日期
      {\fontsize{12pt}{15pt}\selectfont\insertdate\par}
    }
    \vspace{0.5em}
    \vfill
    \end{tcolorbox}
  \endgroup
}


% 文档信息
\title{高等程序设计 - Qt/C++}
\subtitle{第3章:Qt框架基础知识}
\author{王培杰}
\institute{长江大学地球物理与石油资源学院}
\date{\today}

\begin{document}

% 标题页
\begin{frame}
    \titlepage
\end{frame}

% 目录页
\begin{frame}{目录}
    \begin{multicols}{2}
        \tableofcontents[]
    \end{multicols}
\end{frame}


\section{Qt框架概述}
\begin{frame}{目录}
    \begin{multicols}{2}
        \tableofcontents[currentsection]
    \end{multicols}
\end{frame}

\begin{frame}{Qt框架简介}
    \begin{columns}
        \begin{column}{0.58\textwidth}
    \begin{ytublock}{Qt是什么?}
        \begin{itemize}
            \item 跨平台C++应用程序开发框架
            \item 由Trolltech公司开发(现为Qt Company)
            \item 提供GUI、网络、数据库、多媒体等功能
            \item 支持桌面、移动、嵌入式平台
        \end{itemize}
    \end{ytublock}
    \begin{ytublock}{Qt开发的优秀软件}
        \begin{itemize}
            \item Google Earth
            \item Autodesk Maya
            \item Adobe Photoshop
            \item WPS Office
            \item Virtual Box
            \item COMSOL Multiphysics
        \end{itemize}
    \end{ytublock}
\end{column}
\begin{column}{0.38\textwidth}
    \begin{ytublock}{技术优势}
        \begin{itemize}
            \item 跨平台支持
            \item 丰富的API
            \item 优秀的性能
            \item 活跃的社区
            \item 完善的文档
        \end{itemize}
    \end{ytublock}
    \begin{ytublock}{开发优势}
        \begin{itemize}
            \item 快速开发
            \item 代码复用
            \item 维护简单
            \item 部署方便
            \item 开源免费
        \end{itemize}
    \end{ytublock}
\end{column}
\end{columns}
\end{frame}


\section{Qt核心模块}
\begin{frame}{目录}
    \begin{multicols}{2}
        \tableofcontents[currentsection]
    \end{multicols}
\end{frame}

\begin{frame}{Qt模块架构}
    \begin{ytublock}{核心模块}
        \begin{itemize}
            \item \texttt{QtCore} - 基础核心类(对象模型、事件系统、线程、文件、容器等)
            \item \texttt{QtGui} - 图形用户界面类(窗口、绘图、事件处理等)
            \item \texttt{QtWidgets} - 桌面GUI组件类(按钮、对话框、布局等)
            \item \texttt{QtNetwork} - 网络编程类(TCP/IP、UDP、HTTP等)
            \item \texttt{QtSql} - 数据库支持类(数据库连接、查询、事务等)
            \item \texttt{QtCharts} - 图表组件类(图表绘制)
            \item \texttt{Qt3D} - 3D图形类(3D图形渲染)
            \item \texttt{QtMultimedia} - 多媒体类(音频、视频、图像处理等)
            \item \texttt{QtXml} - XML支持类(XML解析、生成等)
            \item \texttt{QtSvg} - SVG支持类(SVG解析、生成等)
            \item \texttt{QtQml} - QML支持类(QML支持)
            \item \texttt{QtQuick} - 声明式UI类(QML支持)
        \end{itemize}
    \end{ytublock}
\end{frame}

\begin{frame}[fragile]{Qt应用程序结构}
    \begin{columns}
        \begin{column}{0.48\textwidth}
            \inputminted[firstline=1,lastline=20]{cpp}{code/qt_example.cpp}
        \end{column}
        \begin{column}{0.48\textwidth}
            \inputminted[firstline=20,lastline=40]{cpp}{code/qt_example.cpp}
        \end{column}
    \end{columns}
\end{frame}



\section{Qt-Widgets编程}
\begin{frame}{目录}
    \begin{multicols}{2}
        \tableofcontents[currentsection]
    \end{multicols}
\end{frame}


\begin{frame}{Qt 桌面GUI组件类}
    \begin{ytublock}{常用Qt桌面GUI组件}
        {\scriptsize
        \begin{multicols}{2}
        \begin{itemize}
            \item \textbf{QWidget}:所有窗口组件的基类,提供窗口管理功能。
            \item \textbf{QDialog}:对话框组件,支持模态和非模态对话框。
            \item \textbf{QMainWindow}:主窗口组件,集成菜单栏、工具栏、状态栏等。
            \item \textbf{QFrame}:框架组件,提供边框和背景。
            \item \textbf{QScrollArea}:滚动区域组件,支持内容滚动。
            \item \textbf{QSplitter}:分割器组件,可拖动分割线调整子窗口大小。
            \item \textbf{QTabWidget}:标签页组件,支持多标签切换。
            \item \textbf{QStackedWidget}:堆叠组件,多个子窗口堆叠显示。
            \item \textbf{QToolBar}:工具栏组件,放置常用操作按钮。
            \item \textbf{QStatusBar}:状态栏组件,显示状态信息。
            \item \textbf{QMenu}、\textbf{QMenuBar}:菜单组件,提供菜单栏和下拉菜单。
            \item \textbf{QAction}:动作组件,封装可复用的操作。
            \item \textbf{QButtonGroup}:按钮组组件,管理一组按钮。
            \item \textbf{QGroupBox}:分组框组件,分组相关控件。
            \item \textbf{QLineEdit}:单行文本输入框。
            \item \textbf{QTextEdit}:多行文本输入框。
            \item \textbf{QComboBox}:下拉列表框。
            \item \textbf{QListWidget}:列表控件。
            \item \textbf{QTreeWidget}:树形控件。
            \item \textbf{QTableWidget}:表格控件。
            \item \textbf{QCalendarWidget}:日历控件。
            \item \textbf{QDateEdit}、\textbf{QTimeEdit}、\textbf{QDateTimeEdit}:日期/时间输入控件。
            \item \textbf{QSpinBox}、\textbf{QDoubleSpinBox}:整数/浮点数输入框。
            \item \textbf{QSlider}:滑块控件。
            \item \textbf{QScrollBar}:滚动条控件。
            \item \textbf{QProgressBar}:进度条控件。
            \item \textbf{QRubberBand}:橡皮筋选择框。
        \end{itemize}
        \end{multicols}
        }
    \end{ytublock}
\end{frame}

\begin{frame}{QWidget}
    \begin{ytublock}{QWidget 详解}
        \begin{itemize}
            \item \textbf{QWidget} 是 Qt 所有用户界面对象的基类,几乎所有可视化控件(如按钮、文本框、窗口等)都直接或间接继承自 QWidget。
            \item 它提供了窗口的基本管理功能,包括显示、隐藏、移动、调整大小、设置父子关系、事件处理等。
            \item QWidget 支持层次化的父子结构,子控件会自动跟随父控件移动和显示,便于界面组织和管理。
            \item 通过重写 QWidget 的 \texttt{paintEvent()}、\texttt{mousePressEvent()}、\texttt{keyPressEvent()} 等虚函数,可以实现自定义绘制和事件响应。
            \item QWidget 支持多种属性设置,如背景色、字体、光标、焦点策略、透明度等,灵活性极高。
            \item QWidget 既可以作为顶层窗口(无父对象),也可以作为其他控件的子部件嵌入到界面中。
            \item 典型用法:自定义控件时,继承 QWidget 并实现相关事件处理和绘制逻辑。
        \end{itemize}
    \end{ytublock}
\end{frame}

\begin{frame}[fragile]{QWidget 使用示例(一)  }
    \begin{columns}
        \begin{column}{0.48\textwidth}
            \inputminted[firstline=1,lastline=16]{cpp}{code/qt_widget_example.cpp}
        \end{column}
        \begin{column}{0.48\textwidth}
            \inputminted[firstline=17,lastline=31]{cpp}{code/qt_widget_example.cpp}
        \end{column}
    \end{columns}
\end{frame}

\begin{frame}[fragile]{QWidget 使用示例 (二)  }
    \begin{columns}
        \begin{column}{0.48\textwidth}
            \inputminted[firstline=33,lastline=51]{cpp}{code/qt_widget_example.cpp}
        \end{column}
        \begin{column}{0.48\textwidth}
            \inputminted[firstline=52,lastline=68]{cpp}{code/qt_widget_example.cpp}
        \end{column}
    \end{columns}
\end{frame}

\begin{frame}[fragile]{QWidget 使用示例(三)  }
    \begin{columns}
        \begin{column}{0.48\textwidth}
            \inputminted[firstline=69,lastline=83]{cpp}{code/qt_widget_example.cpp}
        \end{column}
        \begin{column}{0.48\textwidth}
            \inputminted[firstline=84,lastline=98]{cpp}{code/qt_widget_example.cpp}
        \end{column}
    \end{columns}
\end{frame}

\begin{frame}{QDialog}
    \begin{ytublock}{QDialog 详解}
        \begin{itemize}
            \item \textbf{QDialog} 是 Qt 中用于实现对话框窗口的基类,广泛用于弹出式交互界面,如设置窗口、消息提示、文件选择等。
            \item QDialog 支持模态(Modal)和非模态(Modeless)两种显示方式。模态对话框会阻塞父窗口的输入,常用于需要用户立即处理的情况;非模态对话框则允许用户继续操作主窗口。
            \item QDialog 提供了丰富的窗口管理功能,包括显示(\texttt{show()})、隐藏(\texttt{hide()})、移动(\texttt{move()})、调整大小(\texttt{resize()})、设置父子关系、事件处理等。
            \item 可以通过重写 QDialog 的虚函数(如 \texttt{accept()}、\texttt{reject()}、\texttt{done()})自定义对话框的关闭行为和结果返回。
            \item QDialog 通常配合标准按钮(如“确定”、“取消”)使用,支持信号槽机制,便于与主窗口或其他控件交互。
            \item QDialog 支持布局管理,可以灵活地添加各种控件,实现复杂的对话界面。
            \item 典型用法:自定义对话框时,继承 QDialog 并实现界面布局、事件处理和业务逻辑。
        \end{itemize}
    \end{ytublock}
\end{frame}

\begin{frame}[fragile]{QDialog 使用示例}
    \begin{columns}
        \begin{column}{0.48\textwidth}
            \inputminted[firstline=1,lastline=20]{cpp}{code/qt_dialog_example.cpp}
        \end{column}
        \begin{column}{0.48\textwidth}
            \inputminted[firstline=21,lastline=40]{cpp}{code/qt_dialog_example.cpp}
        \end{column}
    \end{columns}
\end{frame}

\begin{frame}{QMainWindow}
    \begin{ytublock}{QMainWindow 详解}
        \begin{itemize}
            \item \textbf{QMainWindow} 是 Qt 中用于实现主窗口的基类,通常用于构建应用程序的主界面。
            \item QMainWindow 提供了丰富的窗口管理功能,包括菜单栏、工具栏、状态栏、中央窗口区域等。
            \item 中央窗口区域可以放置各种控件,如 QWidget、QDialog 等。
            \item QMainWindow 支持布局管理,可以灵活地添加各种控件,实现复杂的界面。
            \item 典型用法:构建应用程序主界面时,继承 QMainWindow 并实现界面布局、事件处理和数据交互逻辑。
        \end{itemize}
    \end{ytublock}
\end{frame}

\begin{frame}[fragile]{QMainWindow 使用示例}
    \begin{columns}
        \begin{column}{0.48\textwidth}
            \inputminted[firstline=1,lastline=15]{cpp}{code/qt_mainwindow_example.cpp}
        \end{column}
        \begin{column}{0.48\textwidth}
            \inputminted[firstline=17,lastline=32]{cpp}{code/qt_mainwindow_example.cpp}
        \end{column}
    \end{columns}
\end{frame}

\begin{frame}{QFrame}
    \begin{ytublock}{QFrame 详解}
        \begin{itemize}
            \item \textbf{QFrame} 是 Qt 中用于实现可定制边框和背景的基类控件,常作为其他控件的容器或分隔线。
            \item QFrame 支持多种边框样式(如 Box、Panel、HLine、VLine、StyledPanel 等),可通过 \texttt{setFrameShape()} 和 \texttt{setFrameShadow()} 设置形状和阴影效果。
            \item 可以通过 \texttt{setLineWidth()}、\texttt{setMidLineWidth()} 等方法调整边框宽度。
            \item QFrame 支持设置背景色、背景图片等属性,便于实现美观的界面分隔和装饰。
            \item QFrame 作为基类,常被用来自定义控件,重写其事件处理函数(如 \texttt{paintEvent()})可实现自定义绘制逻辑。
            \item 典型用法包括:作为分隔线(水平线/垂直线)、面板容器、装饰性边框等。
            \item QFrame 也常用于布局中,帮助组织和美化界面结构。
        \end{itemize}
    \end{ytublock}
\end{frame}

\begin{frame}[fragile]{QFrame 使用示例}
    \begin{columns}
        \begin{column}{0.48\textwidth}
            \inputminted[firstline=1,lastline=17]{cpp}{code/qt_frame_example.cpp}
        \end{column}
        \begin{column}{0.48\textwidth}
            \inputminted[firstline=18,lastline=34]{cpp}{code/qt_frame_example.cpp}
        \end{column}
    \end{columns}
\end{frame}

\begin{frame}{QPushButton}
    \begin{ytublock}{QPushButton 详解}
        \begin{itemize}
            \item \textbf{QPushButton} 是 Qt 中常用的按钮控件,适用于各种窗口和对话框,作为用户交互的主要入口。
            \item 支持设置文本、图标,或图标与文本组合,灵活满足不同界面需求。
            \item 可通过 \texttt{setCheckable()} 设置为可切换(开关)状态,适合实现切换类功能。
            \item 支持信号槽机制,常用于触发操作、提交表单、切换状态等场景。
            \item 应用场景:主窗口常用操作入口、编辑器快捷操作、图形界面工具集等。
            \item 典型用法:在 QWidget 或 QMainWindow 中创建 QPushButton,设置属性并连接槽函数,实现交互逻辑。
        \end{itemize}
    \end{ytublock}
\end{frame}

\begin{frame}[fragile]{QPushButton 使用示例}
    \begin{columns}
        \begin{column}{0.48\textwidth}
            \inputminted[firstline=1,lastline=17]{cpp}{code/qt_pushbutton_example.cpp}
        \end{column}
        \begin{column}{0.48\textwidth}
            \inputminted[firstline=18,lastline=34]{cpp}{code/qt_pushbutton_example.cpp}
        \end{column}
    \end{columns}
\end{frame}

\begin{frame}{QLabel}
    \begin{ytublock}{QLabel 详解}
        \begin{itemize}
            \item \textbf{QLabel} 是 Qt 中用于显示文本或图像的控件,常用于界面中的标签、提示信息、图片展示等场景。
            \item 支持设置文本、图片、对齐方式、字体、颜色等属性。
            \item 可以显示富文本(HTML)、超链接,也可用作图片显示控件。
            \item 常与其他控件(如 QLineEdit、QSpinBox 等)配合,用于描述或提示输入内容。
            \item 典型用法:在 QWidget 或 QMainWindow 中创建 QLabel,设置内容和样式,作为界面静态信息展示。
        \end{itemize}
    \end{ytublock}
\end{frame}

\begin{frame}[fragile]{QLabel 使用示例}
    \begin{columns}
        \begin{column}{0.48\textwidth}
            \inputminted[firstline=1,lastline=15]{cpp}{code/qt_label_example.cpp}
        \end{column}
        \begin{column}{0.48\textwidth}
            \inputminted[firstline=16,lastline=30]{cpp}{code/qt_label_example.cpp}
        \end{column}
    \end{columns}
\end{frame}

\begin{frame}{QLineEdit}
    \begin{ytublock}{QLineEdit 详解}
        \begin{itemize}
            \item \textbf{QLineEdit} 是 Qt 中用于单行文本输入的控件,常用于表单、搜索框、命令输入等场景。
            \item 支持设置占位符文本(placeholder)、最大长度、只读、密码模式等属性。
            \item 可通过 \texttt{setText()}、\texttt{text()} 设置和获取内容。
            \item 支持输入校验(Validator)、自动补全(Completer)、输入掩码(InputMask)等高级功能。
            \item 常用信号:\texttt{textChanged}(文本变化)、\texttt{returnPressed}(回车确认)、\texttt{editingFinished}(编辑完成)。
            \item 典型用法:在 QWidget 或 QMainWindow 中创建 QLineEdit,设置属性并连接槽函数,实现数据输入与交互。
        \end{itemize}
    \end{ytublock}
\end{frame}

\begin{frame}[fragile]{QLineEdit 使用示例}
    \begin{columns}
        \begin{column}{0.48\textwidth}
            \inputminted[firstline=1,lastline=19]{cpp}{code/qt_lineedit_example.cpp}
        \end{column}
        \begin{column}{0.48\textwidth}
            \inputminted[firstline=20,lastline=38]{cpp}{code/qt_lineedit_example.cpp}
        \end{column}
    \end{columns}
\end{frame}

\begin{frame}{QTextEdit}
    \begin{ytublock}{QTextEdit 详解}
        \begin{itemize}
            \item \textbf{QTextEdit} 是 Qt 中用于多行富文本编辑和显示的控件,支持纯文本和富文本(HTML)。
            \item 可用于实现文本编辑器、日志窗口、富文本输入、代码编辑等场景。
            \item 支持设置字体、颜色、对齐、段落格式、插入图片和超链接等丰富功能。
            \item 通过 \texttt{setPlainText()}、\texttt{setHtml()} 设置内容,\texttt{toPlainText()}、\texttt{toHtml()} 获取内容。
            \item 支持撤销/重做、查找替换、只读模式、自动换行、拖拽、剪切板操作等。
            \item 常用信号:\texttt{textChanged}(内容变化)、\texttt{cursorPositionChanged}(光标变化)。
            \item 典型用法:在 QWidget 或 QMainWindow 中创建 QTextEdit,设置属性并连接槽函数,实现多行文本输入与显示。
        \end{itemize}
    \end{ytublock}
\end{frame}

\begin{frame}[fragile]{QTextEdit 使用示例}
    \begin{columns}
        \begin{column}{0.48\textwidth}
            \inputminted[firstline=1,lastline=18]{cpp}{code/qt_textedit_example.cpp}
        \end{column}
        \begin{column}{0.48\textwidth}
            \inputminted[firstline=19,lastline=36]{cpp}{code/qt_textedit_example.cpp}
        \end{column}
    \end{columns}
\end{frame}

\begin{frame}{QScrollArea}
    \begin{ytublock}{QScrollArea 详解}
        \begin{itemize}
            \item \textbf{QScrollArea} 是 Qt 中用于实现可滚动区域的控件,适用于内容超出可视范围时的显示需求。
            \item 支持自动显示和隐藏水平、垂直滚动条,能够根据内容和窗口大小智能调整。
            \item 可通过设置属性自定义滚动条的样式、位置、行为等,提升界面美观性和交互体验。
            \item 常用场景:图片浏览器、表单、长文本、复杂控件集合等需要滚动显示的界面。
            \item 使用方法:将需要滚动的控件(如 QWidget、QLabel 等)作为 QScrollArea 的子控件,通过 \texttt{setWidget()} 设置。
        \end{itemize}
    \end{ytublock}
\end{frame}

\begin{frame}[fragile]{QScrollArea 使用示例}
    \begin{columns}
        \begin{column}{0.48\textwidth}
            \inputminted[firstline=1,lastline=21]{cpp}{code/qt_scrollarea_example.cpp}
        \end{column}
        \begin{column}{0.48\textwidth}
            \inputminted[firstline=22,lastline=40]{cpp}{code/qt_scrollarea_example.cpp}
        \end{column}
    \end{columns}
\end{frame}

\begin{frame}{QSplitter}
    \begin{ytublock}{QSplitter 详解}
        \begin{itemize}
            \item \textbf{QSplitter} 是 Qt 中用于实现可拖动分割线的控件,常用于分割窗口或容器。
            \item QSplitter 支持水平和垂直两种分割方式,可通过设置属性自定义分割线的位置、大小、行为等。
            \item 支持设置分割线的样式、位置、大小等属性。
            \item 常用场景:窗口分割、布局调整、界面分隔等。
            \item 使用方法:将需要分割的控件(如 QWidget、QFrame 等)作为 QSplitter 的子控件,通过 \texttt{addWidget()} 添加。
        \end{itemize}
    \end{ytublock}
\end{frame}

\begin{frame}[fragile]{QSplitter 使用示例}
    \begin{columns}
        \begin{column}{0.48\textwidth}
            \inputminted[firstline=1,lastline=15]{cpp}{code/qt_splitter_example.cpp}
        \end{column}
        \begin{column}{0.48\textwidth}
            \inputminted[firstline=16,lastline=31]{cpp}{code/qt_splitter_example.cpp}
        \end{column}
    \end{columns}
\end{frame}

\begin{frame}{QListWidget}
    \begin{ytublock}{QListWidget 详解}
        \begin{itemize}
            \item \textbf{QListWidget} 是 Qt 提供的用于显示和管理条目列表的控件,适合用于简单的列表选择、条目管理等场景。
            \item 支持单选、多选、拖拽排序、条目编辑、图标显示等功能。
            \item 可通过 \texttt{addItem()}、\texttt{addItems()}、\texttt{insertItem()} 等方法动态添加条目。
            \item 支持条目自定义(如设置图标、字体、颜色等),也可通过 \texttt{QListWidgetItem} 进行高级定制。
            \item 常用信号:\texttt{itemClicked}(条目点击)、\texttt{itemDoubleClicked}(双击)、\texttt{currentItemChanged}(当前项变化)等。
            \item 典型用法:在 QWidget 或 QMainWindow 中创建 QListWidget,添加条目并连接槽函数,实现列表数据的交互管理。
        \end{itemize}
    \end{ytublock}
\end{frame}

\begin{frame}[fragile]{QListWidget 使用示例}
    \begin{columns}
        \begin{column}{0.48\textwidth}
            \inputminted[firstline=1,lastline=15]{cpp}{code/qt_listwidget_example.cpp}
        \end{column}
        \begin{column}{0.48\textwidth}
            \inputminted[firstline=16,lastline=30]{cpp}{code/qt_listwidget_example.cpp}
        \end{column}
    \end{columns}
\end{frame}

\begin{frame}{QTreeWidget}
    \begin{ytublock}{QTreeWidget 详解}
        \begin{itemize}
            \item \textbf{QTreeWidget} 是 Qt 提供的树形结构控件,适合用于分层数据的展示与编辑,如文件夹结构、组织架构等。
            \item 支持多级节点、节点展开/收起、节点复选框、图标、富文本等功能。
            \item 可通过 \texttt{addTopLevelItem()}、\texttt{addChild()} 等方法动态添加节点。
            \item 支持拖拽排序、节点编辑、节点选择(单选/多选)、右键菜单等高级交互。
            \item 常用信号:\texttt{itemClicked}(节点点击)、\texttt{itemChanged}(节点内容变化)、\texttt{itemExpanded}(节点展开)等。
            \item 典型用法:在 QWidget 或 QMainWindow 中创建 QTreeWidget,设置列数、表头,添加节点并连接槽函数,实现树形数据的交互管理。
        \end{itemize}
    \end{ytublock}
\end{frame}

\begin{frame}[fragile]{QTreeWidget 使用示例}
    \begin{columns}
        \begin{column}{0.48\textwidth}
            \inputminted[firstline=1,lastline=19]{cpp}{code/qt_treewidget_example.cpp}
        \end{column}
        \begin{column}{0.48\textwidth}
            \inputminted[firstline=20,lastline=34]{cpp}{code/qt_treewidget_example.cpp}
        \end{column}
    \end{columns}
\end{frame}

\begin{frame}{QTabWidget}
    \begin{ytublock}{QTabWidget 详解}
        \begin{itemize}
            \item \textbf{QTabWidget} 是 Qt 中用于实现选项卡式界面的控件,适合将多个页面或功能模块整合在同一窗口中,便于用户切换和管理。
            \item 支持动态添加、删除选项卡,每个选项卡可承载任意 QWidget 子控件,如表单、文本、图表等。
            \item 可自定义选项卡的标签文本、图标、提示信息,支持设置选项卡的位置(上、下、左、右)、可关闭性等属性。
            \item 适用场景:多文档界面、设置面板、属性编辑器等需要分组展示内容的应用。
            \item 常用方法:\texttt{addTab()}、\texttt{removeTab()}、\texttt{setTabPosition()}、\texttt{setTabsClosable()} 等。
        \end{itemize}
    \end{ytublock}
\end{frame}

\begin{frame}[fragile]{QTabWidget 使用示例}
    \begin{columns}
        \begin{column}{0.48\textwidth}
            \inputminted[firstline=1,lastline=20]{cpp}{code/qt_tabwidget_example.cpp}
        \end{column}
        \begin{column}{0.48\textwidth}
            \inputminted[firstline=21,lastline=41]{cpp}{code/qt_tabwidget_example.cpp}
        \end{column}
    \end{columns}
\end{frame}

\begin{frame}{QStackedWidget}
    \begin{ytublock}{QStackedWidget 详解}
        \begin{itemize}
            \item \textbf{QStackedWidget} 是 Qt 中用于实现堆叠式界面的控件,适合将多个子窗口堆叠显示,常用于多步骤向导、标签页等场景。
            \item 支持动态添加、删除子窗口,每个子窗口可承载任意 QWidget 子控件,如表单、文本、图表等。
            \item 可自定义子窗口的样式、位置、大小等属性。
            \item 适用场景:多步骤向导、标签页、属性编辑器等需要分组展示内容的应用。
            \item 常用方法:\texttt{addWidget()}、\texttt{removeWidget()}、\texttt{setCurrentIndex()} 等。
        \end{itemize}
    \end{ytublock}
\end{frame}

\begin{frame}[fragile]{QStackedWidget 使用示例}
    \begin{columns}
        \begin{column}{0.48\textwidth}
            \inputminted[firstline=1,lastline=19]{cpp}{code/qt_stackedwidget_example.cpp}
        \end{column}
        \begin{column}{0.48\textwidth}
            \inputminted[firstline=20,lastline=39]{cpp}{code/qt_stackedwidget_example.cpp}
        \end{column}
    \end{columns}
\end{frame}

\begin{frame}{QToolBar}
    \begin{ytublock}{QToolBar 详解}
        \begin{itemize}
            \item \textbf{QToolBar} 是 Qt 中用于实现工具栏的控件,通常用于主窗口(QMainWindow)中,便于用户快速访问常用操作。
            \item 工具栏可以包含按钮、下拉菜单、分隔符、控件(如下拉框、输入框等),支持图标和文本的灵活组合。
            \item 支持动态添加、移除 QAction,可以通过 \texttt{addAction()}、\texttt{addWidget()}、\texttt{addSeparator()} 等方法管理工具栏内容。
            \item 工具栏可设置停靠位置(上、下、左、右),支持浮动、隐藏、拖拽等交互特性。
            \item 可通过 \texttt{setMovable()} 控制工具栏是否可移动,通过 \texttt{setFloatable()} 控制是否可浮动。
            \item 适用场景:主窗口常用操作入口、编辑器快捷操作、图形界面工具集等。
            \item 典型用法:在 QMainWindow 中创建 QToolBar,添加常用操作按钮,并与槽函数连接,实现快速操作。
        \end{itemize}
    \end{ytublock}
\end{frame}

\begin{frame}[fragile]{QToolBar 使用示例}
    \begin{columns}
        \begin{column}{0.48\textwidth}
            \inputminted[firstline=1,lastline=18]{cpp}{code/qt_toolbar_example.cpp}
        \end{column}
        \begin{column}{0.48\textwidth}
            \inputminted[firstline=19,lastline=33]{cpp}{code/qt_toolbar_example.cpp}
        \end{column}
    \end{columns}
\end{frame}

\begin{frame}{QStatusBar}
    \begin{ytublock}{QStatusBar 详解}
        \begin{itemize}
            \item \textbf{QStatusBar} 是 Qt 中用于实现状态栏的控件,通常用于主窗口(QMainWindow)中,用于显示应用状态、提示信息等。
            \item 状态栏可以包含文本、图标、进度条等控件,支持动态更新内容和自定义样式。
            \item QStatusBar 通常固定在主窗口底部(不可浮动),支持临时消息(\texttt{showMessage()})、永久控件(\texttt{addPermanentWidget()})等功能。
            \item 常用方法:\texttt{showMessage()}、\texttt{clearMessage()}、\texttt{addWidget()}、\texttt{addPermanentWidget()} 等。
            \item 典型用法:在 QMainWindow 中创建 QStatusBar,显示操作提示、进度信息、状态指示等。
        \end{itemize}
    \end{ytublock}
\end{frame}

\begin{frame}[fragile]{QStatusBar 使用示例}
    \begin{columns}
        \begin{column}{0.48\textwidth}
            \inputminted[firstline=1,lastline=19]{cpp}{code/qt_statusbar_example.cpp}
        \end{column}
        \begin{column}{0.48\textwidth}
            \inputminted[firstline=20,lastline=35]{cpp}{code/qt_statusbar_example.cpp}
        \end{column}
    \end{columns}
\end{frame}

\begin{frame}{QMenu}
    \begin{ytublock}{QMenu 详解}
        \begin{itemize}
            \item \textbf{QMenu} 是 Qt 中用于实现下拉菜单的控件,通常用于主窗口(QMainWindow)中,显示菜单项。
            \item QMenu 可包含多个 QAction,支持图标与文本灵活组合,也可嵌套子菜单(QMenu)。
            \item 支持通过 \texttt{addAction()}、\texttt{addMenu()}、\texttt{addSeparator()} 等方法动态管理菜单内容。
            \item 菜单项可与槽函数连接,实现响应式操作。
            \item QMenu 可独立弹出(如右键菜单),也可作为菜单栏(QMenuBar)子菜单使用。
            \item 常见场景:主菜单、右键上下文菜单、功能分组菜单等。
            \item 典型用法:在 QMainWindow 中创建 QMenuBar,添加 QMenu,再添加 QAction,并与槽函数连接。
        \end{itemize}
    \end{ytublock}
\end{frame}

\begin{frame}{QMenuBar}
    \begin{ytublock}{QMenuBar 详解}
        \begin{itemize}
            \item \textbf{QMenuBar} 是 Qt 中用于实现菜单栏的控件,通常用于主窗口(QMainWindow)中,显示菜单项。
            \item QMenuBar 可包含多个 QMenu,支持图标与文本灵活组合,也可嵌套子菜单(QMenu)。
            \item 支持通过 \texttt{addMenu()}、\texttt{addSeparator()} 等方法动态管理菜单栏内容。
            \item 菜单栏可设置停靠位置(上、下、左、右),支持浮动、隐藏、拖拽等交互特性。
            \item 可通过 \texttt{setMovable()} 控制菜单栏是否可移动,通过 \texttt{setFloatable()} 控制是否可浮动。
            \item 适用场景:主窗口常用操作入口、编辑器快捷操作、图形界面工具集等。
            \item 典型用法:在 QMainWindow 中创建 QMenuBar,添加 QMenu,再添加 QAction,并与槽函数连接。
        \end{itemize}
    \end{ytublock}
\end{frame}

\begin{frame}[fragile]{QMenu / QMenuBar 使用示例}
    \begin{columns}
        \begin{column}{0.48\textwidth}
            \inputminted[firstline=1,lastline=19]{cpp}{code/qt_menu_example.cpp}
        \end{column}
        \begin{column}{0.48\textwidth}
            \inputminted[firstline=20,lastline=38]{cpp}{code/qt_menu_example.cpp}
        \end{column}
    \end{columns}
\end{frame}

\begin{frame}{QToolButton}
    \begin{ytublock}{QToolButton 详解}
        \begin{itemize}
            \item \textbf{QToolButton} 是 Qt 提供的专用工具按钮控件,常用于主窗口(QMainWindow)或工具栏(QToolBar)中,作为功能快捷入口。
            \item 支持同时显示图标和文本,可通过 \texttt{setIcon()}、\texttt{setText()} 灵活设置。
            \item 可通过 \texttt{setToolButtonStyle()} 控制图标与文本的排列方式(如仅图标、仅文本、图标在上/左等)。
            \item 支持 \texttt{setAutoRaise()},实现扁平风格和悬浮高亮效果,提升界面美观性。
            \item 通过 \texttt{setPopupMode()} 可设置弹出菜单模式(如即时弹出、延迟弹出、仅菜单)。
            \item 可用 \texttt{setDefaultAction()} 绑定 QAction,实现与菜单栏、工具栏一致的行为。
            \item 支持 \texttt{setCheckable()},可设置为可切换(开关)状态,适合工具属性切换。
            \item 常见应用:工具栏按钮、带下拉菜单的操作按钮、属性切换按钮等。
        \end{itemize}
    \end{ytublock}
\end{frame}

\begin{frame}[fragile]{QToolButton 使用示例}
    \begin{columns}
        \begin{column}{0.48\textwidth}
            \inputminted[firstline=1,lastline=19]{cpp}{code/qt_toolbutton_example.cpp}
        \end{column}
        \begin{column}{0.48\textwidth}
            \inputminted[firstline=20,lastline=39]{cpp}{code/qt_toolbutton_example.cpp}
        \end{column}
    \end{columns}
\end{frame}

\begin{frame}{QRadioButton}
    \begin{ytublock}{QRadioButton 详解}
        \begin{itemize}
            \item \textbf{QRadioButton} 是 Qt 中用于实现单选按钮的控件,通常用于主窗口(QMainWindow)中,用于显示单选按钮。
            \item 支持设置文本、图标,或图标与文本组合,灵活满足不同界面需求。
            \item 可通过 \texttt{setChecked()} 设置为选中状态,通过 \texttt{isChecked()} 获取选中状态。
            \item 支持信号槽机制,常用于触发操作、提交表单、切换状态等场景。
            \item 应用场景:主窗口常用操作入口、编辑器快捷操作、图形界面工具集等。
            \item 典型用法:在 QWidget 或 QMainWindow 中创建 QRadioButton,设置属性并连接槽函数,实现交互逻辑。
        \end{itemize}
    \end{ytublock}
\end{frame}

\begin{frame}[fragile]{QRadioButton 使用示例}
    \begin{columns}
        \begin{column}{0.48\textwidth}
            \inputminted[firstline=1,lastline=19]{cpp}{code/qt_radiobutton_example.cpp}
        \end{column}
        \begin{column}{0.48\textwidth}
            \inputminted[firstline=20,lastline=38]{cpp}{code/qt_radiobutton_example.cpp}
        \end{column}
    \end{columns}
\end{frame}

\begin{frame}{QCheckBox}
    \begin{ytublock}{QCheckBox 详解}
        \begin{itemize}
            \item \textbf{QCheckBox} 是 Qt 中用于实现复选框的控件,通常用于主窗口(QMainWindow)中,用于显示复选框。
            \item 支持设置文本、图标,或图标与文本组合,灵活满足不同界面需求。
            \item 可通过 \texttt{setChecked()} 设置为选中状态,通过 \texttt{isChecked()} 获取选中状态。
            \item 支持信号槽机制,常用于触发操作、提交表单、切换状态等场景。
            \item 应用场景:主窗口常用操作入口、编辑器快捷操作、图形界面工具集等。
            \item 典型用法:在 QWidget 或 QMainWindow 中创建 QCheckBox,设置属性并连接槽函数,实现交互逻辑。
        \end{itemize}
    \end{ytublock}
\end{frame}

\begin{frame}[fragile]{QCheckBox 使用示例}
    \begin{columns}
        \begin{column}{0.48\textwidth}
            \inputminted[firstline=1,lastline=19]{cpp}{code/qt_checkbox_example.cpp}
        \end{column}
        \begin{column}{0.48\textwidth}
            \inputminted[firstline=20,lastline=38]{cpp}{code/qt_checkbox_example.cpp}
        \end{column}
    \end{columns}
\end{frame}

\begin{frame}{QComboBox}
    \begin{ytublock}{QComboBox 详解}
        \begin{itemize}
            \item \textbf{QComboBox} 是 Qt 中用于实现下拉列表的控件,通常用于主窗口(QMainWindow)中,用于显示下拉列表。
            \item 支持设置文本、图标,或图标与文本组合,灵活满足不同界面需求。
            \item 可通过 \texttt{addItem()} 添加选项,通过 \texttt{currentText()} 获取当前选项。
            \item 支持信号槽机制,常用于触发操作、提交表单、切换状态等场景。
            \item 应用场景:主窗口常用操作入口、编辑器快捷操作、图形界面工具集等。
            \item 典型用法:在 QWidget 或 QMainWindow 中创建 QComboBox,设置属性并连接槽函数,实现交互逻辑。
        \end{itemize}
    \end{ytublock}
\end{frame}

\begin{frame}[fragile]{QComboBox 使用示例}
    \begin{columns}
        \begin{column}{0.48\textwidth}
            \inputminted[firstline=1,lastline=15]{cpp}{code/qt_combobox_example.cpp}
        \end{column}
        \begin{column}{0.48\textwidth}
            \inputminted[firstline=16,lastline=30]{cpp}{code/qt_combobox_example.cpp}
        \end{column}
    \end{columns}
\end{frame}

\begin{frame}{QSpinBox}
    \begin{ytublock}{QSpinBox 详解}
        \begin{itemize}
            \item \textbf{QSpinBox} 是 Qt 中用于实现计数器的控件,通常用于主窗口(QMainWindow)中,用于显示计数器。
            \item 支持设置文本、图标,或图标与文本组合,灵活满足不同界面需求。
            \item 可通过 \texttt{setRange()} 设置范围,通过 \texttt{setSingleStep()} 设置步长。
            \item 支持信号槽机制,常用于触发操作、提交表单、切换状态等场景。
            \item 应用场景:主窗口常用操作入口、编辑器快捷操作、图形界面工具集等。
            \item 典型用法:在 QWidget 或 QMainWindow 中创建 QSpinBox,设置属性并连接槽函数,实现交互逻辑。
        \end{itemize}
    \end{ytublock}
\end{frame}

\begin{frame}[fragile]{QSpinBox 使用示例}
    \begin{columns}
        \begin{column}{0.48\textwidth}
            \inputminted[firstline=1,lastline=15]{cpp}{code/qt_spinbox_example.cpp}
        \end{column}
        \begin{column}{0.48\textwidth}
            \inputminted[firstline=16,lastline=31]{cpp}{code/qt_spinbox_example.cpp}
        \end{column}
    \end{columns}
\end{frame}

\begin{frame}{QDoubleSpinBox}
    \begin{ytublock}{QDoubleSpinBox 详解}
        \begin{itemize}
            \item \textbf{QDoubleSpinBox} 是 Qt 中用于实现双精度浮点数计数器的控件,通常用于主窗口(QMainWindow)中,用于显示双精度浮点数计数器。
            \item 支持设置文本、图标,或图标与文本组合,灵活满足不同界面需求。
            \item 可通过 \texttt{setRange()} 设置范围,通过 \texttt{setSingleStep()} 设置步长,通过 \texttt{setDecimals()} 设置小数位数。
            \item 支持信号槽机制,常用于触发操作、提交表单、切换状态等场景。
            \item 应用场景:主窗口常用操作入口、编辑器快捷操作、图形界面工具集等。
            \item 典型用法:在 QWidget 或 QMainWindow 中创建 QDoubleSpinBox,设置属性并连接槽函数,实现交互逻辑。
        \end{itemize}
    \end{ytublock}
\end{frame}

\begin{frame}[fragile]{QDoubleSpinBox 使用示例}
    \begin{columns}
        \begin{column}{0.48\textwidth}
            \inputminted[firstline=1,lastline=17]{cpp}{code/qt_doublespinbox_example.cpp}
        \end{column}
        \begin{column}{0.48\textwidth}
            \inputminted[firstline=18,lastline=34]{cpp}{code/qt_doublespinbox_example.cpp}
        \end{column}
    \end{columns}
\end{frame}

\begin{frame}{QDateTimeEdit}
    \begin{ytublock}{QDateTimeEdit 详解}
        \begin{itemize}
            \item \textbf{QDateTimeEdit} 是 Qt 中用于实现日期和时间编辑器的控件,通常用于主窗口(QMainWindow)中,用于显示日期和时间编辑器。
            \item \textbf{QDateEdit} 是 Qt 中用于实现日期编辑器的控件,通常用于主窗口(QMainWindow)中,用于显示日期编辑器。
            \item \textbf{QTimeEdit} 是 Qt 中用于实现时间编辑器的控件,通常用于主窗口(QMainWindow)中,用于显示时间编辑器。
            \item 支持设置日期和时间,或日期和时间组合,灵活满足不同界面需求。
            \item 支持信号槽机制,常用于触发操作、提交表单、切换状态等场景。
            \item 应用场景:主窗口常用操作入口、编辑器快捷操作、图形界面工具集等。
            \item 典型用法:在 QWidget 或 QMainWindow 中创建 QDateTimeEdit,设置属性并连接槽函数,实现交互逻辑。
        \end{itemize}
    \end{ytublock}
\end{frame}

\begin{frame}[fragile]{QDateTimeEdit 使用示例}
    \begin{columns}
        \begin{column}{0.48\textwidth}
            \inputminted[firstline=15,lastline=28]{cpp}{code/qt_datetimeedit_example.cpp}
        \end{column}
        \begin{column}{0.48\textwidth}
            \inputminted[firstline=29,lastline=45]{cpp}{code/qt_datetimeedit_example.cpp}
        \end{column}
    \end{columns}
\end{frame}

\begin{frame}{QCalendarWidget}
    \begin{ytublock}{QCalendarWidget 详解}
        \begin{itemize}
            \item \textbf{QCalendarWidget} 是 Qt 中用于实现日历的控件,通常用于主窗口(QMainWindow)中,用于显示日历。
            \item 支持设置日期,灵活满足不同界面需求。
            \item 支持信号槽机制,常用于触发操作、提交表单、切换状态等场景。
            \item 应用场景:主窗口常用操作入口、编辑器快捷操作、图形界面工具集等。
            \item 典型用法:在 QWidget 或 QMainWindow 中创建 QCalendarWidget,设置属性并连接槽函数,实现交互逻辑。
        \end{itemize}
    \end{ytublock}
\end{frame}

\begin{frame}[fragile]{QCalendarWidget 使用示例}
    \begin{columns}
        \begin{column}{0.48\textwidth}
            \inputminted[firstline=1,lastline=18]{cpp}{code/qt_calendarwidget_example.cpp}
        \end{column}
        \begin{column}{0.48\textwidth}
            \inputminted[firstline=19,lastline=36]{cpp}{code/qt_calendarwidget_example.cpp}
        \end{column}
    \end{columns}
\end{frame}

\begin{frame}{QSlider}
    \begin{ytublock}{QSlider 详解}
        \begin{itemize}
            \item \textbf{QSlider} 是 Qt 中用于实现滑动条的控件,常用于让用户在一定范围内选择数值。
            \item 支持水平(Horizontal)和垂直(Vertical)两种方向。
            \item 可设置最小值、最大值、步长、初始值等属性。
            \item 支持信号槽机制,常用信号有 \texttt{valueChanged(int)},可用于实时响应用户操作。
            \item 应用场景:音量调节、进度控制、参数调整等。
            \item 典型用法:在 QWidget 或 QMainWindow 中创建 QSlider,设置属性并连接槽函数,实现交互逻辑。
        \end{itemize}
    \end{ytublock}
\end{frame}

\begin{frame}[fragile]{QSlider 使用示例}
    \begin{columns}
        \begin{column}{0.48\textwidth}
            \inputminted[firstline=1,lastline=19]{cpp}{code/qt_slider_example.cpp}
        \end{column}
        \begin{column}{0.48\textwidth}
            \inputminted[firstline=20,lastline=36]{cpp}{code/qt_slider_example.cpp}
        \end{column}
    \end{columns}
\end{frame}

\begin{frame}{QScrollBar}
    \begin{ytublock}{QScrollBar 详解}
        \begin{itemize}
            \item \textbf{QScrollBar} 是 Qt 中用于实现滚动条的控件,常用于让用户在一定范围内选择数值。
            \item 支持水平(Horizontal)和垂直(Vertical)两种方向。
            \item 可设置最小值、最大值、步长、初始值等属性。
            \item 支持信号槽机制,常用信号有 \texttt{valueChanged(int)},可用于实时响应用户操作。
            \item 应用场景:音量调节、进度控制、参数调整等。
            \item 典型用法:在 QWidget 或 QMainWindow 中创建 QScrollBar,设置属性并连接槽函数,实现交互逻辑。
        \end{itemize}
    \end{ytublock}
\end{frame}

\begin{frame}[fragile]{QScrollBar 使用示例}
    \begin{columns}
        \begin{column}{0.48\textwidth}
            \inputminted[firstline=1,lastline=17]{cpp}{code/qt_scrollbar_example.cpp}
        \end{column}
        \begin{column}{0.48\textwidth}
            \inputminted[firstline=18,lastline=31]{cpp}{code/qt_scrollbar_example.cpp}
        \end{column}
    \end{columns}
\end{frame}

\begin{frame}{QProgressBar}
    \begin{ytublock}{QProgressBar 详解}
        \begin{itemize}
            \item \textbf{QProgressBar} 是 Qt 中用于实现进度条的控件,常用于显示任务进度。
            \item 支持水平(Horizontal)和垂直(Vertical)两种方向。
            \item 可设置最小值、最大值、步长、初始值等属性。
            \item 支持信号槽机制,常用信号有 \texttt{valueChanged(int)},可用于实时响应用户操作。
            \item 应用场景:任务进度、文件下载、数据传输等。
            \item 典型用法:在 QWidget 或 QMainWindow 中创建 QProgressBar,设置属性并连接槽函数,实现交互逻辑。
        \end{itemize}
    \end{ytublock}
\end{frame}

\begin{frame}[fragile]{QProgressBar 使用示例}
    \begin{columns}
        \begin{column}{0.48\textwidth}
            \inputminted[firstline=14,lastline=28]{cpp}{code/qt_progressbar_example.cpp}
        \end{column}
        \begin{column}{0.48\textwidth}
            \inputminted[firstline=29,lastline=43]{cpp}{code/qt_progressbar_example.cpp}
        \end{column}
    \end{columns}
\end{frame}

\begin{frame}{Qt Designer}
    \begin{ytublock}{Qt Designer 详解}
        \begin{itemize}
            \item \textbf{Qt Designer} 是 Qt 官方提供的可视化界面设计工具,专用于快速开发和设计 Qt 应用程序的 GUI 界面。
            \item 支持拖拽式控件布局,可直观添加、排列和配置各种 Qt 控件(如按钮、标签、输入框等)。
            \item 内置多种布局管理器(如 QHBoxLayout、QVBoxLayout、QGridLayout),便于实现响应式和自适应界面。
            \item 可通过属性编辑器设置控件属性,支持信号与槽的可视化连接,简化界面交互逻辑的搭建。
            \item 支持样式表(QSS)编辑,可实时预览控件美化效果,提升界面美观性。
            \item 生成的 .ui 文件可直接在 Qt Creator 或 C++/Python 项目中加载,支持与代码逻辑分离,便于团队协作和后期维护。
            \item 适用于原型设计、界面快速迭代、跨平台 GUI 开发等多种场景,是 Qt 开发的重要辅助工具。
        \end{itemize}
    \end{ytublock}
\end{frame}

\begin{frame}{QSS}
    \begin{ytublock}{QSS 详解}
        \begin{itemize}
            \item \textbf{QSS} 是 Qt 中用于实现样式表的控件,常用于美化界面。
            \item 支持设置样式表,灵活满足不同界面需求。
            \item 支持信号槽机制,常用于触发操作、提交表单、切换状态等场景。
            \item 应用场景:界面美化、主题切换、样式定制等。
        \end{itemize}
    \end{ytublock}
\end{frame}

\begin{frame}[fragile]{QSS 使用示例 (一)}
    \begin{columns}
        \begin{column}{0.48\textwidth}
            \inputminted[firstline=1,lastline=19]{cpp}{code/qt_qss_example.cpp}
        \end{column}
        \begin{column}{0.48\textwidth}
            \inputminted[firstline=20,lastline=39]{cpp}{code/qt_qss_example.cpp}
        \end{column}
    \end{columns}
\end{frame}

\begin{frame}[fragile]{QSS 使用示例(二)}
    \begin{columns}
        \begin{column}{0.48\textwidth}
            \inputminted[firstline=40,lastline=58]{cpp}{code/qt_qss_example.cpp}
        \end{column}
        \begin{column}{0.48\textwidth}
            \inputminted[firstline=59,lastline=78]{cpp}{code/qt_qss_example.cpp}
        \end{column}
    \end{columns}
\end{frame}

\section{元对象系统}
\begin{frame}{目录}
    \begin{multicols}{2}
        \tableofcontents[currentsection]
    \end{multicols}
\end{frame}

\begin{frame}{元对象系统概述}
    \begin{ytublock}{什么是元对象系统?}
        元对象系统(Meta-Object System)是 Qt 框架中一个核心的、强大的基础设施。它为 Qt 提供了标准 C++ 本身不具备的一些高级功能,使得 Qt 应用程序更加灵活、动态和易于开发。

        简单来说,元对象系统是 Qt 在 C++ 基础上"扩展"出来的一套机制,它让对象能够"知道"关于自身的一些信息(如类名、属性、方法等),并支持对象间的动态通信。
    \end{ytublock}

    \begin{columns}
        \begin{column}{0.48\textwidth}
            \begin{ytublock}{元对象系统的作用}
                \begin{itemize}
                    \item 提供运行时类型信息
                    \item 支持信号槽机制
                    \item 实现属性系统
                    \item 支持动态属性
                    \item 提供反射能力
                \end{itemize}
            \end{ytublock}
        \end{column}
        \begin{column}{0.48\textwidth}
            \begin{ytublock}{核心组件}
                \begin{itemize}
                    \item \texttt{Q\_OBJECT}宏
                    \item \texttt{moc}(Meta-Object Compiler)
                    \item \texttt{QMetaObject}类
                    \item \texttt{QObject}基类
                \end{itemize}
            \end{ytublock}
        \end{column}
    \end{columns}
\end{frame}

\begin{frame}[fragile]{Q\_OBJECT宏}
  \begin{columns}
    \begin{column}{0.48\textwidth}
      \inputminted[firstline=1,lastline=20]{cpp}{code/qobject_example.h}
    \end{column}
    \begin{column}{0.48\textwidth}
      \inputminted[firstline=20,lastline=40]{cpp}{code/qobject_example.h}
    \end{column}
  \end{columns}
\end{frame}

\begin{frame}{Meta-Object Compiler(moc)元对象编译器}
    \begin{ytublock}{moc的作用}
        \begin{itemize}
            \item \textbf{moc(Meta-Object Compiler)} 是Qt框架中专门用于处理元对象系统的工具。
            \item 它会扫描C++头文件,自动生成实现信号、槽、属性等元对象相关功能的代码。
            \item 只有包含 \texttt{Q\_OBJECT} 宏的类才需要moc处理。
            \item moc生成的代码会被编译并链接到最终的程序中,实现运行时类型信息、信号槽机制等高级特性。
            \item 开发者无需手动编写这些元对象相关的底层代码,极大提升了开发效率和代码安全性。
        \end{itemize}
    \end{ytublock}
    \begin{ytublock}{moc的运行过程}
        \begin{itemize}
            \item 扫描头文件,找到包含 \texttt{Q\_OBJECT} 宏的类。
            \item 生成包含信号、槽、属性等元对象相关功能的代码。
            \item 将生成的代码编译并链接到最终的程序中。
        \end{itemize}
    \end{ytublock}
\end{frame}

\begin{frame}[fragile]{QMetaObject类}
    \begin{ytublock}{QMetaObject类的作用}
        \begin{itemize}
            \item QMetaObject类是Qt元对象系统的核心,负责描述和管理QObject派生类的元信息。
            \item 通过QMetaObject,可以在运行时获取类名、属性、信号、槽等信息,实现反射和动态调用。
        \end{itemize}
    \end{ytublock}
    \begin{ytublock}{QMetaObject类的常用方法}
        \begin{itemize}
            \item \texttt{className()}:返回类的名称。
            \item \texttt{indexOfMethod(const char *method)}:返回指定方法的索引。
            \item \texttt{indexOfSignal(const char *signal)}:返回指定信号的索引。
            \item \texttt{invokeMethod(QObject *obj, const char *member, ...)}:在运行时调用对象的方法(槽函数)。
            \item \texttt{propertyCount()}、\texttt{methodCount()}:获取属性和方法的数量。
            \item \texttt{property(int index)}、\texttt{method(int index)}:通过索引获取属性或方法的元信息。
        \end{itemize}
    \end{ytublock}
\end{frame}

\begin{frame}[fragile]{QObject基类}
    \begin{columns}
        \begin{column}{0.48\textwidth}
            \begin{ytublock}{QObject基类的作用}
                \begin{itemize}
                    \item \textbf{QObject} 是Qt框架中所有对象的基类,绝大多数Qt类都直接或间接继承自它。
                    \item 提供了信号槽机制、事件处理、对象树(父子关系)、动态属性、对象名称等核心功能。
                    \item 支持对象间通信、自动内存管理(父对象析构时自动析构子对象),便于资源管理。
                    \item 通过 \texttt{Q\_OBJECT} 宏启用元对象系统,支持运行时类型信息和反射。
                \end{itemize}
            \end{ytublock}
        \end{column}
        \begin{column}{0.48\textwidth}
            \begin{ytublock}{QObject基类的常用方法}
                \begin{itemize}
                    \item \texttt{connect()}:连接信号和槽。
                    \item \texttt{disconnect()}:断开信号和槽的连接。
                    \item \texttt{emit()}:发送信号。
                    \item \texttt{parent()}:获取父对象。
                    \item \texttt{children()}:获取子对象。
                    \item \texttt{setProperty()}:设置属性。
                    \item \texttt{property()}:获取属性。
                    \item \texttt{metaObject()}:获取元对象。
                    \item \texttt{inherits()}:检查是否继承自指定类。
                    \item \texttt{findChild()}:查找子对象。
                    \item \texttt{event()}:事件处理。
                \end{itemize}
            \end{ytublock}
        \end{column}
    \end{columns}
\end{frame}

\section{信号槽机制}
\begin{frame}{目录}
    \begin{multicols}{2}
        \tableofcontents[currentsection]
    \end{multicols}
\end{frame}

\begin{frame}{事件}
\begin{ytublock}{事件的概念}
    \begin{itemize}
        \item 事件(Event)是GUI编程中用于描述用户操作(如鼠标点击、键盘输入)、系统消息(如窗口重绘、定时器超时)等的对象。
        \item 事件机制实现了对象与外部交互的基本方式,是GUI响应用户操作的基础。
    \end{itemize}
\end{ytublock}
\begin{ytublock}{事件的缺陷}
    \begin{itemize}
        \item 事件处理函数通常需要重写基类方法,导致代码分散、可维护性差。
        \item 事件只能在对象内部处理,难以实现对象间的灵活通信。
        \item 事件机制不支持多对象同时响应同一事件,扩展性有限。
        \item 事件的传递和处理流程较为复杂,难以实现解耦和灵活的响应机制。
        \item 难以动态连接和断开事件响应,缺乏运行时的灵活性。
    \end{itemize}
\end{ytublock}
\end{frame}

\begin{frame}[fragile]{C++事件示例}
    \begin{columns}
        \begin{column}{0.48\textwidth}
            \inputminted[firstline=1,lastline=18]{cpp}{code/cpp_event_example.cpp}
        \end{column}
        \begin{column}{0.48\textwidth}
            \inputminted[firstline=20,lastline=40]{cpp}{code/cpp_event_example.cpp}
        \end{column}
    \end{columns}
\end{frame}

\begin{frame}[fragile]{Qt事件示例}
    \begin{columns}
        \begin{column}{0.48\textwidth}
            \inputminted[firstline=1,lastline=20]{cpp}{code/qt_event_example.cpp}
        \end{column}
        \begin{column}{0.48\textwidth}
            \inputminted[firstline=21,lastline=40]{cpp}{code/qt_event_example.cpp}
        \end{column}
    \end{columns}
\end{frame}

\begin{frame}{信号槽概念}
    \begin{itemize}
        \item \textbf{信号(Signal)}:对象在特定状态或事件发生时发出的通知。
        \item \textbf{槽(Slot)}:用于接收和处理信号的普通成员函数或全局函数,是对信号的响应动作。
        \item \textbf{连接(Connection)}:将信号与一个或多个槽函数绑定起来,信号发出时自动调用对应的槽。
        \item \textbf{解耦}:信号的发送者和槽的接收者无需相互了解,提升代码的灵活性和可维护性。
        \item \textbf{对象间通信}:信号槽机制是Qt实现对象间通信的核心方式,支持跨线程、跨对象的消息传递。
        \item \textbf{动态连接与断开}:可以在运行时动态地连接和断开信号与槽,满足复杂场景下的需求。
        \item \textbf{类型安全}:编译器会检查信号和槽的参数类型,避免类型不匹配导致的错误。
        \item \textbf{松耦合设计}:发送者和接收者完全解耦,便于模块化开发和单元测试。
        \item \textbf{支持多对多连接}:一个信号可以连接多个槽,一个槽也可以接收多个信号。
        \item \textbf{自动内存管理}:当对象销毁时,相关的信号槽连接会自动断开。
        \item \textbf{线程安全}:Qt支持跨线程的信号槽通信,自动进行事件队列投递,简化多线程编程。
        \item \textbf{可扩展性强}:信号槽机制支持自定义信号和槽,适用于各种复杂的应用场景。
    \end{itemize}
\end{frame}

\begin{frame}{信号槽机制原理}
    \begin{ytublock}{信号槽对比事件}
        \begin{table}[H]
            \centering
            \begin{tabular}{|p{2.5cm}|p{5.2cm}|p{5.2cm}|}
                \hline
                & \textbf{事件机制} & \textbf{信号槽机制} \\
                \hline
                响应方式 & 通过重写事件处理函数(如 \texttt{event()}、\texttt{mousePressEvent()}),对象内部响应 & 通过信号(Signal)和槽(Slot)机制,对象间通信 \\
                \hline
                适用场景 & 适合处理底层、面向对象自身的消息 & 适合对象间的灵活响应和协作 \\
                \hline
                扩展性与灵活性 & 扩展性和灵活性有限,连接固定 & 支持多对多连接,动态连接和断开,灵活性高 \\
                \hline
                代码维护性 & 代码分散、耦合度高,维护较难 & 发送者和接收者解耦,易于维护和扩展 \\
                \hline
            \end{tabular}
        \end{table}
    \end{ytublock}
\end{frame}

\begin{frame}[fragile]{信号槽基本用法}
    \begin{columns}
        \begin{column}{0.48\textwidth}
            \inputminted[firstline=1,lastline=17]{cpp}{code/qt_signal_slot_example.cpp}
        \end{column}
        \begin{column}{0.48\textwidth}
            \inputminted[firstline=18,lastline=38]{cpp}{code/qt_signal_slot_example.cpp}
        \end{column}
    \end{columns}
\end{frame}

\begin{frame}[fragile]{信号槽连接方式(一)}
    \begin{columns}
        \begin{column}{0.48\textwidth}
            \inputminted[firstline=1,lastline=20]{cpp}{code/qt_connect_example.cpp}
        \end{column}
        \begin{column}{0.48\textwidth}
            \inputminted[firstline=21,lastline=40]{cpp}{code/qt_connect_example.cpp}
        \end{column}
    \end{columns}
\end{frame}

\begin{frame}[fragile]{信号槽连接方式(二)}
    \begin{columns}
        \begin{column}{0.48\textwidth}
            \inputminted[firstline=26,lastline=38]{cpp}{code/qt_connect_advanced_example.cpp}
        \end{column}
        \begin{column}{0.48\textwidth}
            \inputminted[firstline=39,lastline=56]{cpp}{code/qt_connect_advanced_example.cpp}
        \end{column}
    \end{columns}
\end{frame}

\section{事件系统}
\begin{frame}{目录}
    \begin{multicols}{2}
        \tableofcontents[currentsection]
    \end{multicols}
\end{frame}

\begin{frame}{Qt事件系统}
    \begin{ytublock}{事件类型}
        \begin{itemize}
            \item 鼠标事件(Mouse Events)
            \item 键盘事件(Keyboard Events)
            \item 窗口事件(Window Events)
            \item 拖放事件(Drag \& Drop Events)
            \item 定时器事件(Timer Events)
            \item 自定义事件(Custom Events)
        \end{itemize}
    \end{ytublock}

    \begin{ytublock}{事件处理方式}
        \begin{itemize}
            \item 事件循环(Event Loop)
            \item 事件过滤器(Event Filters)
            \item 事件处理器(Event Handlers)
        \end{itemize}
    \end{ytublock}
\end{frame}

\begin{frame}[fragile]{鼠标事件示例}
    \begin{columns}
        \begin{column}{0.48\textwidth}
            \inputminted[firstline=1,lastline=19]{cpp}{code/qt_mouse_event_example.cpp}
        \end{column}
        \begin{column}{0.48\textwidth}
            \inputminted[firstline=21,lastline=41]{cpp}{code/qt_mouse_event_example.cpp}
        \end{column}
    \end{columns}
\end{frame}

\begin{frame}[fragile]{键盘事件示例}
    \begin{columns}
        \begin{column}{0.48\textwidth}
            \inputminted[firstline=1,lastline=21]{cpp}{code/qt_keyboard_event_example.cpp}
        \end{column}
        \begin{column}{0.48\textwidth}
            \inputminted[firstline=22,lastline=38]{cpp}{code/qt_keyboard_event_example.cpp}
        \end{column}
    \end{columns}
\end{frame}

\begin{frame}[fragile]{窗口事件示例}
    \begin{columns}
        \begin{column}{0.48\textwidth}
            \inputminted[firstline=1,lastline=20]{cpp}{code/qt_window_event_example.cpp}
        \end{column}
        \begin{column}{0.48\textwidth}
            \inputminted[firstline=22,lastline=39]{cpp}{code/qt_window_event_example.cpp}
        \end{column}
    \end{columns}
\end{frame}

\begin{frame}[fragile]{拖放事件示例}
    \begin{columns}
        \begin{column}{0.48\textwidth}
            \inputminted[firstline=1,lastline=22]{cpp}{code/qt_drag_drop_event_example.cpp}
        \end{column}
        \begin{column}{0.48\textwidth}
            \inputminted[firstline=23,lastline=44]{cpp}{code/qt_drag_drop_event_example.cpp}
        \end{column}
    \end{columns}
\end{frame}

\begin{frame}[fragile]{定时器事件示例}
    \begin{columns}
        \begin{column}{0.48\textwidth}
            \inputminted[firstline=1,lastline=21]{cpp}{code/qt_timer_event_example.cpp}
        \end{column}
        \begin{column}{0.48\textwidth}
            \inputminted[firstline=22,lastline=39]{cpp}{code/qt_timer_event_example.cpp}
        \end{column}
    \end{columns}
\end{frame}

\begin{frame}[fragile]{自定义事件示例}
    \begin{columns}
        \begin{column}{0.48\textwidth}
            \inputminted[firstline=1,lastline=22]{cpp}{code/qt_custom_event_example.cpp}
        \end{column}
        \begin{column}{0.48\textwidth}
            \inputminted[firstline=23,lastline=41]{cpp}{code/qt_custom_event_example.cpp}
        \end{column}
    \end{columns}
\end{frame}

\begin{frame}[fragile]{事件循环(Event Loop)}
    \begin{ytublock}{事件循环简介}
        \begin{itemize}
            \item 事件循环是Qt应用程序的核心机制,负责持续接收、分发和处理事件与消息,保证界面响应和程序流畅运行。
            \item 事件循环通过 \texttt{QApplication::exec()} 启动,主线程进入循环,等待并处理各种事件(如鼠标、键盘、定时器、自定义事件等)。
        \end{itemize}
    \end{ytublock}
    \begin{ytublock}{事件循环实现方式}
        \begin{itemize}
            \item 通过重写 \texttt{event(QEvent *event)} 方法,可以自定义事件的处理逻辑,实现对特定事件的拦截与响应。
            \item 事件循环自动调用事件处理函数,无需手动轮询,大大简化了事件驱动编程。
        \end{itemize}
    \end{ytublock}
\end{frame}

\begin{frame}[fragile]{事件循环示例}
    \begin{columns}
        \begin{column}{0.48\textwidth}
            \inputminted[firstline=1,lastline=20]{cpp}{code/qt_event_loop_example.cpp}
        \end{column}
        \begin{column}{0.48\textwidth}
            \inputminted[firstline=21,lastline=40]{cpp}{code/qt_event_loop_example.cpp}
        \end{column}
    \end{columns}
\end{frame}

\begin{frame}[fragile]{事件过滤器(Event Filter)}
    \begin{ytublock}{事件过滤器简介}
        \begin{itemize}
            \item Qt事件过滤器允许一个对象监视和拦截另一个QObject对象的事件。。
            \item 常用于全局快捷键、特殊控件行为、日志记录等场景。
            \item 安装事件过滤器:\texttt{target->installEventFilter(filterObject);}
            \item 移除事件过滤器:\texttt{target->removeEventFilter(filterObject);}
        \end{itemize}
    \end{ytublock}
    \begin{ytublock}{事件过滤器实现}
        \begin{itemize}
            \item 定义一个自定义类,继承自 \texttt{QObject},并重写 \texttt{eventFilter(QObject *watched, QEvent *event)} 方法。
            \item 创建事件过滤器对象,并通过 \texttt{installEventFilter()} 安装到目标对象上。
            \item 在 \texttt{eventFilter} 方法中判断事件类型,进行自定义处理,返回 \texttt{true} 表示事件被拦截,\texttt{false} 表示继续传递。
        \end{itemize}
    \end{ytublock}
\end{frame}

\begin{frame}[fragile]{事件过滤器示例}
    \begin{columns}
        \begin{column}{0.48\textwidth}
            \inputminted[firstline=1,lastline=19]{cpp}{code/qt_event_filter_example.cpp}
        \end{column}
        \begin{column}{0.48\textwidth}
            \inputminted[firstline=20,lastline=38]{cpp}{code/qt_event_filter_example.cpp}
        \end{column}
    \end{columns}
\end{frame}

\begin{frame}[fragile]{事件处理器(Event Handler)}
    \begin{ytublock}{事件处理器简介}
        \begin{itemize}
            \item Qt事件处理器允许一个对象处理另一个QObject对象的事件。
            \item 通过重写 \texttt{event(QEvent *event)} 方法,可以在事件到达目标对象前进行处理或拦截。
            \item 常用于全局快捷键、特殊控件行为、日志记录等场景。
        \end{itemize}
    \end{ytublock}
    \begin{ytublock}{事件处理器实现}
        \begin{itemize}
            \item 定义一个自定义类,继承自 \texttt{QObject},并重写 \texttt{event(QEvent *event)} 方法。
        \end{itemize}
    \end{ytublock}
\end{frame}

\begin{frame}[fragile]{事件处理器示例}
    \begin{columns}
        \begin{column}{0.48\textwidth}
            \inputminted[firstline=1,lastline=14]{cpp}{code/qt_event_handler_example.cpp}
        \end{column}
        \begin{column}{0.48\textwidth}
            \inputminted[firstline=15,lastline=27]{cpp}{code/qt_event_handler_example.cpp}
        \end{column}
    \end{columns}
\end{frame}

\section{Qt容器类}
\begin{frame}{目录}
    \begin{multicols}{2}
        \tableofcontents[currentsection]
    \end{multicols}
\end{frame}

\begin{frame}{Qt容器类概述}
    \begin{ytublock}{Qt容器特点}
        \begin{itemize}
            \item 隐式共享(Implicit Sharing): 多个对象共享同一块内存,当其中一个对象修改时,会创建一个新的副本。
            \item 写时复制(Copy-on-Write): 当一个对象被修改时,会创建一个新的副本。
            \item 内存效率高: 容器类使用内存池管理内存,避免频繁的内存分配和释放。
            \item 线程安全: 容器类是线程安全的,可以在多线程环境下使用。
            \item 类型安全: 容器类是类型安全的,可以存储任意类型的数据,可以动态调整大小。
            \item std兼容: 容器类是std兼容的,可以与std容器相互转换。
            \item 算法支持: 容器类支持算法,可以对容器中的数据进行排序、查找、统计等操作。
        \end{itemize}
    \end{ytublock}
\end{frame}

\begin{frame}{Qt容器类对比}
    \begin{ytublock}{常用容器特性一览}
        \begin{table}[H]
            \centering
            \renewcommand{\arraystretch}{1.2}
            \begin{tabular}{|p{2.6cm}|p{2.5cm}|p{5.5cm}|}
                \hline
                \textbf{容器类型} & \textbf{底层结构} & \textbf{主要特点} \\
                \hline
                \texttt{QList/QVector} & 动态数组 & 动态增删高效,随机访问快,适合频繁插入/删除 \\
                \hline
                \texttt{QMap} & 平衡二叉树 & 按键有序,查找/插入/删除效率高 \\
                \hline
                \texttt{QHash} & 哈希表 & 键无序,查找/插入/删除极快 \\
                \hline
                \texttt{QSet} & 哈希表/平衡树 & 唯一元素,查找/插入/删除高效 \\
                \hline
                \texttt{QStack} & 顺序容器封装 & LIFO结构,只访问栈顶 \\
                \hline
                \texttt{QQueue} & 顺序容器封装 & FIFO结构,只访问队首/队尾 \\
                \hline
            \end{tabular}
        \end{table}
    \end{ytublock}
\end{frame}

\begin{frame}[fragile]{QList使用示例}
    \begin{columns}
        \begin{column}{0.48\textwidth}
            \inputminted[firstline=1,lastline=19]{cpp}{code/qt_list_example.cpp}
        \end{column}
        \begin{column}{0.48\textwidth}
            \inputminted[firstline=20,lastline=38]{cpp}{code/qt_list_example.cpp}
        \end{column}
    \end{columns}
\end{frame}

\begin{frame}[fragile]{QMap使用示例}
    \begin{columns}
        \begin{column}{0.48\textwidth}
            \inputminted[firstline=1,lastline=19]{cpp}{code/qt_map_example.cpp}
        \end{column}
        \begin{column}{0.48\textwidth}
            \inputminted[firstline=20,lastline=38]{cpp}{code/qt_map_example.cpp}
        \end{column}
    \end{columns}
\end{frame}

\begin{frame}[fragile]{QHash使用示例}
    \begin{columns}
        \begin{column}{0.48\textwidth}
            \inputminted[firstline=1,lastline=19]{cpp}{code/qt_hash_example.cpp}
        \end{column}
        \begin{column}{0.48\textwidth}
            \inputminted[firstline=20,lastline=38]{cpp}{code/qt_hash_example.cpp}
        \end{column}
    \end{columns}
\end{frame}

\begin{frame}[fragile]{QSet使用示例}
    \begin{columns}
        \begin{column}{0.48\textwidth}
            \inputminted[firstline=1,lastline=18]{cpp}{code/qt_set_example.cpp}
        \end{column}
        \begin{column}{0.48\textwidth}
            \inputminted[firstline=19,lastline=35]{cpp}{code/qt_set_example.cpp}
        \end{column}
    \end{columns}
\end{frame}

\begin{frame}[fragile]{QStack使用示例}
    \begin{columns}
        \begin{column}{0.48\textwidth}
            \inputminted[firstline=1,lastline=16]{cpp}{code/qt_stack_example.cpp}
        \end{column}
        \begin{column}{0.48\textwidth}
            \inputminted[firstline=17,lastline=34]{cpp}{code/qt_stack_example.cpp}
        \end{column}
    \end{columns}
\end{frame}

\begin{frame}[fragile]{QQueue使用示例}
    \begin{columns}
        \begin{column}{0.48\textwidth}
            \inputminted[firstline=1,lastline=17]{cpp}{code/qt_queue_example.cpp}
        \end{column}
        \begin{column}{0.48\textwidth}
            \inputminted[firstline=18,lastline=34]{cpp}{code/qt_queue_example.cpp}
        \end{column}
    \end{columns}
\end{frame}

\section{Qt工具类}
\begin{frame}{目录}
    \begin{multicols}{2}
        \tableofcontents[currentsection]
    \end{multicols}
\end{frame}

\begin{frame}{QString}
    \begin{ytublock}{QString简介}
        \begin{itemize}
            \item \textbf{QString} 是Qt中用于处理Unicode字符串的核心类,支持多语言环境下的高效字符串处理。
            \item 提供丰富的字符串操作:高效的拼接、查找、替换、分割、格式化、截取、大小写转换、去除空白、插入、移除、重复等。
            \item 支持与标准C++字符串类型(如\texttt{std::string}、\texttt{char*})的相互转换,便于与C++标准库协作。
            \item 可以与Qt的其他类(如\texttt{QTextStream}、\texttt{QFile}、\texttt{QByteArray}等)无缝集成,广泛应用于界面、文件IO、网络通信和数据处理等场景。
            \item 内置对Unicode的支持,适合国际化应用开发,能够正确处理多种语言和字符集。
            \item 提供灵活的格式化方法(如\texttt{arg()}、静态格式化等),便于生成动态文本。
            \item 支持与\texttt{QStringList}等容器类协作,方便字符串的批量处理和转换。
        \end{itemize}
    \end{ytublock}
\end{frame}

\begin{frame}[fragile]{QString使用示例(一)}
    \begin{columns}
        \begin{column}{0.48\textwidth}
            \inputminted[firstline=1,lastline=16]{cpp}{code/qt_string_example.cpp}
        \end{column}
        \begin{column}{0.48\textwidth}
            \inputminted[firstline=18,lastline=35]{cpp}{code/qt_string_example.cpp}
        \end{column}
    \end{columns}
\end{frame}

\begin{frame}[fragile]{QString使用示例(二)}
    \begin{columns}
        \begin{column}{0.48\textwidth}
            \inputminted[firstline=37,lastline=55]{cpp}{code/qt_string_example.cpp}
        \end{column}
        \begin{column}{0.48\textwidth}
            \inputminted[firstline=57,lastline=70]{cpp}{code/qt_string_example.cpp}
        \end{column}
    \end{columns}
\end{frame}

\begin{frame}[fragile]{QString使用示例(三)}
    \begin{columns}
        \begin{column}{0.48\textwidth}
            \inputminted[firstline=72,lastline=85]{cpp}{code/qt_string_example.cpp}
        \end{column}
        \begin{column}{0.48\textwidth}
            \inputminted[firstline=87,lastline=101]{cpp}{code/qt_string_example.cpp}
        \end{column}
    \end{columns}
\end{frame}

\begin{frame}[fragile]{QStringList使用示例(一)}
    \begin{columns}
        \begin{column}{0.48\textwidth}
            \inputminted[firstline=1,lastline=17]{cpp}{code/qt_string_list_example.cpp}
        \end{column}
        \begin{column}{0.48\textwidth}
            \inputminted[firstline=18,lastline=30]{cpp}{code/qt_string_list_example.cpp}
        \end{column}
    \end{columns}
\end{frame}

\begin{frame}[fragile]{QStringList使用示例(二)}

    \begin{columns}
        \begin{column}{0.48\textwidth}
            \inputminted[firstline=31,lastline=42]{cpp}{code/qt_string_list_example.cpp}
        \end{column}
        \begin{column}{0.48\textwidth}
            \inputminted[firstline=43,lastline=54]{cpp}{code/qt_string_list_example.cpp}
        \end{column}
    \end{columns}
\end{frame}

\begin{frame}{QByteArray}
    \begin{ytublock}{QByteArray简介}
        \begin{itemize}
            \item \textbf{QByteArray} 是Qt中专门用于高效处理原始二进制数据和字节序列的类,底层采用引用计数和写时复制(copy-on-write)机制,内存管理高效。
            \item 支持与C++标准库的 \texttt{std::string}、\texttt{char*}、\texttt{std::vector<char>} 等类型的无缝互转,便于与标准C++代码协作。
            \item 可与Qt的 \texttt{QTextStream}、\texttt{QFile}、\texttt{QIODevice}、\texttt{QDataStream} 等类直接配合,广泛应用于文件读写、网络通信、数据序列化、加密解密等场景。
            \item 支持多种编码(如Base64、十六进制、URL编码等)与解码操作,便于数据的存储与传输。
            \item 提供丰富的成员函数:如 \texttt{append()}、\texttt{insert()}、\texttt{remove()}、\texttt{replace()}、\texttt{split()}、\texttt{toInt()}、\texttt{toHex()}、\texttt{fromHex()}、\texttt{toBase64()}、\texttt{fromBase64()} 等,满足各种二进制和文本数据处理需求。
            \item 可以与 \texttt{QString} 互相转换,便于文本与二进制数据的混合处理。
            \item 支持直接操作底层数据指针(\texttt{data()}、\texttt{constData()}),适合高性能场景。
        \end{itemize}
    \end{ytublock}
\end{frame}

\begin{frame}[fragile]{QByteArray使用示例(一)}
    \begin{columns}
        \begin{column}{0.48\textwidth}
            \inputminted[firstline=1,lastline=17]{cpp}{code/qt_bytearray_example.cpp}
        \end{column}
        \begin{column}{0.48\textwidth}
            \inputminted[firstline=18,lastline=34]{cpp}{code/qt_bytearray_example.cpp}
        \end{column}
    \end{columns}
\end{frame}

\begin{frame}[fragile]{QByteArray使用示例(二)}
    \begin{columns}
        \begin{column}{0.48\textwidth}
            \inputminted[firstline=35,lastline=52]{cpp}{code/qt_bytearray_example.cpp}
        \end{column}
        \begin{column}{0.48\textwidth}
            \inputminted[firstline=53,lastline=68]{cpp}{code/qt_bytearray_example.cpp}
        \end{column}
    \end{columns}
\end{frame}

\begin{frame}[fragile]{QByteArray使用示例(三)}
    \begin{columns}
        \begin{column}{0.48\textwidth}
            \inputminted[firstline=69,lastline=85]{cpp}{code/qt_bytearray_example.cpp}
        \end{column}
        \begin{column}{0.48\textwidth}
            \inputminted[firstline=86,lastline=95]{cpp}{code/qt_bytearray_example.cpp}
        \end{column}
    \end{columns}
\end{frame}

\begin{frame}{QVariant}
    \begin{ytublock}{QVariant简介}
        \begin{itemize}
            \item \textbf{QVariant} 是Qt中用于存储和管理任意类型数据的通用容器类,底层实现了类型安全的类型擦除(type erasure)。
            \item 可以存储Qt内置的各种基本类型(如 \texttt{int}、\texttt{double}、\texttt{bool}、\texttt{QString}、\texttt{QDateTime}、\texttt{QByteArray} 等)、Qt容器类(如 \texttt{QStringList}、\texttt{QList<int>} 等),也支持注册自定义类型。
            \item 常用于通用接口、模型/视图框架(如 \texttt{QAbstractItemModel})、动态属性、信号槽参数等场景,实现类型无关的数据传递和存储。
            \item 提供类型判断(\texttt{type()}、\texttt{typeName()})、类型转换(\texttt{toInt()}、\texttt{toString()}、\texttt{toList()} 等)、类型安全访问(\texttt{value<T>()})、类型设置(\texttt{setValue()})等丰富接口。
            \item 支持与C++标准类型的互转,便于与标准库协作。
            \item 通过 \texttt{QVariant::isNull()} 和 \texttt{QVariant::isValid()} 判断内容是否为空或有效。
            \item 适合需要存储"任意类型"或"类型不确定"数据的场景,是Qt元对象系统和动态特性的基础之一。
        \end{itemize}
    \end{ytublock}
\end{frame}

\begin{frame}[fragile]{QVariant使用示例(一)}
    \begin{columns}
        \begin{column}{0.48\textwidth}
            \inputminted[firstline=1,lastline=16]{cpp}{code/qt_variant_example.cpp}
        \end{column}
        \begin{column}{0.48\textwidth}
            \inputminted[firstline=18,lastline=30]{cpp}{code/qt_variant_example.cpp}
        \end{column}
    \end{columns}
\end{frame}

\begin{frame}[fragile]{QVariant使用示例(二)}
    \begin{columns}
        \begin{column}{0.48\textwidth}
            \inputminted[firstline=31,lastline=46]{cpp}{code/qt_variant_example.cpp}
        \end{column}
        \begin{column}{0.48\textwidth}
            \inputminted[firstline=47,lastline=61]{cpp}{code/qt_variant_example.cpp}
        \end{column}
    \end{columns}
\end{frame}

\begin{frame}[fragile]{QVariant使用示例(三)}
    \begin{columns}
        \begin{column}{0.48\textwidth}
            \inputminted[firstline=62,lastline=79]{cpp}{code/qt_variant_example.cpp}
        \end{column}
        \begin{column}{0.48\textwidth}
            \inputminted[firstline=80,lastline=93]{cpp}{code/qt_variant_example.cpp}
        \end{column}
    \end{columns}
\end{frame}

\begin{frame}[fragile]{QVariant使用示例(四)}
    \begin{columns}
        \begin{column}{0.48\textwidth}
            \inputminted[firstline=94,lastline=113]{cpp}{code/qt_variant_example.cpp}
        \end{column}
        \begin{column}{0.48\textwidth}
            \inputminted[firstline=12,lastline=16]{cpp}{code/qt_variant_example.cpp}
        \end{column}
    \end{columns}
\end{frame}

\begin{frame}{QFile}
    \begin{ytublock}{QFile简介}
        \begin{itemize}
            \item \textbf{QFile} 是Qt中用于文件操作的核心类,继承自 \texttt{QIODevice},可用于对本地文件进行读写、创建、删除、重命名、拷贝等操作。
            \item 支持文本和二进制文件的读写,能够以只读、只写、读写、追加等多种模式打开文件。
            \item 常用方法包括 \texttt{open()}(打开文件)、\texttt{close()}(关闭文件)、\texttt{read()}、\texttt{write()}、\texttt{exists()}、\texttt{remove()}、\texttt{rename()}、\texttt{copy()} 等。
            \item 支持与 \texttt{QTextStream}、\texttt{QDataStream} 等流类配合,实现高效的文本和二进制数据读写。
            \item 可通过 \texttt{QFileInfo} 获取文件的详细信息(如大小、创建时间、权限等)。
            \item 跨平台,自动处理不同操作系统下的文件路径和编码问题。
            \item 适用于配置文件、日志文件、数据文件等各种文件操作场景。
        \end{itemize}
    \end{ytublock}
\end{frame}

\begin{frame}[fragile]{QFile使用示例(一)}
    \begin{columns}
        \begin{column}{0.48\textwidth}
            \inputminted[firstline=1,lastline=16]{cpp}{code/qt_file_example.cpp}
        \end{column}
        \begin{column}{0.48\textwidth}
            \inputminted[firstline=17,lastline=33]{cpp}{code/qt_file_example.cpp}
        \end{column}
    \end{columns}
\end{frame}

\begin{frame}[fragile]{QFile使用示例(二)}
    \begin{columns}
        \begin{column}{0.48\textwidth}
            \inputminted[firstline=34,lastline=51]{cpp}{code/qt_file_example.cpp}
        \end{column}
        \begin{column}{0.48\textwidth}
            \inputminted[firstline=52,lastline=67]{cpp}{code/qt_file_example.cpp}
        \end{column}
    \end{columns}
\end{frame}

\begin{frame}[fragile]{QFile使用示例(三)}
    \begin{columns}
        \begin{column}{0.48\textwidth}
            \inputminted[firstline=68,lastline=81]{cpp}{code/qt_file_example.cpp}
        \end{column}
        \begin{column}{0.48\textwidth}
            \inputminted[firstline=82,lastline=93]{cpp}{code/qt_file_example.cpp}
        \end{column}
    \end{columns}

\end{frame}

\begin{frame}[fragile]{QDir使用示例(一)}
    \begin{columns}
        \begin{column}{0.48\textwidth}
            \inputminted[firstline=1,lastline=16]{cpp}{code/qt_dir_example.cpp}
        \end{column}
        \begin{column}{0.48\textwidth}
            \inputminted[firstline=17,lastline=34]{cpp}{code/qt_dir_example.cpp}
        \end{column}
    \end{columns}
\end{frame}

\begin{frame}[fragile]{QDir使用示例(二)}
    \begin{columns}
        \begin{column}{0.48\textwidth}
            \inputminted[firstline=35,lastline=53]{cpp}{code/qt_dir_example.cpp}
        \end{column}
        \begin{column}{0.48\textwidth}
            \inputminted[firstline=54,lastline=71]{cpp}{code/qt_dir_example.cpp}
        \end{column}
    \end{columns}
\end{frame}

\begin{frame}[fragile]{QDir使用示例(三)}
    \begin{columns}
        \begin{column}{0.48\textwidth}
            \inputminted[firstline=72,lastline=85]{cpp}{code/qt_dir_example.cpp}
        \end{column}
        \begin{column}{0.48\textwidth}
            \inputminted[firstline=86,lastline=98]{cpp}{code/qt_dir_example.cpp}
        \end{column}
    \end{columns}
\end{frame}

\begin{frame}[fragile]{QDir使用示例(四)}
    \inputminted[firstline=99,lastline=119]{cpp}{code/qt_dir_example.cpp}
\end{frame}

\begin{frame}[fragile]{QFileInfo使用示例}
    \begin{columns}
        \begin{column}{0.48\textwidth}
            \inputminted[firstline=1,lastline=23]{cpp}{code/qt_fileinfo_example.cpp}
        \end{column}
        \begin{column}{0.48\textwidth}
            \inputminted[firstline=24,lastline=42]{cpp}{code/qt_fileinfo_example.cpp}
        \end{column}
    \end{columns}
\end{frame}

\begin{frame}[fragile]{QTextStream简介}
    \begin{itemize}
        \item \textbf{QTextStream} 类是Qt中用于文本流操作的类,主要用于读写文本文件。
        \item 继承自 \textbf{QIODevice},可以像操作二进制文件一样操作文本文件。
        \item 提供了简单易用的接口,支持多种文本编码(如UTF-8、UTF-16、UTF-32等)。
        \item 常用于文件的逐行读取、写入、格式化输出等场景。
    \end{itemize}
\end{frame}

\begin{frame}[fragile]{QTextStream使用示例}
    \begin{columns}
        \begin{column}{0.48\textwidth}
            \inputminted[firstline=1,lastline=19]{cpp}{code/qt_textstream_example.cpp}
        \end{column}
        \begin{column}{0.48\textwidth}
            \inputminted[firstline=21,lastline=34]{cpp}{code/qt_textstream_example.cpp}
        \end{column}
    \end{columns}
\end{frame}

\begin{frame}[fragile]{QDataStream简介}
    \begin{itemize}
        \item \textbf{QDataStream} 类是Qt中用于二进制数据流读写的类,常用于对象、结构体、基本数据类型的序列化与反序列化。
        \item 支持多种Qt数据类型(如QString、QList、QMap等)以及自定义类型的读写。
        \item 主要用于文件、网络等场景下的数据持久化和传输。
        \item 通过重载 \texttt{<<} 和 \texttt{>>} 运算符实现数据的序列化与反序列化。
        \item 可以设置版本号,保证不同Qt版本间的数据兼容性。
    \end{itemize}
\end{frame}

\begin{frame}[fragile]{QDataStream使用示例}
    \begin{columns}
        \begin{column}{0.48\textwidth}
            \inputminted[firstline=1,lastline=23]{cpp}{code/qt_datastream_example.cpp}
        \end{column}
        \begin{column}{0.48\textwidth}
            \inputminted[firstline=24,lastline=42]{cpp}{code/qt_datastream_example.cpp}
        \end{column}
    \end{columns}
\end{frame}

\begin{frame}[fragile]{QT日期与时间}
    \begin{ytublock}{QDateTime简介}
    \begin{itemize}
        \item \textbf{QDateTime} 类是Qt中用于处理日期和时间的类,提供了一个统一的接口来处理各种日期和时间相关的操作。
        \item 支持多种时区、时区转换、日期计算、格式化输出等。
        \item 常用于日期时间计算、格式化输出、数据库操作等场景。
        \item 支持与QDate、QTime、QElapsedTimer等类协作,方便处理日期时间相关操作。
    \end{itemize}
    \end{ytublock}
\end{frame}

\begin{frame}[fragile]{QT日期与时间示例(一)}
    \begin{columns}
        \begin{column}{0.48\textwidth}
            \inputminted[firstline=1,lastline=19]{cpp}{code/qt_datetime_example.cpp}
        \end{column}
        \begin{column}{0.48\textwidth}
            \inputminted[firstline=20,lastline=32]{cpp}{code/qt_datetime_example.cpp}
        \end{column}
    \end{columns}
\end{frame}

\begin{frame}[fragile]{QT日期与时间示例(二)}
    \begin{columns}
        \begin{column}{0.48\textwidth}
            \inputminted[firstline=33,lastline=45]{cpp}{code/qt_datetime_example.cpp}
        \end{column}
        \begin{column}{0.48\textwidth}
            \inputminted[firstline=46,lastline=62]{cpp}{code/qt_datetime_example.cpp}
        \end{column}
    \end{columns}
\end{frame}

\begin{frame}[fragile]{QT日期与时间示例(三)}
    \begin{columns}
        \begin{column}{0.48\textwidth}
            \inputminted[firstline=63,lastline=82]{cpp}{code/qt_datetime_example.cpp}
        \end{column}
        \begin{column}{0.48\textwidth}
            \inputminted[firstline=83,lastline=97]{cpp}{code/qt_datetime_example.cpp}
        \end{column}
    \end{columns}
\end{frame}

\begin{frame}{QProcess}
    \begin{ytublock}{QProcess简介}
        \begin{itemize}
            \item \textbf{QProcess} 类是Qt中用于启动和管理外部进程的类,提供了丰富的接口用于与外部程序进行交互。
            \item 支持异步和同步启动外部进程,可以读取和写入子进程的标准输入、标准输出和标准错误。
            \item 可用于执行系统命令、脚本、调用其他可执行程序,并获取其输出结果。
            \item 支持跨平台,自动处理不同操作系统下的进程调用方式。
            \item 常用方法包括 \texttt{start()}(启动进程)、\texttt{waitForFinished()}(等待进程结束)、\texttt{readAllStandardOutput()}(读取标准输出)、\texttt{readAllStandardError()}(读取标准错误)、\texttt{write()}(向进程写入数据)等。
            \item 适用于需要与外部程序协作、自动化脚本、批处理等场景。
        \end{itemize}
    \end{ytublock}
\end{frame}

\begin{frame}[fragile]{QProcess使用示例}
    \begin{columns}
        \begin{column}{0.48\textwidth}
            \inputminted[firstline=1,lastline=21]{cpp}{code/qt_process_example.cpp}
        \end{column}
        \begin{column}{0.48\textwidth}
            \inputminted[firstline=22,lastline=40]{cpp}{code/qt_process_example.cpp}
        \end{column}
    \end{columns}
\end{frame}

\begin{frame}{QSettings}
    \begin{ytublock}{QSettings简介}
        \begin{itemize}
            \item \textbf{QSettings} 类是Qt中用于读写应用程序配置文件的工具类,支持跨平台,自动适配不同操作系统的配置存储方式(如Windows注册表、INI文件、macOS的plist等)。
            \item 通过键值对的方式存储和读取各种类型的数据(如int、double、QString、QVariant等),支持分组(Group)和层级结构,便于组织复杂配置。
            \item 常用方法包括 \texttt{setValue(key, value)}(写入配置)、\texttt{value(key, default)}(读取配置)、\texttt{remove(key)}(删除配置)、\texttt{contains(key)}(判断配置项是否存在)等。
            \item 支持自动保存和加载,无需手动管理文件读写,适合保存用户设置、应用参数、窗口状态等。
            \item 支持多种构造方式,可指定组织名、应用名、文件格式等,灵活适配不同项目需求。
            \item 典型用法示例见下页代码。
        \end{itemize}
    \end{ytublock}
\end{frame}

\begin{frame}[fragile]{QSettings使用示例(一)}
    \begin{columns}
        \begin{column}{0.48\textwidth}
            \inputminted[firstline=1,lastline=16]{cpp}{code/qt_settings_example.cpp}
        \end{column}
        \begin{column}{0.48\textwidth}
            \inputminted[firstline=17,lastline=32]{cpp}{code/qt_settings_example.cpp}
        \end{column}
    \end{columns}
\end{frame}

\begin{frame}[fragile]{QSettings使用示例(二)}
    \begin{columns}
        \begin{column}{0.48\textwidth}
            \inputminted[firstline=33,lastline=53]{cpp}{code/qt_settings_example.cpp}
        \end{column}
        \begin{column}{0.48\textwidth}
            \begin{ytublock}{Ini文件内容}
                \begin{verbatim}
[MyCompany/MyApp]
username=alice
window/width=800
window/height=600
volume=0.75
isAdmin=true
[network]
host=192.168.1.100
port=8080
                \end{verbatim}
            \end{ytublock}
        \end{column}
    \end{columns}
\end{frame}

\begin{frame}{QTimer}
    \begin{ytublock}{QTimer简介}
        \begin{itemize}
            \item \textbf{QTimer} 类是Qt中用于定时器的类,提供了简单的定时器功能。
            \item 支持单次定时和多次定时,可以设置定时器的间隔时间。
            \item 常用方法包括 \texttt{start()}(启动定时器)、\texttt{stop()}(停止定时器)、\texttt{setInterval()}(设置定时器间隔时间)、\texttt{setSingleShot()}(设置定时器是否单次定时)等。
            \item 适用于需要定时执行某些操作的场景。
        \end{itemize}
    \end{ytublock}
\end{frame}

\begin{frame}[fragile]{QTimer使用示例}
    \begin{columns}
        \begin{column}{0.48\textwidth}
            \inputminted[firstline=1,lastline=15]{cpp}{code/qt_timer_example.cpp}
        \end{column}
        \begin{column}{0.48\textwidth}
            \inputminted[firstline=17,lastline=34]{cpp}{code/qt_timer_example.cpp}
        \end{column}
    \end{columns}
\end{frame}

\begin{frame}{QThread}
    \begin{ytublock}{QThread简介}
        \begin{itemize}
            \item \textbf{QThread} 类是Qt中用于线程的类,提供了简单的线程功能。
            \item 支持线程的创建、启动、停止、暂停、恢复等操作。
            \item 常用方法包括 \texttt{start()}(启动线程)、\texttt{stop()}(停止线程)、\texttt{pause()}(暂停线程)、\texttt{resume()}(恢复线程)等。
            \item 适用于需要多线程处理的场景。
        \end{itemize}
    \end{ytublock}
\end{frame}

\begin{frame}[fragile]{QThread使用示例}
    \begin{columns}
        \begin{column}{0.48\textwidth}
            \inputminted[firstline=1,lastline=15]{cpp}{code/qt_thread_example.cpp}
        \end{column}
        \begin{column}{0.48\textwidth}
            \inputminted[firstline=17,lastline=32]{cpp}{code/qt_thread_example.cpp}
        \end{column}
    \end{columns}
\end{frame}

\begin{frame}{QtConcurrent}
    \begin{ytublock}{QtConcurrent简介}
        \begin{itemize}
            \item \textbf{QtConcurrent} 类是Qt中用于并行计算的类,提供了简单的并行计算功能。
            \item 支持并行计算,可以设置并行计算的线程数。
            \item 常用方法包括 \texttt{run()}(启动并行计算)、\texttt{waitForFinished()}(等待并行计算结束)等。
            \item 适用于需要并行计算的场景。
        \end{itemize}
    \end{ytublock}
\end{frame}

\begin{frame}[fragile]{QtConcurrent使用示例}
    \begin{columns}
        \begin{column}{0.48\textwidth}
            \inputminted[firstline=1,lastline=19]{cpp}{code/qt_concurrent_example.cpp}
        \end{column}
        \begin{column}{0.48\textwidth}
            \inputminted[firstline=20,lastline=38]{cpp}{code/qt_concurrent_example.cpp}
        \end{column}
    \end{columns}
\end{frame}

\begin{frame}{QPainter}
    \begin{ytublock}{QPainter简介}
        \begin{itemize}
            \item \textbf{QPainter} 类是Qt中用于绘图的类,提供了强大的绘图功能。
            \item 支持绘制各种图形、文本、图像等。
            \item 支持路径、笔刷、渐变等。
            \item 支持绘制到各种QPaintDevice,如QWidget、QImage等。
            \item 常用在paintEvent()中进行自定义绘制。
            \item 支持多种绘制模式,如填充、描边、剪切等。
            \item 支持多种绘制效果,如阴影、透明、渐变等。
        \end{itemize}
    \end{ytublock}
\end{frame}

\begin{frame}[fragile]{QPainter使用示例(一)}
    \begin{columns}
        \begin{column}{0.48\textwidth}
            \inputminted[firstline=1,lastline=19]{cpp}{code/qt_painter_example.cpp}
        \end{column}
        \begin{column}{0.48\textwidth}
            \inputminted[firstline=20,lastline=35]{cpp}{code/qt_painter_example.cpp}
        \end{column}
    \end{columns}
\end{frame}

\begin{frame}[fragile]{QPainter使用示例(二)}
    \begin{columns}
        \begin{column}{0.48\textwidth}
            \inputminted[firstline=36,lastline=54]{cpp}{code/qt_painter_example.cpp}
        \end{column}
        \begin{column}{0.48\textwidth}
            \inputminted[firstline=55,lastline=71]{cpp}{code/qt_painter_example.cpp}
        \end{column}
    \end{columns}
\end{frame}

\begin{frame}[fragile]{QPainter使用示例(三)}
    \begin{columns}
        \begin{column}{0.48\textwidth}
            \inputminted[firstline=72,lastline=87]{cpp}{code/qt_painter_example.cpp}
        \end{column}
        \begin{column}{0.48\textwidth}
            \inputminted[firstline=88,lastline=96]{cpp}{code/qt_painter_example.cpp}
        \end{column}
    \end{columns}
\end{frame}


\section{总结}
\begin{frame}{目录}
    \begin{multicols}{2}
        \tableofcontents[currentsection]
    \end{multicols}
\end{frame}

\begin{frame}{总结}
    \begin{ytublock}{本章要点回顾}
        \begin{itemize}
            \item \textbf{Qt框架概述}:Qt是一个跨平台C++开发框架,提供GUI、网络、多媒体等模块,支持桌面、移动和嵌入式平台。
            \item \textbf{核心模块与结构}:包括QtCore、QtGui、QtWidgets等模块;应用程序基于QObject和元对象系统构建。
            \item \textbf{元对象系统}:通过Q\_OBJECT宏、moc编译器和QMetaObject实现运行时类型信息、信号槽和属性系统。
            \item \textbf{信号槽机制}:Qt的核心通信方式,支持对象间松耦合,支持多对多连接和跨线程通信,优于传统事件处理。
            \item \textbf{事件系统}:包括鼠标、键盘等事件;通过事件循环、过滤器和处理器实现灵活响应。
            \item \textbf{容器与工具类}:Qt提供高效容器(如QList、QMap)和工具(如QString、QByteArray、QFile、QTimer、QThread),支持隐式共享和线程安全。
            \item \textbf{Qt Widgets编程}:涵盖基本控件(如QWidget、QDialog、QMainWindow)和高级组件(如QPushButton、QLabel、QSlider、QProgressBar),用于构建桌面GUI。
        \end{itemize}
    \end{ytublock}
\end{frame}


\begin{frame}{建议}
    \begin{ytublock}{学习建议}
        \begin{itemize}
            \item 多动手实践信号槽和事件机制,尝试自定义信号与槽,深入理解Qt的动态特性和对象间通信原理。
            \item 结合本章示例代码,动手实现常见Widgets的创建、属性设置和信号响应,掌握控件的基本用法。
            \item 练习不同的布局管理器(如QHBoxLayout、QVBoxLayout、QGridLayout等),尝试实现复杂界面布局,提升界面设计能力。
            \item 阅读Qt官方文档和API手册,查阅控件的更多属性和方法,善用文档解决实际开发中的问题。
            \item 尝试将多个控件组合,开发小型实用工具或Demo项目,巩固所学知识。
            \item 主动探索高级主题,如多线程编程、网络通信、自定义控件绘制、样式表(QSS)美化等,拓展Qt开发视野。
            \item 参与Qt开源社区或查阅优秀开源项目,学习他人代码风格和工程组织方式,提升综合开发能力。
        \end{itemize}
    \end{ytublock}
\end{frame}

\end{document}
