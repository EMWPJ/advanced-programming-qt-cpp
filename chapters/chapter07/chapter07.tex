\documentclass[UTF8,aspectratio=169]{beamer}



% 基本包
\usepackage[UTF8]{ctex}
\usepackage{graphicx}
\usepackage{amsmath}
\usepackage{amsfonts}
\usepackage{amssymb}
% \usepackage{listings}  % 已替换为minted
\usepackage{xcolor}
\usepackage{hyperref}
\usepackage{booktabs}
\usepackage{multirow}
\usepackage{multicol}
\usepackage{float}
\usepackage{tikz}
\usetikzlibrary{positioning,shapes,arrows,fit,backgrounds}
\usepackage{pgfplots}
\pgfplotsset{compat=1.18}
\usepackage{minted}
\usepackage{fontspec}
\usepackage[most]{tcolorbox}

% Beamer主题设置
\usetheme{Madrid}
\usecolortheme{whale}

% 校徽设置
\logo{
  \IfFileExists{../长江大学校徽.pdf}{
    \begin{tikzpicture}[remember picture,overlay]
      \node[anchor=north east, xshift=-0.2mm, yshift=-0.2mm] at (current page.north east) {
        \includegraphics[height=1.0cm]{../长江大学校徽.pdf}
      };
    \end{tikzpicture}
  }{
    \begin{tikzpicture}[remember picture,overlay]
      \node[anchor=north east, xshift=-3mm, yshift=-3mm] at (current page.north east) {
        \textcolor{red}{\tiny [校徽文件未找到]}
      };
    \end{tikzpicture}
  }
}

% ===== 使用推荐的 font themes =====
\usefonttheme{professionalfonts}  % 允许自定义字体

% 自定义frame标题栏,也缩短长度留出logo空间
% 设置标题栏背景颜色为淡蓝色
\definecolor{frametitlebg}{RGB}{200,215,250} % 淡蓝色,可根据需要调整
\setbeamercolor{frametitle}{bg=frametitlebg, fg=ytublue!80!black}
\setbeamertemplate{frametitle}{
  \ifbeamercolorempty[bg]{frametitle}{}{\nointerlineskip}%
  \begin{tcolorbox}[
    enhanced,
    width=0.90\paperwidth,
    height=2.5ex,
    colback=frametitlebg,
    colframe=frametitlebg,
    boxrule=0pt,
    left=0pt,
    right=0pt,
    top=1pt,
    bottom=0pt,
    boxsep=0pt,
    before skip=0pt,
    after skip=0.1em,  % 减少标题栏和内容之间的间距
  ]
  \vspace{0.2ex} % 减少标题文字上方的空白
  \usebeamerfont{frametitle}\textcolor{ytublue!80!black}{\hspace{1em}\insertframetitle}
  \end{tcolorbox}
}

% ===== 设置现代字体 =====
\setsansfont{Source Sans Pro}     % 正文字体
\setmonofont{Source Code Pro}[Scale=0.9]  % 代码字体,稍微缩小一点

\setbeamertemplate{navigation symbols}{}

% 减少页面间距
\setbeamertemplate{itemize items}[circle]
\setbeamertemplate{enumerate items}[default]
\setlength{\itemsep}{0.1em}
\setlength{\parskip}{0.1em}

% 页码设置
\setbeamertemplate{footline}[frame number]

% 定义流程图样式
\tikzset{
    block/.style = {rectangle, draw, fill=blue!10,
        minimum width=6em, align=center, rounded corners, minimum height=3em},
    line/.style = {draw, -latex'}
}
% 水印设置
\setbeamertemplate{background}{
    \begin{tikzpicture}[remember picture,overlay]
        \node[rotate=-45,scale=0.8,opacity=0.1,color=gray]
             at ([xshift=0.5cm,yshift=0.5cm]current page.south west)
             {\large\textbf{WPJ}};
    \end{tikzpicture}
}

% 自定义颜色
\definecolor{qtgreen}{RGB}{41,128,185}
\definecolor{qtblue}{RGB}{52,73,94}

% 定义长江大学蓝主色调
\definecolor{ytublue}{RGB}{0,84,159}
% 统一block样式
\newtcolorbox{ytublock}[1]{
  colback=white,
  colframe=ytublue!80!black,
  colbacktitle=ytublue!20!white,
  coltitle=ytublue!80!black,
  title={#1},
  fonttitle=\bfseries,
  arc=3mm,
  boxrule=1pt,
  boxsep=1mm,
  left=2mm,
  right=2mm,
  top=0.5mm,
  bottom=0.5mm,
  before skip=3pt,
  after skip=3pt,
  enhanced,
  drop fuzzy shadow=ytublue!20!black
}

% 定义警告块样式
\newtcolorbox{ytualertblock}[1]{
  colback=white,
  colframe=red!80!black,
  colbacktitle=red!20!white,
  coltitle=red!80!black,
  title={#1},
  fonttitle=\bfseries,
  arc=3mm,
  boxrule=1.5pt,
  boxsep=1mm,
  left=2mm,
  right=2mm,
  top=0.5mm,
  bottom=0.5mm,
  before skip=3pt,
  after skip=3pt,
  enhanced,
  drop fuzzy shadow=red!20!black,
  overlay={
    \begin{tcbclipinterior}
      \fill[red!10!white] (interior.south west) rectangle (interior.north east);
    \end{tcbclipinterior}
  }
}

% cpp代码高亮设置
\setminted[cpp]{
    fontsize=\tiny,
    fontfamily=tt,             % 使用等宽字体
    linenos=true,
    frame=lines,               % 上下两条线,简洁清爽(比 tb 更现代)
    framesep=3mm,              % 内边距
    rulecolor=\color{blue!20}, % 线条颜色浅蓝,不刺眼
    bgcolor=blue!10,           % 浅蓝色背景
    baselinestretch=1.2,       % 行距稍大,更易读
    breaklines=true,
    breakautoindent=true,
    tabsize=4,
    xleftmargin=5mm,
    xrightmargin=5mm,
    numbersep=8pt,             % 行号与代码间距
    % ===== 其他美化 =====
    obeytabs=true,             % 尊重 tab 字符
    samepage=false,            % 允许跨页(重要!避免空白)
    escapeinside=||,           % 可在代码中使用 |LaTeX| 插入 LaTeX 命令
}

% 设置标题页颜色,与frame标题保持一致
\setbeamercolor{title}{bg=frametitlebg, fg=ytublue!80!black}
\setbeamercolor{subtitle}{bg=frametitlebg, fg=ytublue!70!black}
\setbeamercolor{author}{bg=frametitlebg, fg=ytublue!80!black}
\setbeamercolor{institute}{bg=frametitlebg, fg=ytublue!80!black}
\setbeamercolor{date}{bg=frametitlebg, fg=ytublue!80!black}

% 自定义标题页样式,全部内容同一个tcolorbox,居中排版,字体和间距区分
\setbeamertemplate{title page}{
  \vbox{}
  \begingroup
    \centering
    \begin{tcolorbox}[
      enhanced,
      width=0.92\paperwidth,
      colback=frametitlebg,
      colframe=frametitlebg,
      boxrule=0pt,
      left=0pt,
      right=0pt,
      top=4mm,
      bottom=4mm,
      boxsep=0pt,
      before skip=0pt,
      after skip=1.2em,
    ]
    % 标题
    {\centering
      {\fontsize{24pt}{27pt}\selectfont\textcolor{ytublue!80!black}{\inserttitle}\par}
      \vspace{1.2em}
      % 副标题
      {\fontsize{21pt}{24pt}\selectfont\textcolor{ytublue!70!black}{\insertsubtitle}\par}
      \vspace{2.0em}
      % 作者
      {\fontsize{12pt}{15pt}\selectfont\insertauthor\par}
      \vspace{0.7em}
      % 单位
      {\fontsize{12pt}{15pt}\selectfont\insertinstitute\par}
      \vspace{0.7em}
      % 日期
      {\fontsize{12pt}{15pt}\selectfont\insertdate\par}
    }
    \vspace{0.5em}
    \vfill
    \end{tcolorbox}
  \endgroup
}


% 文档信息
\title{高等程序设计 - Qt/C++}
\subtitle{第7章:C++科学计算}
\author{王培杰}
\institute{长江大学地球物理与石油资源学院}
\date{\today}

\begin{document}

% 标题页
\begin{frame}
    \titlepage
\end{frame}

% 目录页
\begin{frame}{目录}
    \begin{multicols}{2}
        \tableofcontents[hideothersubsections]
    \end{multicols}
\end{frame}

% ========== C++标准库cmath ==========

\section{C++标准库cmath}
\begin{frame}{目录}
    \begin{multicols}{2}
        \tableofcontents[currentsection,hideothersubsections]
    \end{multicols}
\end{frame}

\begin{frame}{C++标准库cmath简介}
    \begin{ytublock}{cmath库}
        \begin{itemize}
            \item C++标准库中的数学函数库
            \item 提供基础数学运算函数
            \item 包含三角函数、指数对数、幂函数等
            \item 无需额外安装,编译器自带
            \item 跨平台兼容性好
            \item 是学习科学计算的基础
        \end{itemize}
    \end{ytublock}
\end{frame}

\begin{frame}{cmath主要函数分类}
    \begin{columns}
        \begin{column}{0.48\textwidth}
            \begin{ytublock}{三角函数}
                \begin{itemize}
                    \item \texttt{sin}, \texttt{cos}, \texttt{tan} - 基本三角函数
                    \item \texttt{asin}, \texttt{acos}, \texttt{atan} - 反三角函数
                    \item \texttt{sinh}, \texttt{cosh}, \texttt{tanh} - 双曲函数
                    \item \texttt{atan2} - 双参数反正切
                \end{itemize}
            \end{ytublock}

            \begin{ytublock}{指数与对数}
                \begin{itemize}
                    \item \texttt{exp} - 自然指数
                    \item \texttt{log} - 自然对数
                    \item \texttt{log10} - 常用对数
                    \item \texttt{log2} - 以2为底的对数
                \end{itemize}
            \end{ytublock}
        \end{column}
        \begin{column}{0.48\textwidth}
            \begin{ytublock}{幂函数与取整}
                \begin{itemize}
                    \item \texttt{pow} - 幂运算
                    \item \texttt{sqrt} - 平方根
                    \item \texttt{cbrt} - 立方根
                    \item \texttt{ceil} - 向上取整
                    \item \texttt{floor} - 向下取整
                    \item \texttt{round} - 四舍五入
                \end{itemize}
            \end{ytublock}

            \begin{ytublock}{其他函数}
                \begin{itemize}
                    \item \texttt{abs}, \texttt{fabs} - 绝对值
                    \item \texttt{fmod} - 取模
                    \item \texttt{fmax}, \texttt{fmin} - 最值
                \end{itemize}
            \end{ytublock}
        \end{column}
    \end{columns}
\end{frame}

\begin{frame}{cmath使用示例}
    \begin{ytublock}{基本使用}
        \begin{itemize}
            \item \textbf{头文件}:\texttt{\#include <cmath>}
            \item \textbf{命名空间}:函数在\texttt{std}命名空间中
            \item \textbf{精度}:支持\texttt{float}, \texttt{double}, \texttt{long double}
        \end{itemize}
    \end{ytublock}

    \vspace{0.2cm}
    \begin{ytublock}{示例代码}
        \begin{itemize}
            \item \texttt{double x = std::sin(3.14159 / 2);} // 计算sin(π/2)
            \item \texttt{double y = std::pow(2.0, 3.0);} // 计算2³
            \item \texttt{double z = std::sqrt(16.0);} // 计算√16
            \item \texttt{double w = std::log(2.71828);} // 计算ln(e)
            \item \texttt{double angle = std::atan2(y, x);} // 计算角度
        \end{itemize}
    \end{ytublock}
\end{frame}

\begin{frame}{cmath的局限性}
    \begin{columns}
        \begin{column}{0.48\textwidth}
            \begin{ytublock}{功能限制}
                \begin{itemize}
                    \item 只提供基础数学函数
                    \item 缺少特殊函数(贝塞尔、伽马等)
                    \item 不支持高精度计算
                    \item 不支持向量化运算
                    \item 无统计分布函数
                \end{itemize}
            \end{ytublock}
        \end{column}
        \begin{column}{0.48\textwidth}
            \begin{ytublock}{性能限制}
                \begin{itemize}
                    \item 单元素计算,无SIMD优化
                    \item 不适合大规模向量运算
                    \item 无并行计算支持
                    \item 性能优化有限
                \end{itemize}
            \end{ytublock}

            \begin{ytublock}{学习路径}
                \begin{itemize}
                    \item cmath是基础,需要掌握
                    \item 进阶到Boost扩展功能
                    \item 使用Eigen进行矩阵运算
                    \item 使用MKL获得高性能
                \end{itemize}
            \end{ytublock}
        \end{column}
    \end{columns}
\end{frame}

% ========== Boost科学计算库 ==========

\section{Boost科学计算库}
\begin{frame}{目录}
    \begin{multicols}{2}
        \tableofcontents[currentsection,hideothersubsections]
    \end{multicols}
\end{frame}

\begin{frame}{Boost简介}
    \centering
    \includegraphics[width=0.95\textwidth]{images/boost/boost_overview.pdf}

\end{frame}

\begin{frame}{Boost.Math模块}
    \centering
    \includegraphics[width=0.95\textwidth]{images/boost/boost_math.pdf}
\end{frame}

\begin{frame}{Boost.Math详细功能}
    \begin{columns}
        \begin{column}{0.48\textwidth}
            \begin{ytublock}{特殊函数}
                \begin{itemize}
                    \item \textbf{贝塞尔函数}:
                      \begin{itemize}
                        \item \texttt{cyl\_bessel\_j(n, x)}
                        \item \texttt{cyl\_bessel\_i(n, x)}
                      \end{itemize}
                    \item \textbf{伽马函数}:
                      \begin{itemize}
                        \item \texttt{tgamma(x)} - 伽马函数
                        \item \texttt{lgamma(x)} - 对数伽马
                      \end{itemize}
                    \item \textbf{椭圆积分}:\texttt{ellint\_1(k)}, \texttt{ellint\_2(k)}
                    \item \textbf{误差函数}:\texttt{erf(x)}, \texttt{erfc(x)}
                \end{itemize}
            \end{ytublock}

            \begin{ytublock}{统计分布}
                \begin{itemize}
                    \item \textbf{连续分布}:正态、t、卡方、F分布
                    \item \textbf{离散分布}:二项、泊松、几何分布
                    \item \textbf{操作}:\texttt{pdf()}, \texttt{cdf()}, \texttt{quantile()}
                \end{itemize}
            \end{ytublock}
        \end{column}
        \begin{column}{0.48\textwidth}
            \begin{ytublock}{数值工具}
                \begin{itemize}
                    \item \textbf{求根算法}:
                      \begin{itemize}
                        \item \texttt{bisect()} - 二分法
                        \item \texttt{newton()} - 牛顿法
                      \end{itemize}
                    \item \textbf{插值方法}:线性插值、三次样条
                    \item \textbf{数值积分}:梯形法则、辛普森法则
                \end{itemize}
            \end{ytublock}

            \begin{ytublock}{安装与使用}
                \begin{itemize}
                    \item \textbf{包管理器}:
                      \begin{itemize}
                        \item Linux: \texttt{sudo apt-get install libboost-all-dev}
                        \item macOS: \texttt{brew install boost}
                      \end{itemize}
                    \item \textbf{CMake}:\texttt{find\_package(Boost REQUIRED COMPONENTS math)}
                    \item \textbf{头文件}:\texttt{\#include <boost/math/special\_functions.hpp>}
                \end{itemize}
            \end{ytublock}
        \end{column}
    \end{columns}
\end{frame}

\begin{frame}{Boost.Math使用示例}
    \begin{ytublock}{特殊函数示例}
        \begin{itemize}
            \item \texttt{double gamma\_val = boost::math::tgamma(5.0);} // 计算Γ(5)
            \item \texttt{double bessel\_j = boost::math::cyl\_bessel\_j(0, 1.0);} // J₀(1)
            \item \texttt{double erf\_val = boost::math::erf(1.0);} // 误差函数
        \end{itemize}
    \end{ytublock}

    \vspace{0.2cm}
    \begin{ytublock}{统计分布示例}
        \begin{itemize}
            \item \texttt{boost::math::normal\_distribution<> norm(0, 1);} // 标准正态分布
            \item \texttt{double prob = pdf(norm, 1.0);} // 概率密度
            \item \texttt{double cum = cdf(norm, 1.0);} // 累积分布
        \end{itemize}
    \end{ytublock}
\end{frame}

\begin{frame}{Boost.Multiprecision高精度计算}
    \centering
    \includegraphics[width=0.75\textwidth]{images/boost/boost_multiprecision.pdf}
\end{frame}

\begin{frame}{Boost.Multiprecision详细功能}
    \begin{columns}
        \begin{column}{0.48\textwidth}
            \begin{ytublock}{高精度整数}
                \begin{itemize}
                    \item \texttt{cpp\_int} - 任意精度整数
                    \item \texttt{int128\_t}, \texttt{int256\_t} - 固定精度
                    \item 支持标准运算符:+、-、*、/、\%
                    \item 支持高级函数:\texttt{sqrt()}
                    \item 无溢出问题
                \end{itemize}
            \end{ytublock}

            \begin{ytublock}{高精度浮点}
                \begin{itemize}
                    \item \texttt{cpp\_dec\_float\_50} - 50位小数
                    \item \texttt{cpp\_dec\_float\_100} - 100位小数
                    \item \texttt{cpp\_bin\_float} - 二进制浮点(更高精度)
                    \item 精确的十进制表示,适合金融计算
                \end{itemize}
            \end{ytublock}
        \end{column}
        \begin{column}{0.48\textwidth}
            \begin{ytublock}{使用示例}
                \begin{itemize}
                    \item \texttt{\#include <boost/multiprecision/cpp\_int.hpp>}
                    \item \texttt{using namespace boost::multiprecision;}
                    \item \texttt{cpp\_int n = 12345678901234567890;}
                    \item \texttt{cpp\_int result = n * n;} // 无溢出
                    \item \texttt{cpp\_dec\_float\_50 pi = 3.141592653589793238462643383279502884197;}
                \end{itemize}
            \end{ytublock}

        \end{column}
    \end{columns}
\end{frame}

\begin{frame}{Boost.Random随机数生成}
    \centering
    \includegraphics[width=0.6\textwidth]{images/boost/boost_random.pdf}
\end{frame}

\begin{frame}{Boost.Accumulators统计累加器}
    \centering
    \includegraphics[width=0.8\textwidth]{images/boost/boost_accumulators.pdf}
\end{frame}

\begin{frame}{Boost.Geometry几何算法}
    \centering
    \includegraphics[width=0.9\textwidth]{images/boost/boost_geometry.pdf}
\end{frame}
\begin{frame}{Boost安装与使用}
    \begin{columns}
        \begin{column}{0.48\textwidth}
            \begin{ytublock}{安装方式}
                \begin{itemize}
                    \item Linux: \texttt{sudo apt install libboost-all-dev}
                    \item macOS: \texttt{brew install boost}
                    \item Windows: vcpkg或源码编译
                    \item 验证版本: \texttt{dpkg -l | grep libboost}
                \end{itemize}
            \end{ytublock}

            \begin{ytublock}{编译链接}
                \begin{itemize}
                    \item 头文件库:无需链接
                    \item 编译时添加: \texttt{-I/usr/include/boost}
                    \item 需要链接的库: \texttt{-lboost\_system -lboost\_filesystem}
                \end{itemize}
            \end{ytublock}
        \end{column}
        \begin{column}{0.48\textwidth}
            \begin{ytublock}{CMake集成}
                \begin{itemize}
                    \item \texttt{find\_package(Boost REQUIRED)}
                    \item \texttt{target\_link\_libraries(target Boost::boost)}
                    \item 指定组件: \texttt{find\_package(Boost REQUIRED COMPONENTS system filesystem)}
                \end{itemize}
            \end{ytublock}
        \end{column}
    \end{columns}
\end{frame}

% ========== Eigen线性代数库 ==========

\section{Eigen线性代数库}
\begin{frame}{目录}
    \begin{multicols}{2}
        \tableofcontents[currentsection,hideothersubsections]
    \end{multicols}
\end{frame}

\begin{frame}{Eigen简介}
    \centering
    \includegraphics[width=0.75\textwidth]{images/eigen/eigen_overview.pdf}
\end{frame}

\begin{frame}{Eigen核心模块}
    \centering
    \includegraphics[width=0.95\textwidth]{images/eigen/eigen_core.pdf}
\end{frame}

\begin{frame}{Eigen基本类型}
    \begin{columns}
        \begin{column}{0.48\textwidth}
            \begin{ytublock}{动态大小}
                \begin{itemize}
                    \item \texttt{MatrixXd} - 动态矩阵(double)
                    \item \texttt{VectorXd} - 动态向量(double)
                    \item \texttt{ArrayXd} - 动态数组(元素级运算)
                    \item 运行时确定尺寸
                \end{itemize}
            \end{ytublock}

            \begin{ytublock}{固定大小}
                \begin{itemize}
                    \item \texttt{Matrix3d} - 3x3矩阵(double)
                    \item \texttt{Vector3d} - 3维向量(double)
                    \item \texttt{Matrix4f} - 4x4矩阵(float)
                    \item 编译时确定尺寸,性能更好
                \end{itemize}
            \end{ytublock}
        \end{column}
        \begin{column}{0.48\textwidth}
            \begin{ytublock}{命名规则}
                \begin{itemize}
                    \item \texttt{Matrix} - 矩阵类型
                    \item \texttt{Vector} - 向量类型
                    \item \texttt{Array} - 数组类型(元素级)
                    \item \texttt{X} - 动态大小
                    \item \texttt{d} - double, \texttt{f} - float
                \end{itemize}
            \end{ytublock}

            \begin{ytublock}{类型转换}
                \begin{itemize}
                    \item \texttt{Matrix::array()} - 转为Array
                    \item \texttt{Array::matrix()} - 转为Matrix
                    \item 零开销转换
                \end{itemize}
            \end{ytublock}
        \end{column}
    \end{columns}
\end{frame}

\begin{frame}{Eigen稠密矩阵模块}
    \centering
    \includegraphics[width=0.95\textwidth]{images/eigen/eigen_dense.pdf}
\end{frame}

\begin{frame}{Eigen基本运算}
    \begin{columns}
        \begin{column}{0.48\textwidth}
            \begin{ytublock}{矩阵运算}
                \begin{itemize}
                    \item 加减:\texttt{A + B}, \texttt{A - B}
                    \item 乘法:\texttt{A * B}(矩阵乘法)
                    \item 标量:\texttt{A * scalar}, \texttt{A / scalar}
                    \item 转置:\texttt{A.transpose()}
                    \item 共轭转置:\texttt{A.adjoint()}
                \end{itemize}
            \end{ytublock}

            \begin{ytublock}{元素访问}
                \begin{itemize}
                    \item \texttt{A(i, j)} - 矩阵元素
                    \item \texttt{v[i]} - 向量元素
                    \item \texttt{A.block(i,j,m,n)} - 子矩阵
                    \item \texttt{A.row(i)}, \texttt{A.col(j)} - 行/列
                \end{itemize}
            \end{ytublock}
        \end{column}
        \begin{column}{0.48\textwidth}
            \begin{ytublock}{初始化方法}
                \begin{itemize}
                    \item \texttt{MatrixXd::Zero(m,n)} - 零矩阵
                    \item \texttt{MatrixXd::Identity(m,n)} - 单位矩阵
                    \item \texttt{MatrixXd::Random(m,n)} - 随机矩阵
                    \item 逗号初始化:\texttt{A << 1,2,3,4,5,6;}
                \end{itemize}
            \end{ytublock}

            \begin{ytublock}{统计与范数}
                \begin{itemize}
                    \item \texttt{A.sum()}, \texttt{A.mean()} - 和、均值
                    \item \texttt{A.norm()} - Frobenius范数
                    \item \texttt{A.minCoeff()}, \texttt{A.maxCoeff()} - 最值
                    \item \texttt{A.determinant()} - 行列式
                \end{itemize}
            \end{ytublock}
        \end{column}
    \end{columns}
\end{frame}

\begin{frame}{Eigen矩阵分解}
    \centering
    \includegraphics[width=0.95\textwidth]{images/eigen/eigen_solvers.pdf}
\end{frame}

\begin{frame}{Eigen矩阵分解方法}
    \begin{columns}
        \begin{column}{0.48\textwidth}
            \begin{ytublock}{LU分解}
                \begin{itemize}
                    \item \texttt{PartialPivLU<MatrixXd> lu(A);}
                    \item \texttt{x = lu.solve(b);}
                    \item 适用于一般方阵
                    \item 复杂度:O(n³)
                \end{itemize}
            \end{ytublock}

            \begin{ytublock}{QR分解}
                \begin{itemize}
                    \item \texttt{HouseholderQR<MatrixXd> qr(A);}
                    \item \texttt{x = qr.solve(b);}
                    \item 适用于超定系统
                    \item 最小二乘问题
                \end{itemize}
            \end{ytublock}
        \end{column}
        \begin{column}{0.48\textwidth}
            \begin{ytublock}{Cholesky分解}
                \begin{itemize}
                    \item \texttt{LLT<MatrixXd> chol(A);}
                    \item \texttt{x = chol.solve(b);}
                    \item 适用于对称正定矩阵
                    \item 复杂度:O(n³/3),更快
                \end{itemize}
            \end{ytublock}

            \begin{ytublock}{SVD分解}
                \begin{itemize}
                    \item \texttt{JacobiSVD<MatrixXd> svd(A);}
                    \item 最稳定的分解方法
                    \item 适用于任意矩阵
                    \item 可用于伪逆、降维
                \end{itemize}
            \end{ytublock}
        \end{column}
    \end{columns}
\end{frame}

\begin{frame}{Eigen特征值与特征向量模块}
    \centering
    \includegraphics[width=0.95\textwidth]{images/eigen/eigen_eigenvalues.pdf}
\end{frame}

\begin{frame}{Eigen使用示例}
    \begin{ytublock}{线性方程组求解}
        \begin{itemize}
            \item \texttt{\#include <Eigen/Dense>}
            \item \texttt{using namespace Eigen;}
            \item \texttt{MatrixXd A(3, 3);}
            \item \texttt{A << 1, 2, 3, 4, 5, 6, 7, 8, 10;}
            \item \texttt{VectorXd b(3);}
            \item \texttt{b << 3, 3, 4;}
            \item \texttt{VectorXd x = A.colPivHouseholderQr().solve(b);}
        \end{itemize}
    \end{ytublock}

    \vspace{0.2cm}
    \begin{ytublock}{特征值计算}
        \begin{itemize}
            \item \texttt{SelfAdjointEigenSolver<MatrixXd> eigensolver(A);}
            \item \texttt{VectorXd eigenvalues = eigensolver.eigenvalues();}
            \item \texttt{MatrixXd eigenvectors = eigensolver.eigenvectors();}
        \end{itemize}
    \end{ytublock}
\end{frame}

\begin{frame}{Eigen稀疏矩阵模块}
    \centering
    \includegraphics[width=0.95\textwidth]{images/eigen/eigen_sparse.pdf}
\end{frame}

\begin{frame}{Eigen几何变换模块}
    \centering
    \includegraphics[width=0.95\textwidth]{images/eigen/eigen_geometry.pdf}
\end{frame}

\begin{frame}{Eigen安装与配置}
    \begin{columns}
        \begin{column}{0.48\textwidth}
            \begin{ytublock}{安装方法}
                \begin{itemize}
                    \item 下载源码解压即可
                    \item 纯头文件库,无需编译
                    \item 包管理器安装更方便
                      \begin{itemize}
                        \item Linux: \texttt{apt-get install libeigen3-dev}
                        \item macOS: \texttt{brew install eigen}
                      \end{itemize}
                \end{itemize}
            \end{ytublock}
        \end{column}
        \begin{column}{0.48\textwidth}
            \begin{ytublock}{CMake配置}
                \begin{itemize}
                    \item \texttt{find\_path(EIGEN3\_INCLUDE\_DIR Eigen3)}
                    \item \texttt{include\_directories(\$\{EIGEN3\_INCLUDE\_DIR\})}
                    \item 或使用 \texttt{find\_package(Eigen3)}
                \end{itemize}
            \end{ytublock}

            \begin{ytublock}{头文件包含}
                \begin{itemize}
                    \item \texttt{\#include <Eigen/Dense>} - 所有稠密矩阵
                    \item \texttt{\#include <Eigen/Sparse>} - 稀疏矩阵
                    \item \texttt{\#include <Eigen/Geometry>} - 几何变换
                \end{itemize}
            \end{ytublock}
        \end{column}
    \end{columns}
\end{frame}

% ========== Intel OneMKL库 ==========

\section{Intel OneMKL库}
\begin{frame}{目录}
    \begin{multicols}{2}
        \tableofcontents[currentsection,hideothersubsections]
    \end{multicols}
\end{frame}

\begin{frame}{OneMKL简介}
    \centering
    \includegraphics[width=0.95\textwidth]{images/mkl/mkl_overview.pdf}

    \begin{ytublock}{OneMKL 概览}
        \textbf{Intel oneAPI Math Kernel Library}(OneMKL)是 Intel 的高性能数学核心库,专为 x86/x86\_64 平台深度优化。\textbf{模块}:BLAS、LAPACK、ScaLAPACK、FFT、DFT、FFTW3 接口、Pardiso 稀疏求解器、VML(向量数学库)、VSL(向量统计库)等。\textbf{并行}:自动利用多核 CPU、SIMD(AVX/AVX-512)及多线程,支持 OpenMP 与 TBB。\textbf{跨平台}:Windows、Linux、macOS;C/C++、Fortran、Python、DPC++。CMake 一行 \texttt{find\_package(MKL)} 即可启用。
    \end{ytublock}
\end{frame}

\begin{frame}{BLAS基础线性代数子程序}
    \centering
    \includegraphics[width=0.99\textwidth]{images/mkl/blas/mkl_blas_overview.pdf}
\end{frame}

\begin{frame}{BLAS详细结构}
    \centering
    \includegraphics[width=0.99\textwidth]{images/mkl/blas/mkl_blas.pdf}
\end{frame}

\begin{frame}{BLAS Level 1:向量运算}
    \centering
    \includegraphics[width=0.99\textwidth]{images/mkl/blas/mkl_blas_level1.pdf}
\end{frame}

\begin{frame}{BLAS Level 2:矩阵-向量运算}
    \centering
    \includegraphics[width=0.99\textwidth]{images/mkl/blas/mkl_blas_level2.pdf}
\end{frame}

\begin{frame}{BLAS Level 3:矩阵-矩阵运算}
    \centering
    \includegraphics[width=0.99\textwidth]{images/mkl/blas/mkl_blas_level3.pdf}
\end{frame}

\begin{frame}{BLAS函数列表}
    \begin{columns}
        \begin{column}{0.32\textwidth}
            \begin{ytublock}{Level 1:向量运算}
                \begin{itemize}
                    \item \texttt{ddot}、\texttt{sdot} - 向量点积
                    \item \texttt{dnrm2}、\texttt{dasum} - 向量范数
                    \item \texttt{dscal} - 向量缩放
                    \item \texttt{daxpy} - 向量线性组合
                    \item \texttt{dcopy} - 向量复制
                    \item \texttt{idamax} - 最大元素索引
                \end{itemize}
            \end{ytublock}
        \end{column}
        \begin{column}{0.32\textwidth}
            \begin{ytublock}{Level 2:矩阵-向量}
                \begin{itemize}
                    \item \texttt{dgemv} - 一般矩阵向量乘法
                    \item \texttt{dsymv} - 对称矩阵向量乘法
                    \item \texttt{dtrmv} - 三角矩阵向量乘法
                    \item \texttt{dger} - 外积更新
                \end{itemize}
            \end{ytublock}
        \end{column}
        \begin{column}{0.32\textwidth}
            \begin{ytublock}{Level 3:矩阵-矩阵}
                \begin{itemize}
                    \item \texttt{dgemm} - 一般矩阵乘法
                    \item \texttt{dsymm} - 对称矩阵乘法
                    \item \texttt{dtrmm} - 三角矩阵乘法
                    \item \texttt{dsyrk} - 对称秩k更新
                    \item \texttt{dtrsm} - 三角矩阵求解
                \end{itemize}
            \end{ytublock}
        \end{column}
    \end{columns}
\end{frame}

\begin{frame}{LAPACK线性代数包}
    \centering
    \includegraphics[width=0.99\textwidth]{images/mkl/lapack/mkl_lapack.pdf}
\end{frame}

\begin{frame}{LAPACK线性方程组求解}
    \centering
    \includegraphics[width=0.99\textwidth]{images/mkl/lapack/linear_equations/mkl_lapack_linear_equations.pdf}
\end{frame}

\begin{frame}{LAPACK矩阵分解方法}
    \centering
    \includegraphics[width=0.99\textwidth]{images/mkl/lapack/linear_equations/mkl_lapack_linear_eq_matrix_factorization.pdf}
\end{frame}

\begin{frame}{LAPACK特征值问题}
    \centering
    \includegraphics[width=0.99\textwidth]{images/mkl/lapack/eigenvalue/mkl_lapack_eigenvalue_drivers.pdf}
\end{frame}

\begin{frame}{LAPACK主要功能}
    \begin{columns}
        \begin{column}{0.48\textwidth}
            \begin{ytublock}{矩阵分解}
                \begin{itemize}
                    \item \texttt{dgesv} - LU分解求解
                    \item \texttt{dpotrf} - Cholesky分解
                    \item \texttt{dgeqrf} - QR分解
                    \item \texttt{dgesvd} - SVD分解
                \end{itemize}
            \end{ytublock}

            \begin{ytublock}{线性方程组}
                \begin{itemize}
                    \item \texttt{dgesv} - 一般矩阵求解
                    \item \texttt{dposv} - 对称正定求解
                    \item \texttt{dgels} - 最小二乘问题
                \end{itemize}
            \end{ytublock}
        \end{column}
        \begin{column}{0.48\textwidth}
            \begin{ytublock}{特征值问题}
                \begin{itemize}
                    \item \texttt{dsyev} - 对称矩阵特征值
                    \item \texttt{dgeev} - 一般矩阵特征值
                    \item \texttt{dsygv} - 广义特征值
                \end{itemize}
            \end{ytublock}

            \begin{ytublock}{使用场景}
                \begin{itemize}
                    \item 大规模矩阵运算(>1000x1000)
                    \item 高性能计算需求
                    \item 科学计算、工程应用
                    \item 需要极致性能的场景
                \end{itemize}
            \end{ytublock}
        \end{column}
    \end{columns}
\end{frame}

\begin{frame}{MKL稀疏矩阵模块}
    \centering
    \includegraphics[width=0.99\textwidth]{images/mkl/sparse/mkl_sparse_blas_overview.pdf}
\end{frame}

\begin{frame}{稀疏BLAS详细结构}
    \centering
    \includegraphics[width=0.99\textwidth]{images/mkl/sparse/mkl_sparse_blas.pdf}
\end{frame}

\begin{frame}{稀疏矩阵直接求解器}
    \centering
    \includegraphics[width=0.99\textwidth]{images/mkl/sparse/mkl_sparse_directsolvers.pdf}
\end{frame}

\begin{frame}{稀疏矩阵迭代求解器}
    \centering
    \includegraphics[width=0.99\textwidth]{images/mkl/sparse/mkl_sparse_itersolvers.pdf}
\end{frame}

\begin{frame}{MKL FFT快速傅里叶变换}
    \centering
    \includegraphics[width=0.99\textwidth]{images/mkl/FFT/mkl_dft.pdf}
\end{frame}

\begin{frame}{MKL VML向量数学库}
    \centering
    \includegraphics[width=0.99\textwidth]{images/mkl/VML/mkl_vml.pdf}
\end{frame}

\begin{frame}{MKL VSL随机数生成库}
    \centering
    \includegraphics[width=0.99\textwidth]{images/mkl/VSL/mkl_vsl.pdf}
\end{frame}

\begin{frame}{OneMKL安装与配置}
    \begin{columns}
        \begin{column}{0.48\textwidth}
            \begin{ytublock}{安装方法}
                \begin{itemize}
                    \item \textbf{官方安装包}:
                      \begin{itemize}
                        \item 从Intel官网下载
                        \item 运行安装程序
                        \item 设置环境变量 \texttt{MKLROOT}
                      \end{itemize}
                    \item \textbf{包管理器}:
                      \begin{itemize}
                        \item Linux: \texttt{apt-get install libmkl-dev}
                        \item macOS: \texttt{brew install mkl}
                      \end{itemize}
                \end{itemize}
            \end{ytublock}

            \begin{ytublock}{CMake配置}
                \begin{itemize}
                    \item \texttt{find\_package(MKL)}
                    \item \texttt{target\_link\_libraries} 链接库
                    \item 链接库示例:
                      \begin{itemize}
                        \item \texttt{mkl\_intel\_lp64}
                        \item \texttt{mkl\_core}
                        \item \texttt{mkl\_sequential}
                      \end{itemize}
                \end{itemize}
            \end{ytublock}
        \end{column}
        \begin{column}{0.48\textwidth}
            \begin{ytublock}{头文件包含}
                \begin{itemize}
                    \item \texttt{\#include <mkl.h>} - 所有MKL函数
                    \item \texttt{\#include <mkl\_blas.h>} - BLAS函数
                    \item \texttt{\#include <mkl\_lapacke.h>} - LAPACK接口
                    \item \texttt{\#include <mkl\_dfti.h>} - FFT接口
                \end{itemize}
            \end{ytublock}
        \end{column}
    \end{columns}
\end{frame}

% ========== 总结 ==========

\section{总结}

\begin{frame}{总结}
    \begin{columns}
        \begin{column}{0.48\textwidth}
            \begin{ytublock}{本章要点}
                \begin{itemize}
                    \item 掌握C++标准库cmath基础
                    \item 了解Boost科学计算组件
                    \item 学会Eigen线性代数库的使用
                    \item 了解Intel OneMKL高性能计算
                \end{itemize}
            \end{ytublock}
        \end{column}
        \begin{column}{0.48\textwidth}
            \begin{ytublock}{学习路径}
                \begin{itemize}
                    \item 从标准库cmath开始学习基础
                    \item 进阶到Boost扩展数学功能
                    \item 使用Eigen进行矩阵运算
                    \item 需要高性能时考虑MKL
                \end{itemize}
            \end{ytublock}
        \end{column}
    \end{columns}
\end{frame}

\begin{frame}{进一步学习}
    \begin{ytublock}{推荐资源}
        \begin{itemize}
            \item C++标准库文档:\texttt{https://en.cppreference.com}
            \item Boost官方文档:\texttt{https://www.boost.org}
            \item Eigen官方文档:\texttt{https://eigen.tuxfamily.org}
            \item Intel OneMKL文档:\texttt{https://software.intel.com/mkl}
        \end{itemize}
    \end{ytublock}

    \begin{ytublock}{实践项目建议}
        \begin{itemize}
            \item 实现一个简单的线性方程组求解器
            \item 开发一个矩阵运算性能测试工具
            \item 构建一个科学计算工作流示例
        \end{itemize}
    \end{ytublock}
\end{frame}

\end{document}
