\documentclass[UTF8,aspectratio=169]{beamer}



% 基本包
\usepackage[UTF8]{ctex}
\usepackage{graphicx}
\usepackage{amsmath}
\usepackage{amsfonts}
\usepackage{amssymb}
% \usepackage{listings}  % 已替换为minted
\usepackage{xcolor}
\usepackage{hyperref}
\usepackage{booktabs}
\usepackage{multirow}
\usepackage{multicol}
\usepackage{float}
\usepackage{tikz}
\usetikzlibrary{positioning,shapes,arrows,fit,backgrounds}
\usepackage{pgfplots}
\pgfplotsset{compat=1.18}
\usepackage{minted}
\usepackage{fontspec}
\usepackage[most]{tcolorbox}

% Beamer主题设置
\usetheme{Madrid}
\usecolortheme{whale}

% 校徽设置
\logo{
  \IfFileExists{../长江大学校徽.pdf}{
    \begin{tikzpicture}[remember picture,overlay]
      \node[anchor=north east, xshift=-0.2mm, yshift=-0.2mm] at (current page.north east) {
        \includegraphics[height=1.0cm]{../长江大学校徽.pdf}
      };
    \end{tikzpicture}
  }{
    \begin{tikzpicture}[remember picture,overlay]
      \node[anchor=north east, xshift=-3mm, yshift=-3mm] at (current page.north east) {
        \textcolor{red}{\tiny [校徽文件未找到]}
      };
    \end{tikzpicture}
  }
}

% ===== 使用推荐的 font themes =====
\usefonttheme{professionalfonts}  % 允许自定义字体

% 自定义frame标题栏,也缩短长度留出logo空间
% 设置标题栏背景颜色为淡蓝色
\definecolor{frametitlebg}{RGB}{200,215,250} % 淡蓝色,可根据需要调整
\setbeamercolor{frametitle}{bg=frametitlebg, fg=ytublue!80!black}
\setbeamertemplate{frametitle}{
  \ifbeamercolorempty[bg]{frametitle}{}{\nointerlineskip}%
  \begin{tcolorbox}[
    enhanced,
    width=0.90\paperwidth,
    height=2.5ex,
    colback=frametitlebg,
    colframe=frametitlebg,
    boxrule=0pt,
    left=0pt,
    right=0pt,
    top=1pt,
    bottom=0pt,
    boxsep=0pt,
    before skip=0pt,
    after skip=0.1em,  % 减少标题栏和内容之间的间距
  ]
  \vspace{0.2ex} % 减少标题文字上方的空白
  \usebeamerfont{frametitle}\textcolor{ytublue!80!black}{\hspace{1em}\insertframetitle}
  \end{tcolorbox}
}

% ===== 设置现代字体 =====
\setsansfont{Source Sans Pro}     % 正文字体
\setmonofont{Source Code Pro}[Scale=0.9]  % 代码字体,稍微缩小一点

\setbeamertemplate{navigation symbols}{}

% 减少页面间距
\setbeamertemplate{itemize items}[circle]
\setbeamertemplate{enumerate items}[default]
\setlength{\itemsep}{0.1em}
\setlength{\parskip}{0.1em}

% 页码设置
\setbeamertemplate{footline}[frame number]

% 定义流程图样式
\tikzset{
    block/.style = {rectangle, draw, fill=blue!10,
        minimum width=6em, align=center, rounded corners, minimum height=3em},
    line/.style = {draw, -latex'}
}
% 水印设置
\setbeamertemplate{background}{
    \begin{tikzpicture}[remember picture,overlay]
        \node[rotate=-45,scale=0.8,opacity=0.1,color=gray]
             at ([xshift=0.5cm,yshift=0.5cm]current page.south west)
             {\large\textbf{WPJ}};
    \end{tikzpicture}
}

% 自定义颜色
\definecolor{qtgreen}{RGB}{41,128,185}
\definecolor{qtblue}{RGB}{52,73,94}

% 定义长江大学蓝主色调
\definecolor{ytublue}{RGB}{0,84,159}
% 统一block样式
\newtcolorbox{ytublock}[1]{
  colback=white,
  colframe=ytublue!80!black,
  colbacktitle=ytublue!20!white,
  coltitle=ytublue!80!black,
  title={#1},
  fonttitle=\bfseries,
  arc=3mm,
  boxrule=1pt,
  boxsep=1mm,
  left=2mm,
  right=2mm,
  top=0.5mm,
  bottom=0.5mm,
  before skip=3pt,
  after skip=3pt,
  enhanced,
  drop fuzzy shadow=ytublue!20!black
}

% 定义警告块样式
\newtcolorbox{ytualertblock}[1]{
  colback=white,
  colframe=red!80!black,
  colbacktitle=red!20!white,
  coltitle=red!80!black,
  title={#1},
  fonttitle=\bfseries,
  arc=3mm,
  boxrule=1.5pt,
  boxsep=1mm,
  left=2mm,
  right=2mm,
  top=0.5mm,
  bottom=0.5mm,
  before skip=3pt,
  after skip=3pt,
  enhanced,
  drop fuzzy shadow=red!20!black,
  overlay={
    \begin{tcbclipinterior}
      \fill[red!10!white] (interior.south west) rectangle (interior.north east);
    \end{tcbclipinterior}
  }
}

% cpp代码高亮设置
\setminted[cpp]{
    fontsize=\tiny,
    fontfamily=tt,             % 使用等宽字体
    linenos=true,
    frame=lines,               % 上下两条线,简洁清爽(比 tb 更现代)
    framesep=3mm,              % 内边距
    rulecolor=\color{blue!20}, % 线条颜色浅蓝,不刺眼
    bgcolor=blue!10,           % 浅蓝色背景
    baselinestretch=1.2,       % 行距稍大,更易读
    breaklines=true,
    breakautoindent=true,
    tabsize=4,
    xleftmargin=5mm,
    xrightmargin=5mm,
    numbersep=8pt,             % 行号与代码间距
    % ===== 其他美化 =====
    obeytabs=true,             % 尊重 tab 字符
    samepage=false,            % 允许跨页(重要!避免空白)
    escapeinside=||,           % 可在代码中使用 |LaTeX| 插入 LaTeX 命令
}

% 设置标题页颜色,与frame标题保持一致
\setbeamercolor{title}{bg=frametitlebg, fg=ytublue!80!black}
\setbeamercolor{subtitle}{bg=frametitlebg, fg=ytublue!70!black}
\setbeamercolor{author}{bg=frametitlebg, fg=ytublue!80!black}
\setbeamercolor{institute}{bg=frametitlebg, fg=ytublue!80!black}
\setbeamercolor{date}{bg=frametitlebg, fg=ytublue!80!black}

% 自定义标题页样式,全部内容同一个tcolorbox,居中排版,字体和间距区分
\setbeamertemplate{title page}{
  \vbox{}
  \begingroup
    \centering
    \begin{tcolorbox}[
      enhanced,
      width=0.92\paperwidth,
      colback=frametitlebg,
      colframe=frametitlebg,
      boxrule=0pt,
      left=0pt,
      right=0pt,
      top=4mm,
      bottom=4mm,
      boxsep=0pt,
      before skip=0pt,
      after skip=1.2em,
    ]
    % 标题
    {\centering
      {\fontsize{24pt}{27pt}\selectfont\textcolor{ytublue!80!black}{\inserttitle}\par}
      \vspace{1.2em}
      % 副标题
      {\fontsize{21pt}{24pt}\selectfont\textcolor{ytublue!70!black}{\insertsubtitle}\par}
      \vspace{2.0em}
      % 作者
      {\fontsize{12pt}{15pt}\selectfont\insertauthor\par}
      \vspace{0.7em}
      % 单位
      {\fontsize{12pt}{15pt}\selectfont\insertinstitute\par}
      \vspace{0.7em}
      % 日期
      {\fontsize{12pt}{15pt}\selectfont\insertdate\par}
    }
    \vspace{0.5em}
    \vfill
    \end{tcolorbox}
  \endgroup
}


% 文档信息
\title{综合开发实例}
\subtitle{大地电磁(MT)一维反演示例程序}
\author{王培杰}
\institute{长江大学地球物理与石油资源学院}
\date{\today}

\begin{document}

\begin{frame}{课后任务}
    \begin{ytublock}{大地电磁一维反演系统}
        \begin{itemize}
            \item 设计一个完整的大地电磁(MT)一维反演系统,支持以下功能:
            \item 使用模块化设计实现正演计算、Jacobian计算、正则化、优化求解等功能
            \item 使用Intel MKL库进行高性能科学计算(BLAS、LAPACK)
            \item 使用Qt框架构建图形用户界面(GUI)
            \item 使用面向对象设计实现各功能模块的解耦和可扩展性
            \item 使用智能指针管理动态分配的内存资源
            \item 使用多线程技术实现后台计算,避免阻塞UI
            \item 包含完整的参数验证和异常处理机制
        \end{itemize}
    \end{ytublock}
\end{frame}

\begin{frame}{大地电磁反演问题简介}
    \begin{ytublock}{问题描述}
        \begin{itemize}
            \item 大地电磁(MT)反演旨在根据地表观测的电磁响应数据,推断地下介质的电性结构(电阻率分布)。
            \item 该问题广泛应用于地球物理勘探、矿产资源勘查、地热资源开发等领域,是地球物理反演的基础问题之一。
            \item 一维反演假设地下介质在水平方向上均匀,仅在垂直方向上变化,适用于层状地质结构。
        \end{itemize}
    \end{ytublock}
    \begin{ytublock}{核心挑战}
        \begin{itemize}
            \item \textbf{正演问题}:给定地下电性结构,计算地表MT响应(视电阻率、相位)
            \item \textbf{反演问题}:根据观测的MT响应,反推地下电性结构
            \item \textbf{不适定性}:反演问题通常存在多解性,需要正则化约束
            \item \textbf{计算效率}:需要高效的正演计算和优化算法
        \end{itemize}
    \end{ytublock}
\end{frame}

\begin{frame}{MT正演问题}
    \begin{ytublock}{物理模型}
        \begin{itemize}
            \item 一维层状介质模型:地下分为$M$层,每层具有不同的电阻率$\rho_i$(或电导率$\sigma_i = 1/\rho_i$)和不同的层厚度
            \item 极化模式:TE(横电)极化,电场垂直于传播方向
        \end{itemize}
    \end{ytublock}
    \begin{ytublock}{正演方法}
        \begin{itemize}
            \item \textbf{解析法}:使用向上递推阻抗的方法,从底层开始逐层向上计算地表阻抗
            \item \textbf{有限差分法/有限元法}:将Helmholtz方程离散化,求解三对角线性方程组
            \item \textbf{输出数据}:视电阻率$\rho_a$和相位$\phi$,通常以$\log_{10}(\rho_a)$和$\phi$的形式存储
        \end{itemize}
    \end{ytublock}
\end{frame}

\begin{frame}{MT一维正演的解析法}
    \begin{ytublock}{基本思想}
        \begin{itemize}
            \item 使用向上递推阻抗的方法,从最底层(半空间)开始,逐层向上计算到地表的阻抗
            \item 基于传输线理论,利用层间阻抗的连续性条件建立递推关系
            \item 适用于TE(横电)极化模式,电场垂直于传播方向
        \end{itemize}
    \end{ytublock}
    \begin{ytublock}{关键物理量}
        \begin{itemize}
            \item \textbf{特征阻抗}:$Z_{0i} = \sqrt{\frac{i\omega\mu_0}{\sigma_i}} = (1+i)\sqrt{\frac{\omega\mu_0}{2\sigma_i}}$
            \item \textbf{波数}:$k_i = \sqrt{i\omega\mu_0\sigma_i} = (1+i)\sqrt{\frac{\omega\mu_0\sigma_i}{2}}$
            \item 其中$\mu_0 = 4\pi \times 10^{-7}$ H/m为真空磁导率,$\sigma_i = 1/\rho_i$为第$i$层的电导率
        \end{itemize}
    \end{ytublock}
\end{frame}

\begin{frame}{MT一维正演的解析法:递推公式}
    \begin{ytublock}{最底层(半空间)阻抗}
        最底层(第$M-1$层)作为半空间,其阻抗为:
        \[
            Z_{M-1} = Z_{0,M-1} = \sqrt{\frac{i\omega\mu_0}{\sigma_{M-1}}} = (1+i)\sqrt{\frac{\omega\mu_0}{2\sigma_{M-1}}}
        \]
    \end{ytublock}
    \begin{ytublock}{向上递推公式}
        从第$i+1$层向上递推到第$i$层的阻抗递推公式:
        \[
            Z_i = Z_{0i} \cdot \frac{Z_{i+1} + Z_{0i} \tanh(k_i d_i)}{Z_{0i} + Z_{i+1} \tanh(k_i d_i)}
        \]
        其中$d_i$为第$i$层的厚度,$\tanh(k_i d_i)$为复双曲正切函数。
        \begin{itemize}
            \item 递推过程:从$i = M-2$开始,逐层向上计算到$i = 0$(地表层)
        \end{itemize}
    \end{ytublock}
\end{frame}

\begin{frame}{MT一维正演的解析法:复双曲正切计算}
    \begin{ytublock}{复双曲正切函数}
        对于复数$z = x + iy$,双曲正切函数定义为:
        \[
            \tanh(z) = \frac{\sinh(z)}{\cosh(z)} = \frac{e^z - e^{-z}}{e^z + e^{-z}}
        \]
        \begin{itemize}
            \item 对于$z = k_i d_i$,其中$k_i = (1+i)\alpha_i$,$\alpha_i = \sqrt{\frac{\omega\mu_0\sigma_i}{2}}$:
            \[
                e^{k_i d_i} = e^{\alpha_i d_i}(\cos(\alpha_i d_i) + i\sin(\alpha_i d_i))
            \]
            \item 因此:
            \[
                \sinh(k_i d_i) = \frac{1}{2}(e^{k_i d_i} - e^{-k_i d_i}), \quad
                \cosh(k_i d_i) = \frac{1}{2}(e^{k_i d_i} + e^{-k_i d_i})
            \]
        \end{itemize}
    \end{ytublock}
\end{frame}

\begin{frame}{MT一维正演的解析法:响应计算}
    \begin{ytublock}{视电阻率和相位}
        计算得到地表阻抗$Z_0$后,可计算MT响应:
        \begin{itemize}
            \item \textbf{视电阻率}:
            \[
                \rho_a = \frac{|Z_0|^2}{\omega\mu_0} = \frac{Z_0 \cdot Z_0^*}{\omega\mu_0}
            \]
            其中$Z_0^*$为$Z_0$的共轭复数,$|Z_0|^2 = (\text{Re}(Z_0))^2 + (\text{Im}(Z_0))^2$
            \item \textbf{相位}:
            \[
                \phi = \arctan\left(\frac{\text{Im}(Z_0)}{\text{Re}(Z_0)}\right) \quad \text{(弧度)}
            \]
            或
            \[
                \phi = \frac{180}{\pi} \arctan\left(\frac{\text{Im}(Z_0)}{\text{Re}(Z_0)}\right) \quad \text{(度)}
            \]
        \end{itemize}
    \end{ytublock}
    \begin{ytublock}{输出格式}
        通常以$\log_{10}(\rho_a)$和相位$\phi$(度)的形式存储,便于后续处理和可视化
    \end{ytublock}
\end{frame}

\begin{frame}{MT反演问题的数学表述}
    \begin{ytublock}{反演问题的定义}
        给定观测数据$d_{obs} \in \mathbb{R}^{n}$($n$为数据点数,通常$n = 2n_{freq}$,包含视电阻率和相位),
        寻找模型参数$m \in \mathbb{R}^{M}$($M$为模型层数),使得合成数据$d_{syn}(m)$与观测数据$d_{obs}$的差异最小。
        \begin{itemize}
            \item 正演算子:$d_{syn} = F(m)$,其中$F: \mathbb{R}^{M} \rightarrow \mathbb{R}^{n}$为非线性正演算子。
            \item 反演问题通常是不适定的(ill-posed),即解不唯一或不稳定,需要正则化约束。
        \end{itemize}
    \end{ytublock}
    \begin{ytublock}{数据拟合项}
        数据拟合项衡量合成数据与观测数据的差异:
        \[
            \Phi_d(m) = \|d_{obs} - d_{syn}(m)\|^2 = \sum_{i=1}^{n} [d_{obs,i} - d_{syn,i}(m)]^2
        \]
        \begin{itemize}
            \item 使用L2范数(最小二乘)度量数据拟合程度。
        \end{itemize}
    \end{ytublock}
\end{frame}

\begin{frame}{正则化与目标函数}
    \begin{ytublock}{正则化的必要性}
        反演问题的不适定性导致:
        \begin{itemize}
            \item 解不唯一:多个模型可能产生相同的观测数据
            \item 解不稳定:观测数据的小扰动可能导致模型的大变化
            \item 过度拟合:模型可能过度复杂,包含虚假结构
        \end{itemize}
        正则化通过引入先验信息(如模型平滑性)来稳定反演过程。
    \end{ytublock}
    \begin{ytublock}{目标函数}
        阻尼最小二乘(Tikhonov正则化)目标函数:
        \[
            \Phi(m) = \Phi_d(m) + \lambda \Phi_m(m) = \|d_{obs} - d_{syn}(m)\|^2 + \lambda \|L m\|^2
        \]
        其中$\Phi_d(m)$为数据拟合项,$\Phi_m(m) = \|L m\|^2$为模型约束项,$\lambda > 0$为平衡数据拟合和模型约束的正则化参数,$L$为正则化矩阵。
    \end{ytublock}
\end{frame}

\begin{frame}{正则化矩阵的类型}
    \begin{ytublock}{平滑度约束(SMOOTHNESS)}
        二阶差分矩阵,惩罚模型参数的二阶导数,使模型平滑:
        \[
            L_{ij} = \begin{cases}
                1, & j = i \\
                -2, & j = i+1 \\
                1, & j = i+2 \\
                0, & \text{其他}
            \end{cases}
        \]
        对应约束:$\sum_{i=1}^{M-2} (m_i - 2m_{i+1} + m_{i+2})^2$,使相邻层之间的变化平滑。
    \end{ytublock}
    \begin{ytublock}{其他正则化类型}
        \begin{itemize}
            \item \textbf{平坦度约束(FLATNESS)}:一阶差分,$L_{ij} = \delta_{j,i+1} - \delta_{ji}$,惩罚模型梯度
            \item \textbf{最小范数约束(MINIMUM\_NORM)}:$L = I$(单位矩阵),最小化模型参数的L2范数
        \end{itemize}
    \end{ytublock}
\end{frame}

\begin{frame}{高斯-牛顿法:线性化}
    \begin{ytublock}{非线性问题的线性化}
        正演算子$F(m)$是非线性的,在模型$m^{(k)}$附近进行一阶泰勒展开:
        \[
            d_{syn}(m) \approx d_{syn}(m^{(k)}) + J^{(k)} (m - m^{(k)})
        \]
        其中$J^{(k)}$为Jacobian矩阵(灵敏度矩阵),元素为:
        \[
            J_{ij}^{(k)} = \frac{\partial d_{syn,i}}{\partial m_j}\Big|_{m=m^{(k)}}
        \]
        \begin{itemize}
            \item Jacobian矩阵的维度为$n \times M$,表示每个数据对每个模型参数的灵敏度。
        \end{itemize}
    \end{ytublock}
    \begin{ytublock}{残差向量}
        定义残差向量:$r^{(k)} = d_{obs} - d_{syn}(m^{(k)})$, 则线性化后的数据拟合项为$\Phi_d(m) \approx \|r^{(k)} - J^{(k)} \delta m\|^2$, 其中$\delta m = m - m^{(k)}$为模型更新量。
    \end{ytublock}
\end{frame}

\begin{frame}{高斯-牛顿法:正规方程推导}
    \begin{ytublock}{目标函数的线性化}
        将目标函数在$m^{(k)}$附近线性化:$\Phi(m) \approx \|r^{(k)} - J^{(k)} \delta m\|^2 + \lambda \|L (m^{(k)} + \delta m)\|^2$
    \end{ytublock}
    \begin{ytublock}{最小化条件}
        对$\delta m$求导并令其为零,得到正规方程:
        \[
            [(J^{(k)})^T J^{(k)} + \lambda L^T L] \delta m = (J^{(k)})^T r^{(k)} - \lambda L^T L m^{(k)}
        \]
        \begin{itemize}
            \item 这是对称正定线性方程组,可使用Cholesky分解高效求解。
        \end{itemize}
    \end{ytublock}
\end{frame}

\begin{frame}{阻尼高斯-牛顿法}
    \begin{ytublock}{简化形式}
        \begin{itemize}
            \item 当初始模型$m^{(0)}$接近零或正则化项较小时,可忽略$\lambda L^T L m^{(k)}$项,得到简化形式:
            \[
                [(J^{(k)})^T J^{(k)} + \lambda L^T L] \delta m = (J^{(k)})^T r^{(k)}
            \]
            \item 这是对称正定线性方程组,可使用Cholesky分解高效求解。
        \end{itemize}
    \end{ytublock}
    \begin{ytublock}{迭代更新}
        模型更新公式:
            $m^{(k+1)} = m^{(k)} + \delta m^{(k)}$
        迭代过程:
        \begin{enumerate}
            \item 计算当前模型的合成数据$d_{syn}(m^{(k)})$
            \item 计算残差$r^{(k)} = d_{obs} - d_{syn}(m^{(k)})$
            \item 计算Jacobian矩阵$J^{(k)}$
            \item 求解正规方程得到$\delta m^{(k)}$
            \item 更新模型:$m^{(k+1)} = m^{(k)} + \delta m^{(k)}$
        \end{enumerate}
    \end{ytublock}
\end{frame}

\begin{frame}{Jacobian矩阵的计算}
    \begin{ytublock}{有限差分法}
        由于正演算子的解析导数难以计算,使用有限差分法近似:
        \begin{itemize}
                \item \textbf{前向差分法}:
            \[
                J_{ij} = \frac{\partial d_{syn,i}}{\partial m_j} \approx \frac{d_{syn,i}(m + \epsilon e_j) - d_{syn,i}(m)}{\epsilon}
            \]
            其中$e_j$为第$j$个标准基向量,$\epsilon$为扰动步长(通常$10^{-5}$)。
            \item \textbf{中心差分法}(更精确但计算量加倍):
            \[
                J_{ij} \approx \frac{d_{syn,i}(m + \epsilon e_j) - d_{syn,i}(m - \epsilon e_j)}{2\epsilon}
            \]
        \end{itemize}
    \end{ytublock}
    \begin{ytublock}{计算复杂度}
        \begin{itemize}
            \item 前向差分法需要$M$次正演计算(每列一次)
            \item 中心差分法需要$2M$次正演计算
            \item Jacobian矩阵大小为$n \times M$,通常$n \gg M$(如$n=122$,$M=40$)
        \end{itemize}
    \end{ytublock}
\end{frame}

\begin{frame}{收敛判据与停止条件}
    \begin{ytublock}{收敛判据}
        \begin{itemize}
            \item \textbf{残差判据}:数据拟合残差的L2范数小于容差
            \[
                \|r^{(k)}\| = \|d_{obs} - d_{syn}(m^{(k)})\| < \text{tol\_residual}
            \]
            \item \textbf{模型更新判据}:模型更新量的L2范数小于容差
            \[
                \|\delta m^{(k)}\| < \text{tol\_dm}
            \]
            通常使用$\text{tol\_dm} = 10^{-4}$,表示模型已收敛。
            \item \textbf{最大迭代次数}:防止无限迭代,通常设置$\text{max\_iter} = 20$。
        \end{itemize}
    \end{ytublock}
    \begin{ytublock}{正则化参数的选择}
        $\lambda$的选择影响反演结果:$\lambda$过大:模型过度平滑,数据拟合差;$\lambda$过小:模型可能振荡,不稳定;通常通过L曲线法或交叉验证选择最优$\lambda$。
    \end{ytublock}
\end{frame}

\begin{frame}{系统架构设计}
    \begin{ytublock}{模块化设计}
        \begin{itemize}
            \item \textbf{数据模型模块(mt\_model.h)}:定义所有数据结构(模型参数、观测数据、反演结果等)
            \item \textbf{频率生成器(mt\_frequency\_generator)}:生成MT反演所需的频率数组
            \item \textbf{正演求解器(mt\_forward\_solver)}:封装1D MT正演计算,使用MKL库进行高性能计算
            \item \textbf{Jacobian计算器(mt\_jacobian\_calculator)}:计算反演所需的灵敏度矩阵
            \item \textbf{正则化模块(mt\_regularization)}:构建正则化矩阵,实现模型平滑约束
            \item \textbf{优化求解器(mt\_optimizer)}:求解反演中的正规方程,使用MKL BLAS/LAPACK
            \item \textbf{反演核心(mt\_inversion\_core)}:协调各模块完成反演任务
            \item \textbf{GUI界面(mt\_inversion\_gui)}:提供图形用户界面,支持参数设置、实时进度显示、结果可视化
        \end{itemize}
    \end{ytublock}
\end{frame}

\begin{frame}{开发步骤}
    \begin{columns}
        \column{0.48\textwidth}
        \begin{block}{设计阶段}
            \begin{enumerate}
                \item 设计系统架构
                \item 确定模块划分
                \item 定义接口规范
            \end{enumerate}
        \end{block}
        \begin{block}{核心模块实现}
            \begin{enumerate}
                \setcounter{enumi}{3}
                \item 数据模型模块
                \item 频率生成器
                \item 正演求解器
                \item Jacobian计算器
            \end{enumerate}
        \end{block}

        \column{0.48\textwidth}
        \begin{block}{优化与界面}
            \begin{enumerate}
                \setcounter{enumi}{7}
                \item 正则化模块
                \item 优化求解器
                \item 反演核心协调器
                \item GUI界面开发
            \end{enumerate}
        \end{block}
        \begin{block}{测试与优化}
            \begin{enumerate}
                \setcounter{enumi}{11}
                \item 单元测试
                \item 集成测试
                \item 性能优化
            \end{enumerate}
        \end{block}
    \end{columns}
\end{frame}

\begin{frame}{数据模型模块(mt\_model.h)}
    \begin{ytublock}{核心数据结构}
        \begin{itemize}
            \item 所有数据结构定义在\texttt{MT}命名空间中
            \item \textbf{InversionParams}:反演参数(层数、频率点数、正则化参数等)
            \item \textbf{InversionResult}:反演结果(真实模型、初始模型、最终模型、残差历史等)
            \item \textbf{ModelParams}:模型参数(电阻率对数、层厚度、层深度)
            \item \textbf{FrequencyParams}:频率参数(周期数组、角频率数组)
            \item \textbf{ObservationData}:观测数据(数据数组、标准差)
        \end{itemize}
    \end{ytublock}
    \begin{ytublock}{设计特点}
        \begin{itemize}
            \item 使用\texttt{std::vector}存储数组数据,支持动态大小
            \item 提供默认参数值,简化使用
            \item 结构清晰,便于扩展和维护
        \end{itemize}
    \end{ytublock}
\end{frame}

\begin{frame}{频率生成器模块(mt\_frequency\_generator)}
    \begin{ytublock}{核心功能}
        \begin{itemize}
            \item 生成MT反演所需的频率数组,通常为对数均匀分布
            \item 支持自定义周期范围(默认0.001秒到1000秒)
            \item 自动计算角频率$\omega = 2\pi/T$($T$为周期)
        \end{itemize}
    \end{ytublock}
    \begin{columns}
        \column{0.48\textwidth}
        \begin{block}{主要方法}
            \begin{itemize}
                \item \textbf{generate()}:生成频率数组
                \item \textbf{generateParams()}:生成频率参数结构
                \item 支持默认参数和自定义参数
            \end{itemize}
        \end{block}

        \column{0.48\textwidth}
        \begin{block}{实现特点}
            \begin{itemize}
                \item 对数均匀分布:$\log_{10}(T)$均匀分布
                \item 频率点数可配置(默认61个)
                \item 返回周期和角频率两个数组
            \end{itemize}
        \end{block}
    \end{columns}
\end{frame}

\begin{frame}{正演求解器模块(mt\_forward\_solver)}
    \begin{ytublock}{核心功能}
        \begin{itemize}
            \item 封装1D MT正演计算,使用向上递推阻抗的解析法
            \item 输入:模型参数(电阻率对数、层厚度)、角频率数组
            \item 输出:MT响应数据(视电阻率对数、相位)
        \end{itemize}
    \end{ytublock}
    \begin{columns}
        \column{0.48\textwidth}
        \begin{block}{算法流程}
            \begin{itemize}
                \item \textbf{计算电导率}:$\sigma = 1/(10^{m_{\log\rho}})$
                \item \textbf{向上递推}:从底层开始,逐层向上计算阻抗
                \item \textbf{计算响应}:根据地表阻抗计算视电阻率和相位
                \item 使用MKL复数运算库
            \end{itemize}
        \end{block}

        \column{0.48\textwidth}
        \begin{block}{MKL函数使用}
            \begin{itemize}
                \item \texttt{MKL\_Complex16}:复数类型
                \item 复数运算:乘法、除法、开方
                \item 高效计算:利用MKL优化
            \end{itemize}
        \end{block}
    \end{columns}
\end{frame}

\begin{frame}{Jacobian计算器模块(mt\_jacobian\_calculator)}
    \begin{ytublock}{核心功能}
        \begin{itemize}
            \item 计算反演所需的Jacobian矩阵(灵敏度矩阵)
            \item Jacobian矩阵元素:$J_{ij} = \frac{\partial d_i}{\partial m_j}$,表示第$i$个数据对第$j$个模型参数的灵敏度
        \end{itemize}
    \end{ytublock}
    \begin{columns}
        \column{0.48\textwidth}
        \begin{block}{计算方法}
            \begin{itemize}
                \item \textbf{前向差分}:$J_{ij} \approx \frac{d_i(m+\epsilon e_j) - d_i(m)}{\epsilon}$
                \item \textbf{中心差分}:$J_{ij} \approx \frac{d_i(m+\epsilon e_j) - d_i(m-\epsilon e_j)}{2\epsilon}$
                \item 扰动步长$\epsilon$可配置(默认$10^{-5}$)
            \end{itemize}
        \end{block}

        \column{0.48\textwidth}
        \begin{block}{实现特点}
            \begin{itemize}
                \item 依赖正演求解器进行计算
                \item 支持两种差分方法
                \item 矩阵大小:$n_{data} \times M$($M$为模型参数个数)
            \end{itemize}
        \end{block}
    \end{columns}
\end{frame}

\begin{frame}{正则化模块(mt\_regularization)}
    \begin{ytublock}{核心功能}
        \begin{itemize}
            \item 构建正则化矩阵$L$,用于反演中的模型平滑约束
            \item 目标:防止反演结果过度振荡,提高稳定性
        \end{itemize}
    \end{ytublock}
    \begin{columns}
        \column{0.48\textwidth}
        \begin{block}{正则化类型}
            \begin{itemize}
                \item \textbf{SMOOTHNESS}:平滑度约束(二阶差分)
                \item \textbf{FLATNESS}:平坦度约束(一阶差分)
                \item \textbf{MINIMUM\_NORM}:最小范数约束(单位矩阵)
            \end{itemize}
        \end{block}

        \column{0.48\textwidth}
        \begin{block}{主要方法}
            \begin{itemize}
                \item \textbf{buildLMatrix()}:构建正则化矩阵$L$
                \item \textbf{computeLTL()}:计算$L^T L$
                \item \textbf{setType()}:设置正则化类型
            \end{itemize}
        \end{block}
    \end{columns}
\end{frame}

\begin{frame}{优化求解器模块(mt\_optimizer)}
    \begin{ytublock}{核心功能}
        \begin{itemize}
            \item 求解反演中的正规方程:$(J^T J + \lambda L^T L) \delta m = J^T r$
            \item 使用MKL BLAS/LAPACK进行高性能矩阵运算
        \end{itemize}
    \end{ytublock}
    \begin{columns}
        \column{0.48\textwidth}
        \begin{block}{MKL函数使用}
            \begin{itemize}
                \item \texttt{cblas\_dsyrk}:计算$J^T J$(对称矩阵)
                \item \texttt{cblas\_dgemv}:计算$J^T r$
                \item \texttt{LAPACKE\_dposv}:求解对称正定线性方程组(Cholesky分解)
            \end{itemize}
        \end{block}

        \column{0.48\textwidth}
        \begin{block}{主要方法}
            \begin{itemize}
                \item \textbf{solve()}:求解正规方程
                \item \textbf{computeJTJ()}:计算$J^T J$
                \item \textbf{computeJTr()}:计算$J^T r$
                \item 支持Cholesky和LU分解
            \end{itemize}
        \end{block}
    \end{columns}
\end{frame}

\begin{frame}{反演核心模块(mt\_inversion\_core)}
    \begin{ytublock}{协调器设计}
        \begin{itemize}
            \item 作为协调器,使用各个模块化组件完成反演任务
            \item 保持向后兼容的接口,同时提供模块访问接口
        \end{itemize}
    \end{ytublock}
    \begin{columns}
        \column{0.48\textwidth}
        \begin{block}{核心功能}
            \begin{itemize}
                \item \textbf{invert()}:执行反演主流程
                \item \textbf{generateRandomModel()}:生成随机模型
                \item \textbf{computeLayerThicknesses()}:计算层厚度
                \item 进度回调支持
            \end{itemize}
        \end{block}

        \column{0.48\textwidth}
        \begin{block}{模块访问}
            \begin{itemize}
                \item \textbf{getForwardSolver()}:获取正演求解器
                \item \textbf{getJacobianCalculator()}:获取Jacobian计算器
                \item \textbf{getRegularization()}:获取正则化模块
                \item \textbf{getOptimizer()}:获取优化求解器
            \end{itemize}
        \end{block}
    \end{columns}
\end{frame}

\begin{frame}{GUI界面模块(mt\_inversion\_gui)}
    \begin{ytublock}{核心功能}
        \begin{itemize}
            \item 使用Qt框架构建图形用户界面
            \item 支持参数设置、实时进度显示、结果可视化
            \item 使用多线程技术实现后台计算,避免阻塞UI
        \end{itemize}
    \end{ytublock}
    \begin{columns}
        \column{0.48\textwidth}
        \begin{block}{界面组件}
            \begin{itemize}
                \item \textbf{参数面板}:层数、频率点数、正则化参数等
                \item \textbf{结果表格}:显示各层的真实模型、初始模型、反演结果
                \item \textbf{图表显示}:模型对比图、视电阻率曲线、相位曲线、残差下降曲线
                \item \textbf{日志输出}:显示反演过程的详细信息
            \end{itemize}
        \end{block}

        \column{0.48\textwidth}
        \begin{block}{技术特点}
            \begin{itemize}
                \item \textbf{多线程}:使用\texttt{QThread}实现后台计算
                \item \textbf{信号槽}:使用Qt信号槽机制实现线程间通信
                \item \textbf{Qt Charts}:使用Qt Charts进行数据可视化
                \item \textbf{智能指针}:使用\texttt{QPointer}管理线程指针
            \end{itemize}
        \end{block}
    \end{columns}
\end{frame}

\begin{frame}{知识点:命名空间(namespace)}
    \begin{ytublock}{命名空间的作用}
        \begin{itemize}
            \item 命名空间用于组织代码,避免命名冲突
            \item 所有MT相关模块定义在\texttt{MT}命名空间中
            \item 使用方式:\texttt{MT::ForwardSolver}、\texttt{MT::ModelParams}等
        \end{itemize}
    \end{ytublock}
\end{frame}

\begin{frame}{知识点:Intel MKL库}
    \begin{ytublock}{MKL简介}
        \begin{itemize}
            \item Intel Math Kernel Library(MKL)是高性能数学库
            \item 提供优化的BLAS、LAPACK、FFT等函数
            \item 支持多线程并行计算,充分利用CPU性能
        \end{itemize}
    \end{ytublock}
    \begin{ytublock}{本项目使用的MKL函数}
        \begin{itemize}
            \item \textbf{BLAS}:\texttt{cblas\_dsyrk}(对称矩阵乘法)、\texttt{cblas\_dgemv}(矩阵向量乘法)
            \item \textbf{LAPACK}:\texttt{LAPACKE\_dposv}(对称正定线性方程组求解)
            \item \textbf{复数运算}:\texttt{MKL\_Complex16}类型及相关运算
        \end{itemize}
    \end{ytublock}
\end{frame}

\begin{frame}{知识点:Qt框架}
    \begin{ytublock}{Qt简介}
        \begin{itemize}
            \item Qt是跨平台的C++图形用户界面框架
            \item 提供丰富的GUI组件、信号槽机制、多线程支持等
            \item 支持Windows、Linux、macOS等多个平台
        \end{itemize}
    \end{ytublock}
    \begin{ytublock}{本项目使用的Qt特性}
        \begin{itemize}
            \item \textbf{QWidget}:窗口和控件基类
            \item \textbf{QThread}:多线程支持,实现后台计算
            \item \textbf{Qt Charts}:数据可视化(折线图、散点图等)
            \item \textbf{信号槽}:实现对象间通信和线程间通信
            \item \textbf{智能指针}:\texttt{QPointer}自动管理对象生命周期
        \end{itemize}
    \end{ytublock}
\end{frame}

\begin{frame}{知识点:多线程编程}
    \begin{ytublock}{QThread使用}
        \begin{itemize}
            \item \texttt{QThread}是Qt提供的线程类,继承自\texttt{QObject}
            \item 重写\texttt{run()}方法实现线程逻辑
            \item 使用信号槽机制实现线程间通信
        \end{itemize}
    \end{ytublock}
    \begin{columns}
        \column{0.48\textwidth}
        \begin{block}{线程类定义}
            \begin{minted}[fontsize=\tiny]{cpp}
class InversionWorkerThread 
    : public QThread {
    Q_OBJECT
protected:
    void run() override {
        // 执行反演计算
        m_result = m_core->invert(params);
        emit inversionFinished(m_result);
    }
signals:
    void progressUpdated(int, double, double);
    void inversionFinished(const Result&);
};
            \end{minted}
        \end{block}

        \column{0.48\textwidth}
        \begin{block}{线程安全要点}
            \begin{itemize}
                \item \textbf{volatile}:确保多线程可见性
                \item \textbf{QPointer}:自动管理线程指针
                \item \textbf{信号槽}:线程安全的通信机制
                \item \textbf{优雅停止}:使用标志位请求停止
            \end{itemize}
        \end{block}
    \end{columns}
\end{frame}

\begin{frame}{知识点:智能指针}
    \begin{ytublock}{std::unique\_ptr}
        \begin{itemize}
            \item \texttt{std::unique\_ptr}是C++11引入的智能指针
            \item 独占所有权,自动管理内存
            \item 不可复制,但可以移动
        \end{itemize}
    \end{ytublock}
    \begin{ytublock}{QPointer}
        \begin{itemize}
            \item \texttt{QPointer}是Qt提供的智能指针
            \item 专门用于管理\texttt{QObject}及其派生类
            \item 对象被删除时自动置为\texttt{nullptr},避免悬空指针
            \item 适用于多线程环境中的对象管理
        \end{itemize}
    \end{ytublock}
\end{frame}

\begin{frame}{知识点:模块化设计}
    \begin{ytublock}{模块化原则}
        \begin{itemize}
            \item \textbf{单一职责}:每个模块只负责一个功能
            \item \textbf{低耦合}:模块之间依赖关系简单清晰
            \item \textbf{高内聚}:模块内部功能紧密相关
            \item \textbf{可扩展}:易于添加新功能或替换实现
        \end{itemize}
    \end{ytublock}
    \begin{ytublock}{本项目模块化优势}
        \begin{itemize}
            \item 每个模块可以独立测试和维护
            \item 可以轻松替换实现(如不同的正演方法)
            \item 代码结构清晰,易于理解和扩展
            \item 支持高级定制,同时保持简单接口
        \end{itemize}
    \end{ytublock}
\end{frame}

\begin{frame}{CMake构建系统}
    \begin{ytublock}{CMake配置}
        \begin{itemize}
            \item 使用CMake管理项目构建
            \item 自动检测MKL库和Qt库
            \item 支持Windows、Linux、macOS多平台
        \end{itemize}
    \end{ytublock}
    \begin{columns}
        \column{0.48\textwidth}
        \begin{block}{关键配置}
            \begin{itemize}
                \item 查找MKL库:\texttt{find\_package(MKL)}
                \item 查找Qt库:\texttt{find\_package(Qt5)}
                \item 设置编译选项:C++17标准
                \item 链接库:MKL、Qt5::Core、Qt5::Widgets、Qt5::Charts
            \end{itemize}
        \end{block}

        \column{0.48\textwidth}
        \begin{block}{构建步骤}
            \begin{itemize}
                \item \texttt{cmake ..}:配置项目
                \item \texttt{cmake --build .}:编译项目
                \item 生成可执行文件:\texttt{mt1d\_inversion\_gui.exe}
            \end{itemize}
        \end{block}
    \end{columns}
\end{frame}

\begin{frame}{系统测试与验证}
    \begin{columns}
        \column{0.48\textwidth}
        \begin{block}{测试策略}
            \begin{itemize}
                \item \textbf{单元测试}:每个模块独立测试
                \item \textbf{集成测试}:测试模块间的协作
                \item \textbf{功能测试}:测试完整反演流程
                \item \textbf{性能测试}:测试计算效率和内存使用
            \end{itemize}
        \end{block}

        \column{0.48\textwidth}
        \begin{block}{验证方法}
            \begin{itemize}
                \item 使用已知模型测试正演计算精度
                \item 使用合成数据测试反演恢复能力
                \item 检查残差下降曲线是否收敛
                \item 验证GUI界面的响应性和稳定性
            \end{itemize}
        \end{block}
    \end{columns}
    \begin{ytublock}{测试结果}
        \begin{itemize}
            \item 正演计算精度:相对误差 $< 0.1\%$
            \item 反演恢复能力:对已知模型的反演误差 $< 5\%$
            \item 计算效率:单次反演(40层,61频点)耗时 $< 10$ 秒
            \item GUI响应性:后台计算不阻塞界面,实时更新进度
        \end{itemize}
    \end{ytublock}
\end{frame}

\begin{frame}{总结}
    \begin{columns}
        \column{0.48\textwidth}
        \begin{block}{系统特点}
            \begin{itemize}
                \item \textbf{模块化设计}:清晰的模块划分
                \item \textbf{高性能计算}:MKL库优化
                \item \textbf{友好界面}:Qt GUI可视化
                \item \textbf{多线程支持}:后台计算不阻塞
                \item \textbf{代码规范}:现代C++特性
            \end{itemize}
        \end{block}

        \column{0.48\textwidth}
        \begin{block}{技术要点}
            \begin{itemize}
                \item 命名空间组织代码
                \item MKL库高性能计算
                \item Qt框架构建GUI
                \item 多线程后台计算
                \item 智能指针管理内存
                \item CMake项目构建
            \end{itemize}
        \end{block}
    \end{columns}
    \begin{ytublock}{学习收获}
        \begin{itemize}
            \item 掌握了MT反演的理论基础和实现方法
            \item 学习了模块化设计、高性能计算、GUI开发等实践技能
            \item 理解了科学计算软件开发的完整流程
            \item 提升了C++编程能力和工程实践能力
        \end{itemize}
    \end{ytublock}
\end{frame}

\end{document}

