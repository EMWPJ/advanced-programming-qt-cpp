\documentclass[UTF8,12pt,a4paper]{ctexart}

% 基本包
\usepackage{amsmath}
\usepackage{amsfonts}
\usepackage{amssymb}
\usepackage{graphicx}
\usepackage{booktabs}
\usepackage{multirow}
\usepackage{geometry}
\usepackage{fancyhdr}
\usepackage[most]{tcolorbox}
\usepackage{hyperref}
\usepackage{enumitem}
\usepackage{tikz}
\usepackage{pgfplots}
\usepackage{siunitx} % 改进的单位支持
\pgfplotsset{compat=1.18}

% 单位设置
\sisetup{
    inter-unit-product = \ensuremath{{}\cdot{}},
    per-mode = symbol
}

% 页面设置
\geometry{left=2.5cm,right=2.5cm,top=3cm,bottom=3cm,headheight=14.5pt}
\pagestyle{fancy}
\fancyhf{}
\fancyhead[C]{大地电磁一维正反演理论}
\fancyfoot[C]{\thepage}

% 自定义颜色
\definecolor{ytublue}{RGB}{0,84,159}
\definecolor{theoremcolor}{RGB}{230,240,255}

% 自定义环境
\newtcolorbox{theorembox}[1]{
    colback=theoremcolor,
    colframe=ytublue,
    title=#1,
    fonttitle=\bfseries,
    arc=3mm,
    boxrule=1pt,
    boxsep=2mm,
    breakable % 允许跨页
}

\newtcolorbox{definitionbox}[1]{
    colback=white,
    colframe=ytublue!50!black,
    title=#1,
    fonttitle=\bfseries,
    arc=3mm,
    boxrule=1pt,
    boxsep=2mm,
    breakable % 允许跨页
}


% 标题信息
\title{\textbf{大地电磁一维正反演理论讲义}}
\author{王培杰}
\date{\today}

\begin{document}

\maketitle

\setcounter{tocdepth}{2} % 目录只显示到二级标题(subsection),不显示三级标题(subsubsection)
\tableofcontents
\newpage

\section{引言}

\subsection{大地电磁法简介}

大地电磁法(Magnetotelluric, MT)是一种被动源地球物理勘探方法,利用天然电磁场作为场源,通过观测地表电磁场的变化来推断地下介质的电性结构。该方法由Tikhonov(1950年)和Cagniard(1953年)独立提出,具有以下特点:

\begin{itemize}
    \item \textbf{探测深度大}:利用低频电磁波(周期从毫秒到数千秒),可探测从几米到数百公里的深度
    \item \textbf{成本低}:使用天然场源,无需人工发射设备
    \item \textbf{不受地形限制}:适合复杂地形条件下的勘探
    \item \textbf{对电阻率敏感}:可有效区分导电和电阻层
\end{itemize}

MT方法广泛应用于矿产资源勘查、地热资源开发、深部地质结构研究、地震活动带研究等领域。

\subsection{一维反演的基本假设}

一维MT反演假设地下介质在水平方向上均匀,仅在垂直方向上变化,即介质参数只依赖于深度$z$。这种假设适用于:

\begin{itemize}
    \item 层状地质结构
    \item 水平层状介质
    \item 缓变的地质构造(水平尺度远大于垂直尺度)
\end{itemize}

虽然实际地质结构往往是二维或三维的,但一维反演作为基础方法,具有以下优点:

\begin{itemize}
    \item 计算效率高:正演计算速度快,适合大规模数据处理
    \item 理论成熟:数学推导完整,数值方法稳定
    \item 易于理解:是学习MT反演理论的重要起点
    \item 作为初始模型:为二维、三维反演提供初始模型
\end{itemize}

\section{MT正演问题}

\subsection{物理模型}

\subsubsection{层状介质模型}

一维MT正演问题考虑$M$层水平层状介质,每层具有不同的电导率$\sigma_i$(或电阻率$\rho_i = 1/\sigma_i$)和厚度$d_i$。模型参数包括:

\begin{itemize}
    \item 第$i$层的电导率:$\sigma_i$(\si{S/m}),表示介质的导电能力
    \item 第$i$层的电阻率:$\rho_i = 1/\sigma_i$(\si{\ohm.m}),表示介质的电阻能力
    \item 第$i$层的厚度:$d_i$(\si{m}),表示该层的垂直厚度
    \item 第$i$层的深度:$z_i = \sum_{j=0}^{i-1} d_j$(\si{m}),表示该层顶部的深度
\end{itemize}

在实际计算中,通常使用电阻率的对数形式作为模型参数:
\begin{equation}
m_i = \log_{10}(\rho_i) = \log_{10}(1/\sigma_i) \label{eq:model_param}
\end{equation}

使用对数形式的原因:
\begin{itemize}
    \item 电阻率通常跨越多个数量级(如\SI{0.1}{\ohm.m}到\SI{10000}{\ohm.m}),对数变换使参数空间更均匀
    \item 避免数值计算的精度问题
    \item 使优化算法更容易收敛
\end{itemize}

\subsubsection{频率范围}

MT方法使用宽频带的天然电磁场,频率范围通常从\SI{e-3}{Hz}到\SI{e3}{Hz}(周期从\SI{e-3}{s}到\SI{e3}{s})。不同频率的电磁波具有不同的穿透深度:

\begin{itemize}
    \item \textbf{高频}(短周期):穿透深度浅,反映浅部结构
    \item \textbf{低频}(长周期):穿透深度深,反映深部结构
\end{itemize}

频率点数通常取61个,在对数尺度上均匀分布:
\begin{equation}
\log_{10}(T_i) = \log_{10}(T_{\min}) + \frac{i-1}{n-1}[\log_{10}(T_{\max}) - \log_{10}(T_{\min})] \label{eq:freq_dist}
\end{equation}
其中$T_i = 1/f_i$为第$i$个频点的周期,$n$为频率点数(通常$n=61$),$T_{\min}$和$T_{\max}$分别为最小和最大周期。

\subsection{Maxwell方程与Helmholtz方程}

\subsubsection{Maxwell方程组}

对于TE(横电)极化模式,电场垂直于传播方向。在一维情况下,假设电场沿$x$方向,磁场沿$y$方向,传播沿$z$方向(垂直向下)。Maxwell方程组简化为:

\begin{align}
    \frac{\partial E_x}{\partial z} &= i\omega\mu_0 H_y \label{eq:faraday_scalar} \\
    \frac{\partial H_y}{\partial z} &= \sigma E_x \label{eq:ampere_scalar}
\end{align}

其中:
\begin{itemize}
    \item $E_x(z)$:$x$方向的电场分量(\si{V/m})
    \item $H_y(z)$:$y$方向的磁场分量(\si{A/m})
    \item $\mu_0 = 4\pi \times 10^{-7}$ \si{H/m}:真空磁导率
    \item $\omega = 2\pi f$:角频率(\si{rad/s})
    \item $\sigma$:电导率(\si{S/m})
\end{itemize}

\subsubsection{Helmholtz方程的推导}

从方程~\eqref{eq:faraday_scalar}和~\eqref{eq:ampere_scalar}可以推导出Helmholtz方程。

首先,对方程~\eqref{eq:faraday_scalar}两边关于$z$求导:
\begin{equation}
\frac{\partial^2 E_x}{\partial z^2} = i\omega\mu_0 \frac{\partial H_y}{\partial z}
\end{equation}

将方程~\eqref{eq:ampere_scalar}代入上式,得到:
\begin{equation}
\frac{\partial^2 E_x}{\partial z^2} = i\omega\mu_0 \sigma E_x
\end{equation}

整理得到Helmholtz方程:
\begin{equation}
\frac{d^2 E_x}{dz^2} + k^2 E_x = 0 \label{eq:helmholtz}
\end{equation}

其中$k$为波数(复数),定义为:
\begin{equation}
k = \sqrt{i\omega\mu_0\sigma} \label{eq:wavenumber_complex}
\end{equation}

\subsubsection{波数的物理意义}

波数$k$是复数,可以表示为:
\begin{equation}
k = (1+i)\sqrt{\frac{\omega\mu_0\sigma}{2}} = \alpha + i\beta \label{eq:wavenumber}
\end{equation}

其中:
\begin{itemize}
    \item $\alpha = \beta = \sqrt{\frac{\omega\mu_0\sigma}{2}}$:衰减系数和相位系数(实数部分和虚数部分相等)
    \item 电磁波在导电介质中传播时会衰减,衰减长度(趋肤深度)为$\delta = 1/\alpha = \sqrt{\frac{2}{\omega\mu_0\sigma}}$
    \item 趋肤深度决定了该频率电磁波的探测深度
\end{itemize}

\textbf{重要物理意义}:频率越低、电导率越小,趋肤深度越大,即穿透深度越深。这解释了为什么MT方法能够探测深部结构。

\subsection{向上递推阻抗法}

\subsubsection{基本思想}

向上递推阻抗法是一种解析方法,基于传输线理论,利用层间阻抗的连续性条件,从最底层(半空间)开始,逐层向上递推到地表,计算地表阻抗。

\textbf{核心思想}:
\begin{enumerate}
    \item 每层介质可以看作一个传输线,具有特征阻抗
    \item 在层界面处,阻抗必须连续(物理边界条件)
    \item 从最底层(已知)开始,利用连续性条件逐层向上计算
\end{enumerate}

\subsubsection{阻抗的定义}

在MT方法中,阻抗定义为电场与磁场的比值:
\begin{equation}
Z(z) = \frac{E_x(z)}{H_y(z)} \label{eq:impedance_def}
\end{equation}

阻抗的单位是欧姆(\si{\ohm}),反映了介质的电性特征。

\subsubsection{特征阻抗和波数}

对于第$i$层均匀介质,求解Helmholtz方程~\eqref{eq:helmholtz}可以得到电磁场的解析解。在第$i$层中,电场可以表示为:
\begin{equation}
E_x(z) = A_i e^{-k_i z} + B_i e^{k_i z}
\end{equation}
其中$A_i$和$B_i$为待定系数,$k_i$为第$i$层的波数:
\begin{equation}
k_i = \sqrt{i\omega\mu_0\sigma_i} = (1+i)\sqrt{\frac{\omega\mu_0\sigma_i}{2}} \label{eq:layer_wavenumber}
\end{equation}

第$i$层的特征阻抗定义为:
\begin{equation}
Z_{0i} = \frac{i\omega\mu_0}{k_i} = \sqrt{\frac{i\omega\mu_0}{\sigma_i}} = (1+i)\sqrt{\frac{\omega\mu_0}{2\sigma_i}} \label{eq:char_impedance}
\end{equation}

特征阻抗反映了介质的固有电性特征,类似于传输线的特性阻抗。

\subsubsection{阻抗递推公式的推导}

考虑两层介质系统:第$i$层(厚度$d_i$)覆盖在第$i+1$层之上。在层界面处,电场和磁场必须连续,因此阻抗也连续。

在第$i$层内部,考虑向上和向下传播的波的叠加。经过推导(基于传输线理论),可以得到从第$i+1$层的阻抗$Z_{i+1}$计算第$i$层顶部阻抗$Z_i$的递推公式:

\begin{equation}
Z_i = Z_{0i} \cdot \frac{Z_{i+1} + Z_{0i} \tanh(k_i d_i)}{Z_{0i} + Z_{i+1} \tanh(k_i d_i)} \label{eq:recursive}
\end{equation}

其中:
\begin{itemize}
    \item $Z_i$:第$i$层顶部的阻抗
    \item $Z_{i+1}$:第$i+1$层顶部的阻抗(即第$i$层底部的阻抗)
    \item $Z_{0i}$:第$i$层的特征阻抗
    \item $k_i$:第$i$层的波数
    \item $d_i$:第$i$层的厚度
    \item $\tanh(k_i d_i)$:复双曲正切函数
\end{itemize}

\subsubsection{复双曲正切函数}

对于复数$z = x + iy$,双曲正切函数定义为:
\begin{equation}
\tanh(z) = \frac{\sinh(z)}{\cosh(z)} = \frac{e^z - e^{-z}}{e^z + e^{-z}} \label{eq:tanh_def}
\end{equation}

对于$z = k_i d_i$,其中$k_i = (1+i)\alpha_i$,$\alpha_i = \sqrt{\frac{\omega\mu_0\sigma_i}{2}}$:
\begin{equation}
e^{k_i d_i} = e^{\alpha_i d_i} e^{i\alpha_i d_i} = e^{\alpha_i d_i}[\cos(\alpha_i d_i) + i\sin(\alpha_i d_i)] \label{eq:exp_complex}
\end{equation}

因此:
\begin{align}
\sinh(k_i d_i) &= \frac{1}{2}(e^{k_i d_i} - e^{-k_i d_i}) \label{eq:sinh} \\
\cosh(k_i d_i) &= \frac{1}{2}(e^{k_i d_i} + e^{-k_i d_i}) \label{eq:cosh}
\end{align}

\subsubsection{递推算法}

\textbf{初始化}:最底层(第$M-1$层)作为半空间,其阻抗等于特征阻抗:
\begin{equation}
Z_{M-1} = Z_{0,M-1} = \sqrt{\frac{i\omega\mu_0}{\sigma_{M-1}}} \label{eq:bottom_impedance}
\end{equation}

\textbf{递推过程}:
\begin{enumerate}
    \item 从$i = M-2$开始(倒数第二层)
    \item 使用公式~\eqref{eq:recursive}计算$Z_i$
    \item $i = i-1$,重复步骤2
    \item 直到$i = 0$(地表层),得到地表阻抗$Z_0$
\end{enumerate}

\textbf{算法伪代码}:
\begin{enumerate}[leftmargin=*,align=left]
    \item[步骤1] 初始化:$Z_{M-1} = Z_{0,M-1}$(最底层阻抗)
    \item[步骤2] 循环:$i$从$M-2$递减到$0$
    \begin{enumerate}
        \item 计算$k_i = \sqrt{i\omega\mu_0\sigma_i}$
        \item 计算$\tanh(k_i d_i)$
        \item 使用公式~\eqref{eq:recursive}计算$Z_i$
    \end{enumerate}
    \item[步骤3] 输出地表阻抗$Z_0$
\end{enumerate}

\subsection{MT响应计算}

计算得到地表阻抗$Z_0$后,可计算MT响应(视电阻率和相位):

\subsubsection{视电阻率}

\begin{theorembox}{视电阻率}
视电阻率定义为:
\begin{equation}
\rho_a = \frac{|Z_0|^2}{\omega\mu_0} = \frac{Z_0 \cdot Z_0^*}{\omega\mu_0} \label{eq:apparent_resistivity}
\end{equation}
其中$Z_0^*$为$Z_0$的共轭复数,$|Z_0|^2 = [\text{Re}(Z_0)]^2 + [\text{Im}(Z_0)]^2$。
\end{theorembox}

\textbf{物理意义}:
\begin{itemize}
    \item 视电阻率$\rho_a$是综合反映地下一定深度范围内介质电性的等效电阻率
    \item 它不是某一层的真实电阻率,而是多层介质的综合响应
    \item 视电阻率的大小反映了介质的导电能力:$\rho_a$大表示电阻性强(导电性差),$\rho_a$小表示导电性强
    \item 不同频率的视电阻率反映不同深度的电性结构(高频反映浅部,低频反映深部)
\end{itemize}

\textbf{推导说明}:对于均匀半空间,阻抗$Z = \sqrt{i\omega\mu_0/\sigma}$,代入上式可以得到$\rho_a = 1/\sigma = \rho$,即视电阻率等于真实电阻率。对于层状介质,视电阻率是等效值。

\subsubsection{相位}

\begin{theorembox}{相位}
相位定义为:
\begin{equation}
\phi = \arctan\left(\frac{\text{Im}(Z_0)}{\text{Re}(Z_0)}\right) \quad \text{(弧度)} \label{eq:phase_rad}
\end{equation}
或
\begin{equation}
\phi = \frac{180}{\pi} \arctan\left(\frac{\text{Im}(Z_0)}{\text{Re}(Z_0)}\right) \quad \text{(度)} \label{eq:phase_deg}
\end{equation}
\end{theorembox}

\textbf{物理意义}:
\begin{itemize}
    \item 相位$\phi$表示电场和磁场之间的相位差,反映了电磁场在传播过程中的延迟
    \item 相位是复数阻抗的幅角,范围通常在$0^\circ$到$90^\circ$之间
    \item 对于均匀半空间,相位为$45^\circ$
    \item 相位对电阻率变化非常敏感,是反演的重要约束信息
    \item 相位信息有助于减少反演的非唯一性
\end{itemize}

\textbf{实际应用}:在实际计算中,通常以$\log_{10}(\rho_a)$和相位$\phi$(度)的形式存储,便于后续处理和可视化。MT数据通常绘制为视电阻率和相位随频率(或周期)变化的曲线,称为MT曲线或视电阻率-相位曲线。

\section{MT反演问题}

\subsection{反演问题的数学表述}

\subsubsection{问题定义}

给定观测数据$\mathbf{d}_{obs} \in \mathbb{R}^{n}$($n$为数据点数,通常$n = 2n_{freq}$,包含视电阻率和相位),寻找模型参数$\mathbf{m} \in \mathbb{R}^{M}$($M$为模型层数),使得合成数据$\mathbf{d}_{syn}(\mathbf{m})$与观测数据$\mathbf{d}_{obs}$的差异最小。

正演算子定义为:
\begin{equation}
\mathbf{d}_{syn} = F(\mathbf{m}) \label{eq:forward_operator}
\end{equation}
其中$F: \mathbb{R}^{M} \rightarrow \mathbb{R}^{n}$为非线性正演算子,将$M$维模型参数映射到$n$维数据空间。

\textbf{数据向量结构}:$\mathbf{d}_{obs}$通常包含
\begin{itemize}
    \item 前$n_{freq}$个元素:$\log_{10}(\rho_a)$(对数视电阻率)
    \item 后$n_{freq}$个元素:$\phi$(相位,单位:度)
\end{itemize}
其中$n_{freq}$为频率点数。

\textbf{模型向量结构}:$\mathbf{m}$包含
\begin{itemize}
    \item $m_i = \log_{10}(\rho_i)$:第$i$层电阻率的对数($i = 0, 1, \ldots, M-1$)
    \item 每层的厚度$d_i$通常预先设定(在对数深度尺度上等间距分布)
\end{itemize}

\subsubsection{不适定性问题}

反演问题通常是不适定的(ill-posed),表现为:

\begin{itemize}
    \item \textbf{解不唯一}:多个不同的模型可能产生相同或非常相似的观测数据。这是由于:
    \begin{itemize}
        \item 数据量有限(通常几十到上百个数据点)
        \item 模型参数较多(通常几十层)
        \item 数据对深部的敏感度较低
    \end{itemize}
    
    \item \textbf{解不稳定}:观测数据的小扰动(测量误差)可能导致模型的大变化。这要求:
    \begin{itemize}
        \item 引入先验信息约束模型
        \item 使用正则化方法稳定解
    \end{itemize}
    
    \item \textbf{过度拟合}:模型可能过度复杂,包含虚假结构(由噪声引起的)。需要通过正则化惩罚复杂模型。
\end{itemize}

因此,需要引入正则化约束来稳定反演过程,得到合理的、稳定的解。

\subsection{目标函数}

\subsubsection{数据拟合项}

数据拟合项衡量合成数据与观测数据的差异,使用L2范数(最小二乘):
\begin{equation}
\Phi_d(\mathbf{m}) = \|\mathbf{d}_{obs} - \mathbf{d}_{syn}(\mathbf{m})\|^2 = \sum_{i=1}^{n} [d_{obs,i} - d_{syn,i}(\mathbf{m})]^2 \label{eq:data_fit}
\end{equation}

数据拟合项的物理意义:
\begin{itemize}
    \item 衡量模型预测与观测的吻合程度
    \item 越小表示模型越能解释观测数据
    \item 但过小的拟合误差可能表示过度拟合
\end{itemize}

\subsubsection{正则化项}

正则化项引入先验信息,约束模型的平滑性或简单性:
\begin{equation}
\Phi_m(\mathbf{m}) = \|L \mathbf{m}\|^2 \label{eq:regularization}
\end{equation}
其中$L$为正则化矩阵,编码先验信息。

正则化项的物理意义:
\begin{itemize}
    \item 反映我们对模型的先验认知(如:模型应该平滑)
    \item 惩罚复杂模型,防止过度拟合
    \item 使解更稳定,更符合地质实际情况
\end{itemize}

\subsubsection{总目标函数}

阻尼最小二乘(Tikhonov正则化)目标函数为:
\begin{equation}
\Phi(\mathbf{m}) = \Phi_d(\mathbf{m}) + \lambda \Phi_m(\mathbf{m}) = \|\mathbf{d}_{obs} - \mathbf{d}_{syn}(\mathbf{m})\|^2 + \lambda \|L \mathbf{m}\|^2 \label{eq:objective}
\end{equation}

其中:
\begin{itemize}
    \item $\Phi_d(\mathbf{m})$:数据拟合项,衡量数据拟合程度
    \item $\Phi_m(\mathbf{m})$:模型约束项(正则化项),衡量模型的复杂程度
    \item $\lambda > 0$:正则化参数,平衡数据拟合和模型约束
    \begin{itemize}
        \item $\lambda$大:更重视模型平滑,可能牺牲数据拟合
        \item $\lambda$小:更重视数据拟合,模型可能振荡
    \end{itemize}
    \item $L$:正则化矩阵,编码先验信息
\end{itemize}

\subsection{正则化矩阵}

\subsubsection{平滑度约束(SMOOTHNESS)}

二阶差分矩阵,惩罚模型参数的二阶导数,使模型平滑:
\begin{equation}
L_{ij} = \begin{cases}
    1, & j = i \\
    -2, & j = i+1 \\
    1, & j = i+2 \\
    0, & \text{其他}
\end{cases} \label{eq:smoothness_matrix}
\end{equation}

对应的约束为:
\begin{equation}
\Phi_m(\mathbf{m}) = \sum_{i=1}^{M-2} (m_i - 2m_{i+1} + m_{i+2})^2
\end{equation}

这惩罚相邻层之间的二阶差分(即二阶导数),使模型平滑。正则化矩阵$L$的维度为$(M-2) \times M$。

\textbf{物理意义}:假设真实地球的电性结构变化是平滑的,避免出现剧烈跳跃的虚假结构。

\subsubsection{平坦度约束(FLATNESS)}

一阶差分矩阵,惩罚模型梯度:
\begin{equation}
L_{ij} = \begin{cases}
    -1, & j = i \\
    1, & j = i+1 \\
    0, & \text{其他}
\end{cases} \label{eq:flatness_matrix}
\end{equation}

对应的约束为:
\begin{equation}
\Phi_m(\mathbf{m}) = \sum_{i=1}^{M-1} (m_{i+1} - m_i)^2
\end{equation}

这惩罚相邻层之间的变化,使模型变化平坦。正则化矩阵$L$的维度为$(M-1) \times M$。

\subsubsection{最小范数约束(MINIMUM\_NORM)}

单位矩阵,最小化模型参数的L2范数:
\begin{equation}
L = I \label{eq:minnorm_matrix}
\end{equation}

对应的约束为:
\begin{equation}
\Phi_m(\mathbf{m}) = \sum_{i=1}^{M} m_i^2
\end{equation}

这使模型参数尽可能小(接近参考模型)。正则化矩阵$L$为$M \times M$单位矩阵。

\textbf{实际应用}:平滑度约束(SMOOTHNESS)最为常用,因为它最符合大多数地质结构的实际情况。

\subsection{高斯-牛顿法}

\subsubsection{非线性问题的线性化}

正演算子$F(\mathbf{m})$是非线性的,无法直接求解。在模型$\mathbf{m}^{(k)}$附近进行一阶泰勒展开:
\begin{equation}
\mathbf{d}_{syn}(\mathbf{m}) \approx \mathbf{d}_{syn}(\mathbf{m}^{(k)}) + J^{(k)} (\mathbf{m} - \mathbf{m}^{(k)}) \label{eq:linearization}
\end{equation}

其中$J^{(k)}$为Jacobian矩阵(也称为灵敏度矩阵或梯度矩阵),元素为:
\begin{equation}
J_{ij}^{(k)} = \frac{\partial d_{syn,i}}{\partial m_j}\Big|_{\mathbf{m}=\mathbf{m}^{(k)}} \label{eq:jacobian}
\end{equation}

Jacobian矩阵的维度为$n \times M$,表示每个数据对每个模型参数的灵敏度(即:模型参数变化时,数据的变化率)。

\textbf{物理意义}:
\begin{itemize}
    \item $J_{ij}$大:数据$i$对模型参数$j$敏感
    \item $J_{ij}$小:数据$i$对模型参数$j$不敏感
    \item 通常浅层参数对高频数据敏感,深层参数对低频数据敏感
\end{itemize}

\subsubsection{残差向量}

定义残差向量(数据拟合误差):
\begin{equation}
\mathbf{r}^{(k)} = \mathbf{d}_{obs} - \mathbf{d}_{syn}(\mathbf{m}^{(k)}) \label{eq:residual}
\end{equation}

线性化后的数据拟合项为:
\begin{equation}
\Phi_d(\mathbf{m}) \approx \|\mathbf{r}^{(k)} - J^{(k)} \delta\mathbf{m}\|^2 \label{eq:linearized_data_fit}
\end{equation}
其中$\delta\mathbf{m} = \mathbf{m} - \mathbf{m}^{(k)}$为模型更新量。

\subsubsection{正规方程的推导}

将目标函数在$\mathbf{m}^{(k)}$附近线性化:
\begin{equation}
\Phi(\mathbf{m}) \approx \|\mathbf{r}^{(k)} - J^{(k)} \delta\mathbf{m}\|^2 + \lambda \|L (\mathbf{m}^{(k)} + \delta\mathbf{m})\|^2 \label{eq:linearized_objective}
\end{equation}

展开第一项:
\begin{equation}
\|\mathbf{r}^{(k)} - J^{(k)} \delta\mathbf{m}\|^2 = (\mathbf{r}^{(k)} - J^{(k)} \delta\mathbf{m})^T (\mathbf{r}^{(k)} - J^{(k)} \delta\mathbf{m})
\end{equation}
\begin{equation}
= (\mathbf{r}^{(k)})^T \mathbf{r}^{(k)} - 2 (\mathbf{r}^{(k)})^T J^{(k)} \delta\mathbf{m} + (\delta\mathbf{m})^T (J^{(k)})^T J^{(k)} \delta\mathbf{m}
\end{equation}

展开第二项(忽略常数项$\|L \mathbf{m}^{(k)}\|^2$):
\begin{equation}
\lambda \|L (\mathbf{m}^{(k)} + \delta\mathbf{m})\|^2 \approx \lambda [(\mathbf{m}^{(k)})^T L^T L \mathbf{m}^{(k)} + 2 (\mathbf{m}^{(k)})^T L^T L \delta\mathbf{m} + (\delta\mathbf{m})^T L^T L \delta\mathbf{m}]
\end{equation}

对$\delta\mathbf{m}$求导并令其为零:
\begin{equation}
\frac{\partial \Phi}{\partial \delta\mathbf{m}} = -2 (J^{(k)})^T \mathbf{r}^{(k)} + 2 (J^{(k)})^T J^{(k)} \delta\mathbf{m} + 2\lambda L^T L (\mathbf{m}^{(k)} + \delta\mathbf{m}) = 0
\end{equation}

整理得到正规方程(Normal Equation):
\begin{equation}
[(J^{(k)})^T J^{(k)} + \lambda L^T L] \delta\mathbf{m} = (J^{(k)})^T \mathbf{r}^{(k)} - \lambda L^T L \mathbf{m}^{(k)} \label{eq:normal_equation}
\end{equation}

当初始模型$\mathbf{m}^{(0)}$接近零或正则化项较小时,可忽略$\lambda L^T L \mathbf{m}^{(k)}$项,得到简化形式:
\begin{equation}
[(J^{(k)})^T J^{(k)} + \lambda L^T L] \delta\mathbf{m} = (J^{(k)})^T \mathbf{r}^{(k)} \label{eq:normal_equation_simple}
\end{equation}

\textbf{方程性质}:
\begin{itemize}
    \item 系数矩阵$(J^{(k)})^T J^{(k)} + \lambda L^T L$是对称正定矩阵
    \item 可使用Cholesky分解高效求解
    \item 正则化项$\lambda L^T L$使矩阵条件数改善,数值更稳定
\end{itemize}

\subsubsection{迭代更新}

模型更新公式为:
\begin{equation}
\mathbf{m}^{(k+1)} = \mathbf{m}^{(k)} + \delta\mathbf{m}^{(k)} \label{eq:update}
\end{equation}

\textbf{迭代过程}:
\begin{enumerate}
    \item 计算当前模型的合成数据$\mathbf{d}_{syn}(\mathbf{m}^{(k)})$
    \item 计算残差$\mathbf{r}^{(k)} = \mathbf{d}_{obs} - \mathbf{d}_{syn}(\mathbf{m}^{(k)})$
    \item 计算Jacobian矩阵$J^{(k)}$
    \item 求解正规方程得到$\delta\mathbf{m}^{(k)}$
    \item 更新模型:$\mathbf{m}^{(k+1)} = \mathbf{m}^{(k)} + \delta\mathbf{m}^{(k)}$
    \item 检查收敛判据,如未收敛则返回步骤1
\end{enumerate}

\subsection{Jacobian矩阵的计算}

\subsubsection{有限差分法}

由于正演算子的解析导数难以计算,使用有限差分法近似Jacobian矩阵。

\textbf{前向差分法}:
\begin{equation}
J_{ij} = \frac{\partial d_{syn,i}}{\partial m_j} \approx \frac{d_{syn,i}(\mathbf{m} + \epsilon \mathbf{e}_j) - d_{syn,i}(\mathbf{m})}{\epsilon} \label{eq:forward_diff}
\end{equation}

其中:
\begin{itemize}
    \item $\mathbf{e}_j$为第$j$个标准基向量(第$j$个分量为1,其他为0)
    \item $\epsilon$为扰动步长,通常$\epsilon = 10^{-5}$或$10^{-6}$
    \item 需要$M$次正演计算(每列一次)
\end{itemize}

\textbf{中心差分法}(更精确但计算量加倍):
\begin{equation}
J_{ij} \approx \frac{d_{syn,i}(\mathbf{m} + \epsilon \mathbf{e}_j) - d_{syn,i}(\mathbf{m} - \epsilon \mathbf{e}_j)}{2\epsilon} \label{eq:central_diff}
\end{equation}
需要$2M$次正演计算。

\textbf{计算复杂度}:
\begin{itemize}
    \item Jacobian矩阵大小为$n \times M$,通常$n \gg M$(如$n=122$,$M=40$)
    \item 前向差分法需要$M$次正演计算
    \item 中心差分法需要$2M$次正演计算
    \item 正演计算是反演的主要计算开销
\end{itemize}

\subsection{收敛判据与停止条件}

\subsubsection{收敛判据}

\textbf{模型更新判据}:模型更新量的L2范数小于容差
\begin{equation}
\|\delta\mathbf{m}^{(k)}\| < \text{tol\_dm} \label{eq:model_update_criterion}
\end{equation}
通常使用$\text{tol\_dm} = 10^{-4}$,表示模型已收敛(更新量很小)。

\textbf{残差判据}:数据拟合残差的L2范数小于容差
\begin{equation}
\|\mathbf{r}^{(k)}\| = \|\mathbf{d}_{obs} - \mathbf{d}_{syn}(\mathbf{m}^{(k)})\| < \text{tol\_residual} \label{eq:residual_criterion}
\end{equation}

\textbf{相对残差判据}:相对残差小于容差
\begin{equation}
\frac{\|\mathbf{r}^{(k)}\|}{\|\mathbf{d}_{obs}\|} < \text{tol\_relative} \label{eq:relative_residual_criterion}
\end{equation}
通常设置$\text{tol\_relative} = 10^{-3}$或更小。

\textbf{最大迭代次数}:防止无限迭代,通常设置$\text{max\_iter} = 20$。

\subsubsection{正则化参数的选择}

正则化参数$\lambda$的选择影响反演结果:
\begin{itemize}
    \item $\lambda$过大:模型过度平滑,数据拟合差,可能丢失真实结构
    \item $\lambda$过小:模型可能振荡,不稳定,可能包含虚假结构
    \item 需要通过L曲线法或交叉验证选择最优$\lambda$,平衡数据拟合和模型平滑度
\end{itemize}

\textbf{L曲线法}:绘制$\log(\|\mathbf{r}\|)$ vs $\log(\|L\mathbf{m}\|)$曲线,选择拐点处的$\lambda$值。

\section{数值实现要点}

\subsection{数值稳定性}

\subsubsection{复数运算}

MT正演涉及大量复数运算,需要注意:

\begin{itemize}
    \item 使用高精度浮点数(double精度,64位)进行所有计算
    \item 避免除零和数值溢出,检查分母是否为零
    \item 检查NaN(Not a Number)和Inf(无穷大)值
    \item 对双曲函数计算要注意数值稳定性(避免$e^x$溢出)
\end{itemize}

\textbf{数值稳定性技巧}:
\begin{itemize}
    \item 在计算$\tanh(z)$时,如果$|z|$很大,直接使用$\tanh(z) \approx \text{sign}(z)$
    \item 使用对数空间进行计算,避免数值溢出
\end{itemize}

\subsubsection{矩阵求解}

正规方程$(J^T J + \lambda L^T L) \delta\mathbf{m} = J^T \mathbf{r}$是对称正定线性方程组,可使用:

\begin{itemize}
    \item \textbf{Cholesky分解}:适用于对称正定矩阵,效率高,数值稳定
    \item \textbf{LU分解}:更通用但效率略低
    \item \textbf{QR分解}:数值稳定性最好,但计算量大
\end{itemize}

推荐使用Cholesky分解,因为系数矩阵是对称正定的。

\subsection{计算效率优化}

\subsubsection{矩阵运算优化}

使用高性能数学库(如Intel MKL、OpenBLAS)进行矩阵运算:

\begin{itemize}
    \item \texttt{cblas\_dsyrk}:计算$J^T J$(对称矩阵的秩$k$更新),利用对称性加速
    \item \texttt{cblas\_dgemv}:计算$J^T \mathbf{r}$(矩阵-向量乘法)
    \item \texttt{LAPACKE\_dposv}:求解对称正定线性方程组(Cholesky分解)
    \item \texttt{cblas\_dgemm}:通用矩阵乘法(备选方案)
\end{itemize}

这些库通常针对现代CPU进行了优化,包括SIMD(单指令多数据)指令集优化。

\subsubsection{并行计算}

MT正反演计算具有天然的并行性:

\begin{itemize}
    \item \textbf{频率循环并行化}:不同频率的正演计算相互独立,可并行执行。对于$n_{freq}$个频率,可使用$n_{freq}$个线程并行计算。
    \item \textbf{Jacobian矩阵并行化}:每个参数的扰动计算独立,可并行计算各列。对于$M$个参数,可使用$M$个线程并行计算。
    \item \textbf{矩阵运算并行化}:BLAS库内部支持多线程并行,自动利用多核CPU。
    \item 建议使用OpenMP或Intel TBB实现多线程并行。
\end{itemize}

\textbf{并行策略}:优先并行化频率循环(粗粒度并行),因为每个频率的正演计算时间较长,并行效率高。


\end{document}