\documentclass[UTF8,aspectratio=169]{beamer}



% 基本包
\usepackage[UTF8]{ctex}
\usepackage{graphicx}
\usepackage{amsmath}
\usepackage{amsfonts}
\usepackage{amssymb}
% \usepackage{listings}  % 已替换为minted
\usepackage{xcolor}
\usepackage{hyperref}
\usepackage{booktabs}
\usepackage{multirow}
\usepackage{multicol}
\usepackage{float}
\usepackage{tikz}
\usetikzlibrary{positioning,shapes,arrows,fit,backgrounds}
\usepackage{pgfplots}
\pgfplotsset{compat=1.18}
\usepackage{minted}
\usepackage{fontspec}
\usepackage[most]{tcolorbox}

% Beamer主题设置
\usetheme{Madrid}
\usecolortheme{whale}

% 校徽设置
\logo{
  \IfFileExists{../长江大学校徽.pdf}{
    \begin{tikzpicture}[remember picture,overlay]
      \node[anchor=north east, xshift=-0.2mm, yshift=-0.2mm] at (current page.north east) {
        \includegraphics[height=1.0cm]{../长江大学校徽.pdf}
      };
    \end{tikzpicture}
  }{
    \begin{tikzpicture}[remember picture,overlay]
      \node[anchor=north east, xshift=-3mm, yshift=-3mm] at (current page.north east) {
        \textcolor{red}{\tiny [校徽文件未找到]}
      };
    \end{tikzpicture}
  }
}

% ===== 使用推荐的 font themes =====
\usefonttheme{professionalfonts}  % 允许自定义字体

% 自定义frame标题栏,也缩短长度留出logo空间
% 设置标题栏背景颜色为淡蓝色
\definecolor{frametitlebg}{RGB}{200,215,250} % 淡蓝色,可根据需要调整
\setbeamercolor{frametitle}{bg=frametitlebg, fg=ytublue!80!black}
\setbeamertemplate{frametitle}{
  \ifbeamercolorempty[bg]{frametitle}{}{\nointerlineskip}%
  \begin{tcolorbox}[
    enhanced,
    width=0.90\paperwidth,
    height=2.5ex,
    colback=frametitlebg,
    colframe=frametitlebg,
    boxrule=0pt,
    left=0pt,
    right=0pt,
    top=1pt,
    bottom=0pt,
    boxsep=0pt,
    before skip=0pt,
    after skip=0.1em,  % 减少标题栏和内容之间的间距
  ]
  \vspace{0.2ex} % 减少标题文字上方的空白
  \usebeamerfont{frametitle}\textcolor{ytublue!80!black}{\hspace{1em}\insertframetitle}
  \end{tcolorbox}
}

% ===== 设置现代字体 =====
\setsansfont{Source Sans Pro}     % 正文字体
\setmonofont{Source Code Pro}[Scale=0.9]  % 代码字体,稍微缩小一点

\setbeamertemplate{navigation symbols}{}

% 减少页面间距
\setbeamertemplate{itemize items}[circle]
\setbeamertemplate{enumerate items}[default]
\setlength{\itemsep}{0.1em}
\setlength{\parskip}{0.1em}

% 页码设置
\setbeamertemplate{footline}[frame number]

% 定义流程图样式
\tikzset{
    block/.style = {rectangle, draw, fill=blue!10,
        minimum width=6em, align=center, rounded corners, minimum height=3em},
    line/.style = {draw, -latex'}
}
% 水印设置
\setbeamertemplate{background}{
    \begin{tikzpicture}[remember picture,overlay]
        \node[rotate=-45,scale=0.8,opacity=0.1,color=gray]
             at ([xshift=0.5cm,yshift=0.5cm]current page.south west)
             {\large\textbf{WPJ}};
    \end{tikzpicture}
}

% 自定义颜色
\definecolor{qtgreen}{RGB}{41,128,185}
\definecolor{qtblue}{RGB}{52,73,94}

% 定义长江大学蓝主色调
\definecolor{ytublue}{RGB}{0,84,159}
% 统一block样式
\newtcolorbox{ytublock}[1]{
  colback=white,
  colframe=ytublue!80!black,
  colbacktitle=ytublue!20!white,
  coltitle=ytublue!80!black,
  title={#1},
  fonttitle=\bfseries,
  arc=3mm,
  boxrule=1pt,
  boxsep=1mm,
  left=2mm,
  right=2mm,
  top=0.5mm,
  bottom=0.5mm,
  before skip=3pt,
  after skip=3pt,
  enhanced,
  drop fuzzy shadow=ytublue!20!black
}

% 定义警告块样式
\newtcolorbox{ytualertblock}[1]{
  colback=white,
  colframe=red!80!black,
  colbacktitle=red!20!white,
  coltitle=red!80!black,
  title={#1},
  fonttitle=\bfseries,
  arc=3mm,
  boxrule=1.5pt,
  boxsep=1mm,
  left=2mm,
  right=2mm,
  top=0.5mm,
  bottom=0.5mm,
  before skip=3pt,
  after skip=3pt,
  enhanced,
  drop fuzzy shadow=red!20!black,
  overlay={
    \begin{tcbclipinterior}
      \fill[red!10!white] (interior.south west) rectangle (interior.north east);
    \end{tcbclipinterior}
  }
}

% cpp代码高亮设置
\setminted[cpp]{
    fontsize=\tiny,
    fontfamily=tt,             % 使用等宽字体
    linenos=true,
    frame=lines,               % 上下两条线,简洁清爽(比 tb 更现代)
    framesep=3mm,              % 内边距
    rulecolor=\color{blue!20}, % 线条颜色浅蓝,不刺眼
    bgcolor=blue!10,           % 浅蓝色背景
    baselinestretch=1.2,       % 行距稍大,更易读
    breaklines=true,
    breakautoindent=true,
    tabsize=4,
    xleftmargin=5mm,
    xrightmargin=5mm,
    numbersep=8pt,             % 行号与代码间距
    % ===== 其他美化 =====
    obeytabs=true,             % 尊重 tab 字符
    samepage=false,            % 允许跨页(重要!避免空白)
    escapeinside=||,           % 可在代码中使用 |LaTeX| 插入 LaTeX 命令
}

% 设置标题页颜色,与frame标题保持一致
\setbeamercolor{title}{bg=frametitlebg, fg=ytublue!80!black}
\setbeamercolor{subtitle}{bg=frametitlebg, fg=ytublue!70!black}
\setbeamercolor{author}{bg=frametitlebg, fg=ytublue!80!black}
\setbeamercolor{institute}{bg=frametitlebg, fg=ytublue!80!black}
\setbeamercolor{date}{bg=frametitlebg, fg=ytublue!80!black}

% 自定义标题页样式,全部内容同一个tcolorbox,居中排版,字体和间距区分
\setbeamertemplate{title page}{
  \vbox{}
  \begingroup
    \centering
    \begin{tcolorbox}[
      enhanced,
      width=0.92\paperwidth,
      colback=frametitlebg,
      colframe=frametitlebg,
      boxrule=0pt,
      left=0pt,
      right=0pt,
      top=4mm,
      bottom=4mm,
      boxsep=0pt,
      before skip=0pt,
      after skip=1.2em,
    ]
    % 标题
    {\centering
      {\fontsize{24pt}{27pt}\selectfont\textcolor{ytublue!80!black}{\inserttitle}\par}
      \vspace{1.2em}
      % 副标题
      {\fontsize{21pt}{24pt}\selectfont\textcolor{ytublue!70!black}{\insertsubtitle}\par}
      \vspace{2.0em}
      % 作者
      {\fontsize{12pt}{15pt}\selectfont\insertauthor\par}
      \vspace{0.7em}
      % 单位
      {\fontsize{12pt}{15pt}\selectfont\insertinstitute\par}
      \vspace{0.7em}
      % 日期
      {\fontsize{12pt}{15pt}\selectfont\insertdate\par}
    }
    \vspace{0.5em}
    \vfill
    \end{tcolorbox}
  \endgroup
}


% 文档信息
\title{课后作业}
\subtitle{线性方程组求解器}
\author{王培杰}
\institute{长江大学地球物理与石油资源学院}
\date{\today}

\begin{document}

\begin{frame}{课后任务}
    \begin{ytublock}{线性方程组求解系统}
        \item 设计一个线性方程组求解系统,支持多种求解方法:
        \begin{itemize}
            \item 使用容器类存储矩阵和向量数据
            \item 使用面向对象多态实现不同的求解算法(高斯消元法、LU分解法、雅可比迭代法、SOR迭代法)
            \item 使用Lambda表达式实现自定义的矩阵运算和向量操作
            \item 使用智能指针管理动态分配的矩阵对象
            \item 包含输入验证和异常处理机制(如奇异矩阵检测、收敛性判断)
        \end{itemize}
    \end{ytublock}
\end{frame}

\begin{frame}{线性方程组求解问题简介}
    \begin{ytublock}{问题描述}
        \item 线性方程组求解旨在寻找满足$Ax=b$的未知向量$x$,其中$A$为$n\times n$系数矩阵,$b$为常数向量。
        \item 该问题广泛应用于科学计算、工程仿真、数据分析等领域,是数值计算的基础问题之一。
    \end{ytublock}
    \begin{ytublock}{常用求解方法}
        \begin{itemize}
            \item \textbf{高斯消元法}:直接法,适用于小型稠密矩阵,数值稳定性高。
            \item \textbf{LU分解法}:将$A$分解为上下三角矩阵,便于多次求解不同右端项。
            \item \textbf{雅可比迭代法}:适合大型稀疏矩阵,易于并行实现。
            \item \textbf{SOR迭代法}:在雅可比基础上加速收敛,适用于对角占优矩阵。
        \end{itemize}
    \end{ytublock}
\end{frame}

\begin{frame}{$Ax=b$——线性方程组简介}
        \begin{itemize}
            \item $A$:$n\times n$的系数矩阵,包含所有方程中未知数的系数。每一行对应一个方程,每一列对应一个未知数。例如:
            \[
                A = \begin{bmatrix}
                    a_{11} & a_{12} & \cdots & a_{1n} \\
                    a_{21} & a_{22} & \cdots & a_{2n} \\
                    \vdots & \vdots & \ddots & \vdots \\
                    a_{n1} & a_{n2} & \cdots & a_{nn}
                \end{bmatrix}
            \]
            \begin{columns}
                \begin{column}{0.48\textwidth}
                    \item $x$:$n$维未知向量,表示待求解的变量。形式为:
                    \[
                        x = \begin{bmatrix}
                            x_1 \\ x_2 \\ \vdots \\ x_n
                        \end{bmatrix}
                    \]
                \end{column}
                \begin{column}{0.48\textwidth}
                    \item $b$:$n$维常数向量,表示每个方程右端的常数项。形式为:
                    \[
                        b = \begin{bmatrix}
                            b_1 \\ b_2 \\ \vdots \\ b_n
                        \end{bmatrix}
                    \]
                \end{column}
            \end{columns}
        \end{itemize}
\end{frame}

\begin{frame}{高斯消元法}
    高斯消元法的基本思想是:通过消元操作,将线性方程组$Ax=b$转化为上三角矩阵,然后通过回代求解。高斯消元法的步骤如下:
    \begin{itemize}
        \item \textbf{选主元(可选)}:在每一步消元前,选取当前列绝对值最大的元素作为主元,通过行交换将其移至对角线位置,提高数值稳定性。
        \item \textbf{消元操作}:从第1行开始,依次对每一行,将该行以下的各行对应列的系数消为0。具体做法如下:\\
        对于第$k$步($k=1,2,\ldots,n-1$),对第$k$行以下的每一行$i$,用如下公式更新:
        \[
            a_{ij} = a_{ij} - \frac{a_{ik}}{a_{kk}} a_{kj}, \quad (j=k,k+1,\ldots,n); b_i = b_i - \frac{a_{ik}}{a_{kk}} b_k
        \]
        其中$a_{ik}/a_{kk}$为消元因子。
        \item \textbf{回代求解}:当矩阵变为上三角后,从最后一行开始,依次向上代入已知的未知数,递推求解出所有未知数的值。回代公式为:
        \[
            x_i = \frac{1}{a_{ii}} \left( b_i - \sum_{j=i+1}^{n} a_{ij} x_j \right), \quad (i=n,n-1,\ldots,1)
        \]
    \end{itemize}
\end{frame}

\begin{frame}{LU分解法}
    LU分解法是一种高效求解线性方程组$Ax=b$的方法,其核心思想是将系数矩阵$A$分解为下三角矩阵$L$和上三角矩阵$U$的乘积,即$A=LU$,从而将原问题转化为两个更易求解的三角方程组。具体步骤如下:
    \begin{itemize}
        \item \textbf{LU分解}:将$n\times n$矩阵$A$分解为下三角矩阵$L$(对角线元素为1)和上三角矩阵$U$,满足$A=LU$。常用的分解方法有Doolittle法($L$主对角线为1)和Crout法($U$主对角线为1)。分解过程通常通过消元操作实现,类似高斯消元,但将消元因子存入$L$矩阵。
        \item \textbf{前向替换}:将$Ax=b$转化为$LUx=b$,令$Ux=y$,先解$Ly=b$。由于$L$为下三角矩阵,$Ly=b$可通过前向替换递推求解:
        \[
            y_i = b_i - \sum_{j=1}^{i-1} l_{ij}y_j, \quad (i=1,2,\ldots,n)
        \]
        \item \textbf{后向替换}:已知$y$后,解$Ux=y$。$U$为上三角矩阵,可通过后向替换递推求解$x$:
        \[
            x_i = \frac{1}{u_{ii}}\left(y_i - \sum_{j=i+1}^{n} u_{ij}x_j\right), \quad (i=n,n-1,\ldots,1)
        \]
    \end{itemize}
\end{frame}

\begin{frame}{LU分解法:Doolittle法详解}
        \item \textbf{目标}:找到$L$和$U$,使$A=LU$,其中
        \[
            L = \begin{bmatrix}
                1      & 0      & \cdots & 0 \\
                l_{21} & 1      & \cdots & 0 \\
                \vdots & \vdots & \ddots & \vdots \\
                l_{n1} & l_{n2} & \cdots & 1
            \end{bmatrix},\quad
            U = \begin{bmatrix}
                u_{11} & u_{12} & \cdots & u_{1n} \\
                0      & u_{22} & \cdots & u_{2n} \\
                \vdots & \vdots & \ddots & \vdots \\
                0      & 0      & \cdots & u_{nn}
            \end{bmatrix}
        \]
        \item \textbf{计算公式}:对$k=1,2,\ldots,n$,依次计算$U$和$L$的元素:
        \begin{itemize}
            \item 计算$U$的第$k$行:
            \[
                u_{kj} = a_{kj} - \sum_{s=1}^{k-1} l_{ks}u_{sj},\quad j=k,k+1,\ldots,n
            \]
            \item 计算$L$的第$k$列(主对角线$L_{kk}=1$):
            \[
                l_{ik} = \frac{1}{u_{kk}}\left(a_{ik} - \sum_{s=1}^{k-1} l_{is}u_{sk}\right),\quad i=k+1,k+2,\ldots,n
            \]
        \end{itemize}
        \item \textbf{分解流程}:逐行逐列递推,先算$U$的第$k$行,再算$L$的第$k$列,直到全部元素求出。
\end{frame}

\begin{frame}{雅可比迭代法}
    雅可比迭代法是一种常用的迭代求解线性方程组$Ax=b$的方法。其基本思想是利用矩阵的分解,将每个未知数的计算与其它未知数的当前近似值分离开来,从而逐步逼近精确解。具体过程如下:

    \begin{itemize}
        \item \textbf{矩阵分解}:将系数矩阵$A$分解为对角矩阵$D$和非对角部分$R$,即$A=D+R$,其中$D$为$A$的主对角线元素构成的对角矩阵,$R=A-D$为其余元素组成的矩阵。
        \item \textbf{迭代公式推导}:原方程$Ax=b$可写为$Dx = b - Rx$,从而得到迭代格式:
        \[
            x^{(k+1)} = D^{-1}(b - R x^{(k)})
        \]
        具体到每个分量,有
        \[
            x_i^{(k+1)} = \frac{1}{a_{ii}}\left(b_i - \sum_{j=1,\,j\neq i}^{n} a_{ij} x_j^{(k)}\right),\quad i=1,2,\ldots,n
        \]
        即每次迭代时,第$i$个分量的新值由其余分量的上一次迭代值计算得到。
        \item \textbf{初始值选择}:给定初始向量$x^{(0)}$,通常可取零向量或任意猜测值。
        \item \textbf{收敛判据}:迭代过程中,若$\|x^{(k+1)}-x^{(k)}\|$小于预设精度$\varepsilon$,则认为收敛。
    \end{itemize}
\end{frame}

\begin{frame}{SOR迭代法}
    SOR(Successive Over-Relaxation,超松弛迭代)法是在高斯-赛德尔迭代法基础上引入松弛因子$\omega$的一种加速收敛的迭代方法,常用于求解大型稀疏线性方程组$Ax=b$。

    \begin{itemize}
        \item \textbf{基本思想}:通过引入松弛因子$\omega$,调整每次迭代的步长,从而加快收敛速度。$\omega=1$时,SOR法退化为高斯-赛德尔法;$\omega>1$为超松弛,$\omega<1$为欠松弛。
        \item \textbf{矩阵分解}:将$A$分解为$A=D+L+U$,其中$D$为对角矩阵,$L$为严格下三角部分,$U$为严格上三角部分。
        \item \textbf{迭代公式推导}:原方程$Ax=b$可写为
        \[
            (D+L)x^{(k+1)} = b - Ux^{(k)}
        \]
        SOR法的迭代格式为
        \[
            x^{(k+1)} = (1-\omega)x^{(k)} + \omega(D+L)^{-1}(b - Ux^{(k)})
        \]
        具体到每个分量,第$i$个分量的迭代公式为
        \[
            x_i^{(k+1)} = (1-\omega)x_i^{(k)} + \frac{\omega}{a_{ii}}\left(b_i - \sum_{j=1}^{i-1} a_{ij}x_j^{(k+1)} - \sum_{j=i+1}^{n} a_{ij}x_j^{(k)}\right)
        \]
        其中$\omega$为松弛因子,$0<\omega<2$。
    \end{itemize}
\end{frame}

\begin{frame}{线性方程组求解系统设计}
    \begin{ytublock}{系统架构}
        \begin{itemize}
            \item \textbf{矩阵与向量类}:基于\texttt{std::vector}实现高效存储与运算,支持Lambda表达式优化常用操作。
            \item \textbf{求解器基类(SolverBase)}:定义统一的虚接口,便于扩展多种算法,体现面向对象多态。
            \item \textbf{具体求解器实现}:
            \begin{itemize}
                \item \textbf{GaussianSolver}:高斯消元法,支持部分选主元与行交换优化。
                \item \textbf{LUSolver}:Doolittle LU分解,前向/后向替代高效实现。
                \item \textbf{JacobiSolver}:雅可比迭代,内置收敛性检查。
                \item \textbf{SORSolver}:SOR加速迭代,支持超松弛参数自定义。
            \end{itemize}
            \item \textbf{系统管理类(LinearSystem)}:封装求解流程,负责输入验证、奇异性检测、精度分析与异常处理。
            \item \textbf{智能指针与RAII}:采用\texttt{std::unique\_ptr}管理动态对象,保证内存安全与异常安全。
        \end{itemize}
    \end{ytublock}
\end{frame}

\begin{frame}{开发步骤}
    \begin{itemize}
        \item \textbf{1. 设计系统架构}:确定矩阵与向量类、求解器基类和具体求解器实现。
        \item \textbf{2. 实现矩阵与向量类}:基于\texttt{std::vector}实现高效存储与运算,支持Lambda表达式优化常用操作。
        \item \textbf{3. 实现求解器基类}:定义统一的虚接口,便于扩展多种算法,体现面向对象多态。
        \item \textbf{4. 实现具体求解器}:实现高斯消元法、LU分解法、雅可比迭代法、SOR迭代法。
        \item \textbf{5. 实现系统管理类}:封装求解流程,负责输入验证、奇异性检测、精度分析与异常处理。
    \end{itemize}
\end{frame}

\begin{frame}{Vector类}
    \begin{ytublock}{向量类设计}
        \item 基于\texttt{std::vector}实现的一维数值向量类
        \item 支持Lambda表达式优化常用运算
        \item 提供完整的向量代数操作接口
    \end{ytublock}

    \begin{columns}
        \column{0.48\textwidth}
        \begin{block}{核心特性}
            \begin{itemize}
                \item \textbf{存储结构}:\texttt{std::vector<double>}连续存储
                \item \textbf{构造函数}:默认、数据、移动构造
                \item \textbf{元素访问}:带边界检查的\texttt{at()}方法
                \item \textbf{标量运算}:\texttt{multiply()}方法
                \item \textbf{范数计算}:\texttt{norm()}方法(2-范数)
                \item \textbf{输出}:格式化的控制台打印
            \end{itemize}
        \end{block}

        \column{0.48\textwidth}
        \begin{block}{Lambda优化}
            \begin{itemize}
                \item 标量乘法:\texttt{std::transform}
                \item 范数计算:\texttt{std::accumulate}
                \item 打印输出:\texttt{std::for\_each}
                \item 性能提升:编译时内联优化
            \end{itemize}
        \end{block}
    \end{columns}
\end{frame}

\begin{frame}{知识点:构造函数}
    \begin{ytublock}{构造函数}
        \item 构造函数是类的一种特殊成员函数,用于在创建对象时初始化对象。
        \item 构造函数可以有参数,也可以没有参数。
        \item 构造函数可以有返回值,也可以没有返回值。
        \item 构造函数可以有多个,也可以没有多个。
        \item \textbf{默认构造函数}:没有参数的构造函数,用于创建对象时自动调用。
        \item \textbf{拷贝构造函数}:用于创建对象时复制另一个对象的值。
        \item \textbf{移动构造函数}:用于创建对象时移动另一个对象的值。
    \end{ytublock}
\end{frame}

\begin{frame}{知识点:C++异常}
    \begin{ytublock}{C++异常处理机制}
        \item C++异常处理用于在程序运行时捕获和处理错误,提升程序健壮性。
        \item \textbf{throw}:抛出异常对象,可为内置类型或自定义类型。
        \item \textbf{try-catch}:\texttt{try}块包裹可能出错的代码,\texttt{catch}块捕获并处理异常。
        \item \textbf{noexcept}:用于声明函数不会抛出异常,便于编译器优化。
        \item 支持异常的层级捕获与重新抛出,便于精细化错误处理。
        \item 建议自定义异常类型继承自\texttt{std::exception},便于统一管理。
    \end{ytublock}
    \begin{ytublock}
        \item std::out_of_range:用于表示索引超出范围的异常。
        \item std::invalid_argument:用于表示无效的参数的异常。
        \item std::runtime_error:用于表示运行时异常。
        \item std::logic_error:用于表示逻辑异常。
        \item std::exception:用于表示所有异常的基类。
        \item std::exception\_ptr:用于表示异常的指针。
    \end{ytublock}
\end{frame}

\begin{frame}{知识点:std::transform + Lambda表达式优化}
    \begin{ytublock}{std::transform详解}
        \item \texttt{std::transform} 是C++标准库中的泛型算法,用于将一个或多个输入区间的元素通过指定的操作变换后,输出到目标区间。
        \item 常见用法为对容器(如\texttt{std::vector})中的每个元素应用某个函数或Lambda表达式,实现批量数据处理。
        \item 语法示例:\texttt{std::transform(first, last, d\_first, unary\_op);},其中\texttt{unary\_op}可以是函数指针、函数对象或Lambda表达式。
        \item 也支持二元操作:\texttt{std::transform(first1, last1, first2, d\_first, binary\_op);},可用于两个容器的元素逐一运算。
        \item 结合Lambda表达式,可实现如标量乘法、元素映射、批量转换等高效操作。例如:\texttt{std::transform(v.begin(), v.end(), v2.begin(), [](double x){ return x * 2; });}
        \item \texttt{std::transform} 通常比手写循环更简洁且易于编译器优化,推荐在需要元素级变换时优先使用。
    \end{ytublock}
\end{frame}

\begin{frame}{知识点:std::move}
    \begin{ytublock}{std::move详解}
        \item \texttt{std::move} 是C++标准库中的一个函数,用于将一个对象转换为右值引用,从而可以被移动构造或赋值。
        \item 常见用法为将一个对象转换为右值引用,从而可以被移动构造或赋值。
        \item 语法示例:\texttt{std::move(x);},其中\texttt{x}是一个对象。
        \item 也支持二元操作:\texttt{std::move(x, y);},可用于两个对象的移动构造或赋值。
    \end{ytublock}
    \begin{ytublock}{右值引用}
        \item 右值引用是C++11引入的一种新的引用类型,用于表示临时对象或右值。
        \item 常见用法为将一个对象转换为右值引用,从而可以被移动构造或赋值。
        \item 语法示例:\texttt{int&& x = 1;},其中\texttt{x}是一个右值引用。
    \end{ytublock}
\end{frame}

\begin{frame}{知识点:&&}
    \begin{ytublock}{&&}
        \item &&是C++11引入的一种新的引用类型,用于表示右值引用。
        \item 常见用法为将一个对象转换为右值引用,从而可以被移动构造或赋值。
        \item 语法示例:\texttt{int&& x = 1;},其中\texttt{x}是一个右值引用。
    \end{ytublock}
\end{frame}

\begin{frame}{知识点:std::accumulate}
    \begin{ytublock}{std::accumulate}
        \item \texttt{std::accumulate} 是C++标准库中的泛型算法,用于对区间内元素进行累加、累乘或自定义聚合操作。
        \item 常见用法包括:求和、求积、字符串拼接、结构体成员累加等。
        \item 基本语法:\texttt{std::accumulate(first, last, init);},其中\texttt{first}和\texttt{last}为迭代器,\texttt{init}为初始值。
        \item 支持自定义二元操作:\texttt{std::accumulate(first, last, init, binary\_op);},可实现如乘积、最大值、字符串拼接等多种聚合。
        \item 结合Lambda表达式可灵活实现复杂聚合逻辑。例如:\texttt{std::accumulate(v.begin(), v.end(), 0.0, [](double a, double b)\{ return a + b * b; \});}
        \item 相比手写循环,\texttt{std::accumulate} 代码更简洁、可读性更高,易于编译器优化。
    \end{ytublock}
\end{frame}

\begin{frame}{知识点:std::for\_each + Lambda表达式}
    \begin{ytublock}{std::for\_each优化用法}
        \item \texttt{std::for\_each} 是C++标准库中的泛型算法,用于对区间内每个元素执行指定操作,常用于遍历和批量处理。
        \item 推荐结合Lambda表达式使用,可实现就地处理、捕获外部变量、简化代码。例如:\texttt{std::for\_each(v.begin(), v.end(), [](int \&x)\{ x *= 2; \});}
        \item 语法:\texttt{std::for\_each(first, last, func);},其中\texttt{func}可为函数指针、函数对象或Lambda表达式。
        \item 支持对任意容器(如\texttt{std::vector}、\texttt{std::list}等)进行高效遍历和操作,代码更简洁、可读性更高。
        \item 与传统for循环相比,\texttt{std::for\_each}更易于并行化和编译器优化,适合大规模数据处理场景。
    \end{ytublock}
\end{frame}

\begin{frame}{Matrix类}
    \begin{ytublock}{矩阵类设计}
        \item 基于\texttt{std::vector<std::vector<double>>}实现的二维数值矩阵类
        \item 支持Lambda表达式优化矩阵运算
        \item 提供完整的矩阵代数操作接口
    \end{ytublock}

    \begin{columns}
        \column{0.48\textwidth}
        \begin{block}{核心特性}
            \begin{itemize}
                \item \textbf{存储结构}:二维向量,行优先存储
                \item \textbf{构造函数}:默认、数据、移动构造
                \item \textbf{元素访问}:带边界检查的\texttt{at()}方法
                \item \textbf{矩阵运算}:\texttt{multiply()}矩阵-向量乘法
                \item \textbf{奇异性检测}:基于行列式的精确判断
                \item \textbf{输出}:格式化的控制台打印
            \end{itemize}
        \end{block}

        \column{0.48\textwidth}
        \begin{block}{Lambda优化}
            \begin{itemize}
                \item 矩阵-向量乘:\texttt{std::inner\_product}
                \item 打印输出:嵌套\texttt{std::for\_each}
                \item 行列式计算:递归算法优化
                \item 对角元素检查:简化的奇异性判断
                \item 性能提升:编译时内联和常量传播
            \end{itemize}
        \end{block}
    \end{columns}
\end{frame}

\begin{frame}{知识点:std::inner\_product}
    \begin{ytublock}{std::inner\_product优化用法}
        \item \texttt{std::inner\_product} 是C++标准库中的泛型算法,常用于高效计算两个区间(如向量)元素的内积,广泛应用于矩阵运算、信号处理等领域。
        \item 基本用法:\texttt{std::inner\_product(first1, last1, first2, init);},其中\texttt{first1}和\texttt{last1}为第一个区间的迭代器,\texttt{first2}为第二个区间的起始迭代器,\texttt{init}为初始值。
        \item 支持自定义二元操作和累加操作:\texttt{std::inner\_product(first1, last1, first2, init, binary\_op1, binary\_op2);},可实现如加权和、逻辑运算等复杂聚合。
        \item 推荐结合Lambda表达式灵活定制内积逻辑。例如:\texttt{std::inner\_product(a.begin(), a.end(), b.begin(), 0.0, std::plus<>(), [](double x, double y)\{ return x*y; \});}
        \item 相比手写循环,\texttt{std::inner\_product} 代码更简洁、可读性更高,易于编译器优化和并行化。
    \end{ytublock}
\end{frame}

\begin{frame}{SolverBase求解器基类}
    \begin{ytublock}{抽象基类设计}
        \item 采用面向对象多态设计,实现算法抽象
        \item 定义统一的求解接口,便于扩展新算法
        \item 使用智能指针管理内存,异常安全
    \end{ytublock}

    \begin{columns}
        \column{0.48\textwidth}
        \begin{block}{设计模式}
            \begin{itemize}
                \item \textbf{策略模式}:不同算法的封装和替换
                \item \textbf{工厂模式}:动态创建具体求解器实例
                \item \textbf{模板方法}:定义求解流程的骨架
                \item \textbf{RAII模式}:智能指针自动资源管理
            \end{itemize}
        \end{block}

        \column{0.48\textwidth}
        \begin{block}{核心接口}
            \begin{itemize}
                \item \textbf{纯虚函数}:\texttt{solve()}统一求解接口
                \item \textbf{智能指针}:\texttt{std::unique\_ptr<Vector>}返回值
                \item \textbf{异常处理}:返回nullptr表示求解失败
                \item \textbf{const正确性}:不修改输入参数
            \end{itemize}
        \end{block}
    \end{columns}
\end{frame}

\begin{frame}{GaussianSolver高斯消元法}
    \begin{ytublock}{具体实现}
        \item 继承SolverBase基类,实现高斯消元算法
        \item 支持部分选主元策略,提高数值稳定性
        \item 使用Lambda表达式优化行交换操作
    \end{ytublock}

    \begin{columns}
        \column{0.48\textwidth}
        \begin{block}{算法流程}
            \begin{itemize}
                \item \textbf{前向消元}:逐列消元,将矩阵化为上三角
                \item \textbf{部分选主元}:选择绝对值最大的主元
                \item \textbf{行交换}:使用\texttt{std::swap\_ranges}优化
                \item \textbf{回代求解}:从最后一列向上递推求解
                \item \textbf{奇异性检测}:检查主元是否接近零
            \end{itemize}
        \end{block}

        \column{0.48\textwidth}
        \begin{block}{Lambda优化}
            \begin{itemize}
                \item 行交换:\texttt{std::swap\_ranges}替代循环
                \item 边界检查:编译时类型安全
                \item 内存布局:连续存储保证缓存友好
                \item 异常处理:返回nullptr表示失败
            \end{itemize}
        \end{block}
    \end{columns}
\end{frame}

\begin{frame}{知识点:std::unique_ptr + std::make_unique}
    \begin{ytublock}{std::unique_ptr}
        \item \texttt{std::unique\_ptr} 是C++标准库中的一个智能指针,用于管理动态分配的内存。
        \item 常见用法为将一个对象转换为右值引用,从而可以被移动构造或赋值。
        \item 语法示例:\texttt{std::unique\_ptr<int> x = std::make\_unique<int>(1);},其中\texttt{x}是一个智能指针。
    \end{ytublock}
    \begin{ytublock}{std::make_unique}
        \item \texttt{std::make\_unique} 是C++标准库中的一个函数,用于创建一个智能指针。
        \item 常见用法为创建一个智能指针。
        \item 语法示例:\texttt{std::unique\_ptr<int> x = std::make\_unique<int>(1);},其中\texttt{x}是一个智能指针。
    \end{ytublock}
\end{frame}

\begin{frame}{知识点:std::swap\_ranges}
    \begin{ytublock}{std::swap 简介}
        \item \texttt{std::swap} 是C++标准库中用于高效交换两个对象内容的函数,简化代码并提升性能。
        \item 常见场景:变量交换、容器元素交换等,适用于所有支持交换操作的类型。
        \item 示例:\texttt{std::swap(a, b);} 可直接交换 \texttt{a} 和 \texttt{b} 的值。
        \item 注意:参与交换的对象需支持交换操作,否则结果未定义。
    \end{ytublock}
    \begin{ytublock}{std::swap\_ranges 优化与应用}
        \item \texttt{std::swap\_ranges} 可高效地一次性交换两个区间的全部元素,避免手动循环,提升代码可读性与执行效率。
        \item 典型应用:矩阵行交换、批量数据重排等,适用于如 \texttt{std::vector}、原生数组等支持随机访问迭代器的容器。
        \item 示例:\texttt{std::swap\_ranges(row1.begin(), row1.end(), row2.begin());} 可高效交换两行数据。
        \item 注意:两个区间长度必须一致且不能重叠,否则行为未定义。
        \item 优化建议:结合Lambda表达式与类型安全检查,可进一步提升代码健壮性和灵活性。
    \end{ytublock}
\end{frame}

\begin{frame}{LUSolver LU分解法}
    \begin{ytublock}{具体实现}
        \item 继承SolverBase基类,实现Doolittle LU分解算法
        \item 支持多次求解不同右端项,提高效率
        \item 使用Lambda表达式优化前向/后向替代
    \end{ytublock}

    \begin{columns}
        \column{0.48\textwidth}
        \begin{block}{算法流程}
            \begin{itemize}
                \item \textbf{LU分解}:将A分解为L(下三角)和U(上三角)
                \item \textbf{Doolittle方法}:L的对角线元素为1
                \item \textbf{前向替代}:解Ly = b得到中间向量y
                \item \textbf{后向替代}:解Ux = y得到最终解x
                \item \textbf{效率优势}:适合多次求解相同系数矩阵
            \end{itemize}
        \end{block}

        \column{0.48\textwidth}
        \begin{block}{Lambda优化}
            \begin{itemize}
                \item 矩阵分解:\texttt{std::accumulate}优化求和
                \item 前向替代:Lambda表达式简化循环
                \item 后向替代:Lambda表达式优化计算
                \item 内存效率:避免重复矩阵拷贝
            \end{itemize}
        \end{block}
    \end{columns}
\end{frame}

\begin{frame}{JacobiSolver雅可比迭代法}
    \begin{ytublock}{具体实现}
        \item 继承SolverBase基类,实现雅可比迭代算法
        \item 支持自定义收敛容差和最大迭代次数
        \item 使用Lambda表达式优化收敛性检查
    \end{ytublock}

    \begin{columns}
        \column{0.48\textwidth}
        \begin{block}{算法流程}
            \begin{itemize}
                \item \textbf{矩阵分解}:$A = D + R$(D为对角矩阵)
                \item \textbf{迭代公式}:$x^{(k+1)} = D^{-1}(b - R x^{(k)})$
                \item \textbf{分量计算}:每个分量独立计算
                \item \textbf{收敛检查}:检查前后两次迭代的差值
                \item \textbf{适用条件}:严格对角占优矩阵
            \end{itemize}
        \end{block}

        \column{0.48\textwidth}
        \begin{block}{Lambda优化}
            \begin{itemize}
                \item 收敛检查:\texttt{std::accumulate}优化误差计算
                \item 迭代过程:Lambda表达式简化循环
                \item 边界检查:编译时类型安全
                \item 内存效率:避免临时变量创建
            \end{itemize}
        \end{block}
    \end{columns}
\end{frame}

\begin{frame}{SORSolver SOR迭代法}
    \begin{ytublock}{具体实现}
        \item 继承SolverBase基类,实现SOR加速迭代算法
        \item 支持自定义松弛因子、收敛容差和最大迭代次数
        \item 使用Lambda表达式优化收敛性检查和迭代过程
    \end{ytublock}

    \begin{columns}
        \column{0.48\textwidth}
        \begin{block}{算法流程}
            \begin{itemize}
                \item \textbf{矩阵分解}:$A = D + L + U$(D对角,L严格下三角,U严格上三角)
                \item \textbf{SOR公式}:$x^{(k+1)} = (1-\omega)x^{(k)} + \omega(D+L)^{-1}(b - U x^{(k)})$
                \item \textbf{松弛因子}:$\omega$控制收敛速度($0 < \omega < 2$)
                \item \textbf{收敛加速}:比雅可比方法收敛更快
                \item \textbf{适用条件}:对角占优矩阵
            \end{itemize}
        \end{block}

        \column{0.48\textwidth}
        \begin{block}{Lambda优化}
            \begin{itemize}
                \item 收敛检查:\texttt{std::accumulate}优化误差计算
                \item 迭代计算:Lambda表达式简化求和
                \item 超松弛:Lambda表达式优化参数应用
                \item 内存效率:避免临时向量创建
            \end{itemize}
        \end{block}
    \end{columns}
\end{frame}

\begin{frame}{LinearSystem系统管理类}
    \begin{ytublock}{门面模式设计}
        \item 封装复杂的求解系统,提供统一接口
        \item 门面模式隐藏子系统的复杂性
        \item 提供完整的求解流程和验证功能
    \end{ytublock}

    \begin{columns}
        \column{0.48\textwidth}
        \begin{block}{核心功能}
            \begin{itemize}
                \item \textbf{系统封装}:封装矩阵A和向量b
                \item \textbf{统一求解}:提供统一的solve接口
                \item \textbf{解管理}:存储精确解和数值解
                \item \textbf{精度验证}:残差计算和误差分析
                \item \textbf{准确性验证}:解的准确性检查
                \item \textbf{报告生成}:完整的精度分析报告
            \end{itemize}
        \end{block}

        \column{0.48\textwidth}
        \begin{block}{Lambda优化}
            \begin{itemize}
                \item 残差计算:\texttt{std::transform}优化
                \item 误差计算:\texttt{std::transform}优化
                \item 数值验证:Lambda表达式简化计算
                \item 报告生成:Lambda表达式格式化输出
            \end{itemize}
        \end{block}
    \end{columns}
\end{frame}

\begin{frame}{主程序main.cpp}
    \begin{ytublock}{系统集成与演示}
        \item 完整的系统集成和算法测试演示
        \item 展示所有组件的协同工作
        \item 验证系统的正确性和性能
    \end{ytublock}

    \begin{columns}
        \column{0.48\textwidth}
        \begin{block}{程序流程}
            \begin{itemize}
                \item \textbf{编码设置}:UTF-8中文显示支持
                \item \textbf{测试矩阵}:创建5×5对角占优系统
                \item \textbf{求解器创建}:实例化四种求解器
                \item \textbf{算法测试}:逐个测试求解器
                \item \textbf{精度验证}:残差计算和误差分析
                \item \textbf{报告生成}:完整的精度分析报告
            \end{itemize}
        \end{block}

        \column{0.48\textwidth}
        \begin{block}{测试策略}
            \begin{itemize}
                \item 5×5对角占优矩阵测试
                \item 四种算法全面验证
                \item 数值精度对比分析
                \item 收敛性检查验证
                \item 中文界面友好展示
            \end{itemize}
        \end{block}
    \end{columns}
\end{frame}

\end{document}