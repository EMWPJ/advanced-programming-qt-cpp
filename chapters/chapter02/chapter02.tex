\documentclass[UTF8,aspectratio=169]{beamer}



% 基本包
\usepackage[UTF8]{ctex}
\usepackage{graphicx}
\usepackage{amsmath}
\usepackage{amsfonts}
\usepackage{amssymb}
% \usepackage{listings}  % 已替换为minted
\usepackage{xcolor}
\usepackage{hyperref}
\usepackage{booktabs}
\usepackage{multirow}
\usepackage{multicol}
\usepackage{float}
\usepackage{tikz}
\usetikzlibrary{positioning,shapes,arrows,fit,backgrounds}
\usepackage{pgfplots}
\pgfplotsset{compat=1.18}
\usepackage{minted}
\usepackage{fontspec}
\usepackage[most]{tcolorbox}

% Beamer主题设置
\usetheme{Madrid}
\usecolortheme{whale}

% 校徽设置
\logo{
  \IfFileExists{../长江大学校徽.pdf}{
    \begin{tikzpicture}[remember picture,overlay]
      \node[anchor=north east, xshift=-0.2mm, yshift=-0.2mm] at (current page.north east) {
        \includegraphics[height=1.0cm]{../长江大学校徽.pdf}
      };
    \end{tikzpicture}
  }{
    \begin{tikzpicture}[remember picture,overlay]
      \node[anchor=north east, xshift=-3mm, yshift=-3mm] at (current page.north east) {
        \textcolor{red}{\tiny [校徽文件未找到]}
      };
    \end{tikzpicture}
  }
}

% ===== 使用推荐的 font themes =====
\usefonttheme{professionalfonts}  % 允许自定义字体

% 自定义frame标题栏,也缩短长度留出logo空间
% 设置标题栏背景颜色为淡蓝色
\definecolor{frametitlebg}{RGB}{200,215,250} % 淡蓝色,可根据需要调整
\setbeamercolor{frametitle}{bg=frametitlebg, fg=ytublue!80!black}
\setbeamertemplate{frametitle}{
  \ifbeamercolorempty[bg]{frametitle}{}{\nointerlineskip}%
  \begin{tcolorbox}[
    enhanced,
    width=0.90\paperwidth,
    height=2.5ex,
    colback=frametitlebg,
    colframe=frametitlebg,
    boxrule=0pt,
    left=0pt,
    right=0pt,
    top=1pt,
    bottom=0pt,
    boxsep=0pt,
    before skip=0pt,
    after skip=0.1em,  % 减少标题栏和内容之间的间距
  ]
  \vspace{0.2ex} % 减少标题文字上方的空白
  \usebeamerfont{frametitle}\textcolor{ytublue!80!black}{\hspace{1em}\insertframetitle}
  \end{tcolorbox}
}

% ===== 设置现代字体 =====
\setsansfont{Source Sans Pro}     % 正文字体
\setmonofont{Source Code Pro}[Scale=0.9]  % 代码字体,稍微缩小一点

\setbeamertemplate{navigation symbols}{}

% 减少页面间距
\setbeamertemplate{itemize items}[circle]
\setbeamertemplate{enumerate items}[default]
\setlength{\itemsep}{0.1em}
\setlength{\parskip}{0.1em}

% 页码设置
\setbeamertemplate{footline}[frame number]

% 定义流程图样式
\tikzset{
    block/.style = {rectangle, draw, fill=blue!10,
        minimum width=6em, align=center, rounded corners, minimum height=3em},
    line/.style = {draw, -latex'}
}
% 水印设置
\setbeamertemplate{background}{
    \begin{tikzpicture}[remember picture,overlay]
        \node[rotate=-45,scale=0.8,opacity=0.1,color=gray]
             at ([xshift=0.5cm,yshift=0.5cm]current page.south west)
             {\large\textbf{WPJ}};
    \end{tikzpicture}
}

% 自定义颜色
\definecolor{qtgreen}{RGB}{41,128,185}
\definecolor{qtblue}{RGB}{52,73,94}

% 定义长江大学蓝主色调
\definecolor{ytublue}{RGB}{0,84,159}
% 统一block样式
\newtcolorbox{ytublock}[1]{
  colback=white,
  colframe=ytublue!80!black,
  colbacktitle=ytublue!20!white,
  coltitle=ytublue!80!black,
  title={#1},
  fonttitle=\bfseries,
  arc=3mm,
  boxrule=1pt,
  boxsep=1mm,
  left=2mm,
  right=2mm,
  top=0.5mm,
  bottom=0.5mm,
  before skip=3pt,
  after skip=3pt,
  enhanced,
  drop fuzzy shadow=ytublue!20!black
}

% 定义警告块样式
\newtcolorbox{ytualertblock}[1]{
  colback=white,
  colframe=red!80!black,
  colbacktitle=red!20!white,
  coltitle=red!80!black,
  title={#1},
  fonttitle=\bfseries,
  arc=3mm,
  boxrule=1.5pt,
  boxsep=1mm,
  left=2mm,
  right=2mm,
  top=0.5mm,
  bottom=0.5mm,
  before skip=3pt,
  after skip=3pt,
  enhanced,
  drop fuzzy shadow=red!20!black,
  overlay={
    \begin{tcbclipinterior}
      \fill[red!10!white] (interior.south west) rectangle (interior.north east);
    \end{tcbclipinterior}
  }
}

% cpp代码高亮设置
\setminted[cpp]{
    fontsize=\tiny,
    fontfamily=tt,             % 使用等宽字体
    linenos=true,
    frame=lines,               % 上下两条线,简洁清爽(比 tb 更现代)
    framesep=3mm,              % 内边距
    rulecolor=\color{blue!20}, % 线条颜色浅蓝,不刺眼
    bgcolor=blue!10,           % 浅蓝色背景
    baselinestretch=1.2,       % 行距稍大,更易读
    breaklines=true,
    breakautoindent=true,
    tabsize=4,
    xleftmargin=5mm,
    xrightmargin=5mm,
    numbersep=8pt,             % 行号与代码间距
    % ===== 其他美化 =====
    obeytabs=true,             % 尊重 tab 字符
    samepage=false,            % 允许跨页(重要!避免空白)
    escapeinside=||,           % 可在代码中使用 |LaTeX| 插入 LaTeX 命令
}

% 设置标题页颜色,与frame标题保持一致
\setbeamercolor{title}{bg=frametitlebg, fg=ytublue!80!black}
\setbeamercolor{subtitle}{bg=frametitlebg, fg=ytublue!70!black}
\setbeamercolor{author}{bg=frametitlebg, fg=ytublue!80!black}
\setbeamercolor{institute}{bg=frametitlebg, fg=ytublue!80!black}
\setbeamercolor{date}{bg=frametitlebg, fg=ytublue!80!black}

% 自定义标题页样式,全部内容同一个tcolorbox,居中排版,字体和间距区分
\setbeamertemplate{title page}{
  \vbox{}
  \begingroup
    \centering
    \begin{tcolorbox}[
      enhanced,
      width=0.92\paperwidth,
      colback=frametitlebg,
      colframe=frametitlebg,
      boxrule=0pt,
      left=0pt,
      right=0pt,
      top=4mm,
      bottom=4mm,
      boxsep=0pt,
      before skip=0pt,
      after skip=1.2em,
    ]
    % 标题
    {\centering
      {\fontsize{24pt}{27pt}\selectfont\textcolor{ytublue!80!black}{\inserttitle}\par}
      \vspace{1.2em}
      % 副标题
      {\fontsize{21pt}{24pt}\selectfont\textcolor{ytublue!70!black}{\insertsubtitle}\par}
      \vspace{2.0em}
      % 作者
      {\fontsize{12pt}{15pt}\selectfont\insertauthor\par}
      \vspace{0.7em}
      % 单位
      {\fontsize{12pt}{15pt}\selectfont\insertinstitute\par}
      \vspace{0.7em}
      % 日期
      {\fontsize{12pt}{15pt}\selectfont\insertdate\par}
    }
    \vspace{0.5em}
    \vfill
    \end{tcolorbox}
  \endgroup
}


% 文档信息
\title{高等程序设计 - Qt/C++}
\subtitle{第2章:C++语言基础与进阶}
\author{王培杰}
\institute{长江大学地球物理与石油资源学院}
\date{\today}

\begin{document}

% 标题页
\begin{frame}
    \titlepage
\end{frame}

% 目录页
\begin{frame}{目录}
    \begin{multicols}{2}
        \tableofcontents[]
    \end{multicols}
\end{frame}

\section{C、C++ 与 Qt}
\begin{frame}{目录}
    \begin{multicols}{2}
        \tableofcontents[currentsection]
    \end{multicols}
\end{frame}

\begin{frame}{C、C++ 与 Qt 关系}
    \centering
    \begin{tikzpicture}[
        font=\sffamily,
        box/.style={rectangle, draw=black!70, rounded corners=5pt, minimum width=3.5cm, minimum height=1.2cm, align=center, fill=##1!15, text=black},
        core/.style={ellipse, draw=black!80, thick, minimum width=4.2cm, minimum height=2.2cm, align=center, fill=red!10, text=black},
        feat/.style={rectangle, draw=gray!50, rounded corners=3pt, fill=gray!10, minimum width=3.2cm, minimum height=1cm, align=center, text=black},
        arrow/.style={-{Latex[length=3mm]}, thick, black!70},
        node distance=0.5cm and 0.8cm
    ]

    % Main nodes
    \node[core] (Qt) {Qt 框架\\\small 基于 C++ 的跨平台开发工具};
    \node[box=green!60, above=of Qt] (Cpp) {C++ 语言\\\small 面向对象、泛型、继承C};
    \node[box, above=of Cpp] (C) {C 语言\\\small 高效、过程式、接近硬件};

    % Arrows for evolution
    \draw[arrow] (C) -- (Cpp);
    \draw[arrow] (Cpp) -- (Qt);

    % Qt features
    \node[feat, below left=0.5cm and 0.8cm of Qt] (gui) {图形界面 (GUI)};
    \node[feat, below=of Qt] (signal) {信号槽机制};
    \node[feat, below right=0.5cm and 0.8cm of Qt] (platform) {跨平台支持};
    \node[feat, above left=0.5cm and 0.8cm of Qt] (tools) {模块化开发工具};
    \node[feat, above right=0.5cm and 0.8cm of Qt] (multilanguage) {多语言支持};

    % Arrows from Qt to features
    \draw[arrow] (Qt) -- (gui);
    \draw[arrow] (Qt) -- (signal);
    \draw[arrow] (Qt) -- (platform);
    \draw[arrow] (Qt) -- (tools);
    \draw[arrow] (Qt) -- (multilanguage);

    % Optional background grouping
    \begin{scope}[on background layer]
        \node[draw=gray!30, rounded corners=12pt, thick, fit=(C)(Cpp)(Qt)(gui)(signal)(platform)(tools), inner sep=0.8cm] {};
    \end{scope}

    \end{tikzpicture}
\end{frame}

\begin{frame}{C、C++ 与 Qt 关系}
        \begin{itemize}
            \item \textbf{C语言}:
            \begin{itemize}
                \item 诞生于20世纪70年代,是一种结构化、过程式的高级编程语言。
                \item 以高效、接近底层硬件著称,广泛应用于操作系统、嵌入式开发、驱动程序等领域。
                \item C语言为C++提供了坚实的基础,C++完全兼容C语言,是C的超集。
            \end{itemize}
            \item \textbf{C++}:
            \begin{itemize}
                \item 由Bjarne Stroustrup于20世纪80年代初开发,在C语言基础上引入了面向对象编程(OOP)、泛型编程等特性。
                \item 支持类、继承、多态、模板等高级特性,适合开发大型复杂系统。
                \item C++不仅能编写高效底层代码,还能实现高层次的抽象,兼容C语言代码,便于项目迁移和扩展。
            \end{itemize}
            \item \textbf{Qt}:
            \begin{itemize}
                \item Qt是一个基于C++的跨平台应用程序开发框架,现由Qt Company维护。
                \item 提供了丰富的GUI(图形用户界面)组件、信号槽机制、网络、数据库、多媒体等模块,极大提升了C++开发效率。
                \item 支持Windows、Linux、macOS、Android等多平台,代码可移植性强。
                \item Qt不仅适用于桌面应用开发,也广泛应用于嵌入式系统、移动端等领域,是C++开发GUI和跨平台应用的首选框架之一。
            \end{itemize}
        \end{itemize}
\end{frame}

\begin{frame}{C、C++ 与 Qt 学习路线(正常版)}
    \begin{itemize}
        \item \textbf{第一阶段:C语言基础}
        \begin{itemize}
            \item 学习C语言的基本语法、数据类型、运算符、流程控制(顺序、选择、循环)。
            \item 掌握函数、数组、指针、结构体、文件操作等核心内容。
            \item 了解内存管理、编译与调试基础,为后续学习打下坚实基础。
        \end{itemize}
        \item \textbf{第二阶段:C++进阶}
        \begin{itemize}
            \item 在C语言基础上,学习C++的面向对象编程(OOP)思想,包括类、对象、继承、多态、封装等。
            \item 掌握C++的模板、STL(标准模板库)、异常处理、运算符重载等高级特性。
            \item 了解C++11/14/17等新标准的常用特性,为现代C++开发做准备。
        \end{itemize}
        \item \textbf{第三阶段:Qt开发实践}
        \begin{itemize}
            \item 学习Qt的基本概念、开发环境搭建、项目结构。
            \item 掌握Qt的信号与槽机制、常用控件、布局管理、事件处理等GUI开发基础。
            \item 进阶学习Qt的多线程、网络编程、数据库、多媒体、绘图等模块。
            \item 了解Qt的跨平台特性,能够在不同操作系统下进行开发与部署。
            \item 通过实际项目练习,提升综合开发能力。
        \end{itemize}
    \end{itemize}
\end{frame}

\begin{frame}{C、C++ 与 Qt 学习路线(非常规路径)}
    \begin{ytublock}{直接上手Qt}
        \begin{itemize}
            \item 跳过传统C/C++基础,直接从Qt入门,先体验图形界面开发的乐趣。
            \item 以实际项目驱动学习,边做边学,遇到C++语法和概念时再查阅补充。
            \item 重点掌握Qt的信号与槽、常用控件、布局管理、事件处理等GUI开发核心。
            \item 随着项目深入,逐步接触Qt的多线程、网络、数据库、多媒体、绘图等高级模块。
            \item 利用Qt的跨平台特性,尝试在不同操作系统下编译和运行项目,提升实战能力。
        \end{itemize}
    \end{ytublock}
    \begin{ytublock}{以Qt为载体学习C++}
        \begin{itemize}
            \item 以Qt项目开发为主线,带动C++语法、面向对象、模板等知识的学习。
            \item 在实际编码中理解类、继承、多态、信号槽等C++与Qt结合的用法。
            \item 通过分析和修改Qt源码或示例,深入理解C++的高级特性和Qt的设计思想。
            \item 结合Qt丰富的模块,逐步掌握C++在实际工程中的应用场景。
            \item 以项目为驱动,理论与实践结合,快速提升C++与Qt的综合开发能力。
        \end{itemize}
    \end{ytublock}
\end{frame}


\section{C语言基础知识}
\begin{frame}{目录}
    \begin{multicols}{2}
        \tableofcontents[currentsection]
    \end{multicols}
\end{frame}

\begin{frame}{C语言概述}
    \begin{ytublock}{C语言的重要性}
        \begin{itemize}
            \item \textbf{系统编程语言}:操作系统、驱动程序、嵌入式系统
            \item \textbf{高效性}:直接内存操作,编译型语言,执行效率高
            \item \textbf{强类型语言}:编译时类型检查,减少运行时错误
            \item \textbf{底层控制}:指针、位操作、内存管理
            \item \textbf{C++基础}:C++完全兼容C语言,是C的超集
        \end{itemize}
    \end{ytublock}

    \begin{ytublock}{C语言特点}
        \begin{itemize}
            \item \textbf{简洁高效}:语法简洁,编译后执行效率高
            \item \textbf{可移植性}:标准C代码可在不同平台编译运行
            \item \textbf{底层抽象}:提供硬件抽象,但不隐藏底层细节
            \item \textbf{过程式编程}:函数式编程范式,模块化设计
        \end{itemize}
    \end{ytublock}
\end{frame}

\begin{frame}{C语言基础语法}
    \begin{ytublock}{基本数据类型}
        \begin{itemize}
            \item \textbf{整数类型}:
            \begin{itemize}
                \item \texttt{int}:标准整型,通常用于存储整数,大小一般为4字节(32位系统/编译器)。
                \item \texttt{short}:短整型,通常为2字节,适合存储较小范围的整数。
                \item \texttt{long}:长整型,通常为4或8字节,能表示更大范围的整数。
                \item \texttt{char}:字符型,通常为1字节,用于存储单个字符或小整数。
            \end{itemize}
            \item \textbf{浮点类型}:
            \begin{itemize}
                \item \texttt{float}:单精度浮点型,通常为4字节,适合存储小数,精度有限。
                \item \texttt{double}:双精度浮点型,通常为8字节,精度更高,适合需要更高精度的计算。
            \end{itemize}
            \item \textbf{修饰符}:
            \begin{itemize}
                \item \texttt{signed}:有符号类型,能表示正数和负数(如\texttt{signed int})。
                \item \texttt{unsigned}:无符号类型,只能表示非负数,范围更大(如\texttt{unsigned int})。
            \end{itemize}
            \item \textbf{类型大小}:不同平台和编译器下,类型所占字节数可能不同。
        \end{itemize}
    \end{ytublock}
\end{frame}

\begin{frame}{C语言基础语法}
    \begin{ytublock}{变量和常量}
        \small
        \begin{itemize}
            \item \textbf{变量声明}:\texttt{类型 变量名;}。变量用于存储可变的数据,声明时可初始化(如\texttt{int b = 10;})。
            \item \textbf{常量定义}:
            \begin{itemize}
                \item \texttt{const}关键字:如\texttt{const int MAX = 100;},定义只读变量,编译期间检查不可修改。
                \item \texttt{\#define}宏定义:如\texttt{\#define PI 3.14},在预处理阶段进行文本替换,常用于全局常量。
            \end{itemize}
            \item \textbf{作用域}:
            \begin{itemize}
                \item \textbf{局部变量}:在函数或代码块内部声明,只在其作用范围内有效。
                \item \textbf{全局变量}:在所有函数外部声明,整个文件都可访问。
                \item \textbf{静态变量}:用\texttt{static}修饰,静态变量在函数多次调用间保持值不变。
            \end{itemize}
            \item \textbf{存储类}:
            \begin{itemize}
                \item \texttt{auto}:自动存储类,局部变量默认类型,现代C已很少显式使用。
                \item \texttt{static}:静态存储类,延长变量生命周期,见上。
                \item \texttt{extern}:外部变量声明,用于引用其他文件的全局变量。
                \item \texttt{register}:将变量存储在CPU寄存器中,提高访问速度,现代编译器通常自动优化。
            \end{itemize}
        \end{itemize}
    \end{ytublock}
\end{frame}

\begin{frame}[fragile]{C语言条件语句:if-else}
    \begin{ytublock}{if-else语句}
        \begin{itemize}
            \item \textbf{基本语法}:\texttt{if (条件) \{语句块1\} else \{语句块2\}}
            \item \textbf{功能}:根据条件表达式的真假,选择执行不同的代码块。
            \item \textbf{嵌套}:if语句可以嵌套使用,实现多级判断。
        \end{itemize}
    \end{ytublock}
    \begin{columns}
        \begin{column}{0.48\textwidth}
            \inputminted[firstline=9,lastline=17]{cpp}{code/c_control_structures.c}
        \end{column}
        \begin{column}{0.48\textwidth}
            \inputminted[firstline=19,lastline=26]{cpp}{code/c_control_structures.c}
        \end{column}
    \end{columns}
    % 代码片段说明:第1-15行为if-else结构及其嵌套的完整示例,含中文注释
\end{frame}

\begin{frame}[fragile]{C语言条件语句:switch-case}
    \begin{ytublock}{switch-case语句}
        \begin{itemize}
            \item \textbf{基本语法}:\texttt{switch (表达式) \{ case 常量: 语句; break; ... default: 语句; \}}
            \item \textbf{功能}:根据表达式的值,选择匹配的case分支执行。
            \item \textbf{注意事项}:每个case后通常加\texttt{break},否则会"贯穿"执行到下一个case。
        \end{itemize}
    \end{ytublock}
    \begin{columns}
        \begin{column}{0.48\textwidth}
            \inputminted[firstline=28,lastline=38]{cpp}{code/c_control_structures.c}
        \end{column}
        \begin{column}{0.48\textwidth}
            \begin{ytublock}{switch条件表达式限定}
                \begin{itemize}
                    \item \textbf{switch条件表达式}:必须是整数类型(如\texttt{int}、\texttt{char})。
                    \item \textbf{case常量}:必须是整数常量,且每个case的常量值必须不同。
                    \item \textbf{default}:可选,用于处理未匹配的case。
                \end{itemize}
            \end{ytublock}
        \end{column}
    \end{columns}
\end{frame}

\begin{frame}{C语言循环结构详解}
    \begin{ytublock}{for循环}
        \begin{itemize}
            \item \textbf{基本语法}:\texttt{for(初始化; 条件; 更新)\{循环体\}}
            \item \textbf{功能}:适合已知循环次数的场景。
            \item \textbf{执行顺序}:初始化 → 条件检查 → 循环体 → 更新 → 条件检查...
        \end{itemize}
    \end{ytublock}
    \inputminted[firstline=44,lastline=49]{cpp}{code/c_control_structures.c}
\end{frame}

\begin{frame}{C语言循环结构详解}
    \begin{ytublock}{while和do-while循环}
        \begin{itemize}
            \item \textbf{while循环语法}:\texttt{while(条件)\{循环体\}}
            \item \textbf{特点}:先判断条件,后执行循环体,可能一次都不执行。
            \item \textbf{do-while循环语法}:\texttt{do\{循环体\}while(条件);}
            \item \textbf{特点}:先执行循环体,再判断条件,至少执行一次。
        \end{itemize}
    \end{ytublock}
    \begin{columns}
        \begin{column}{0.48\textwidth}
            \inputminted[firstline=51,lastline=58]{cpp}{code/c_control_structures.c}
        \end{column}
        \begin{column}{0.48\textwidth}
            \inputminted[firstline=60,lastline=67]{cpp}{code/c_control_structures.c}
        \end{column}
    \end{columns}
\end{frame}

\begin{frame}{C语言循环结构详解}
    \begin{ytublock}{break和continue在循环中的应用}
        \begin{itemize}
            \item \textbf{break}:跳出当前循环,循环提前结束。
            \item \textbf{continue}:跳过本次循环剩余部分,直接进入下一次循环判断。
        \end{itemize}
    \end{ytublock}
    \inputminted[firstline=69,lastline=80]{cpp}{code/c_control_structures.c}
\end{frame}

\begin{frame}{C语言函数的基本概念}
    \begin{ytublock}{什么是函数?}
        \begin{itemize}
            \item \textbf{函数}是实现特定功能的独立代码块,可以重复调用。
            \item 通过函数可以将复杂问题分解为若干小问题,便于模块化设计和代码复用。
            \item C语言的函数包括\textbf{标准库函数}(如\texttt{printf}、\texttt{scanf})和\textbf{用户自定义函数}。
        \end{itemize}
    \end{ytublock}
    \begin{columns}
        \begin{column}{0.48\textwidth}
            \inputminted[firstline=1,lastline=7]{cpp}{code/c_function_simple.c}
        \end{column}
        \begin{column}{0.48\textwidth}
            \inputminted[firstline=9,lastline=17]{cpp}{code/c_function_simple.c}
        \end{column}
    \end{columns}
\end{frame}

\begin{frame}{C语言函数的声明与定义}
    \begin{ytublock}{函数声明}
        \begin{itemize}
            \item 作用:告诉编译器函数的名称、返回值类型和参数类型。
            \item 位置:一般写在文件开头或头文件中。
            \item 语法:\texttt{返回类型 函数名(参数类型列表);}
            \item 例:\texttt{int add(int a, int b);}
        \end{itemize}
    \end{ytublock}
    \begin{ytublock}{函数定义}
        \begin{itemize}
            \item 作用:给出函数的具体实现。
            \item 位置:函数声明之后,函数调用之前。
            \item 语法:\texttt{返回类型 函数名(参数列表)\{函数体\}}
            \item 例:\texttt{int add(int a, int b)\{ return a+b; \}}
        \end{itemize}
    \end{ytublock}
\end{frame}

\begin{frame}{C语言函数的调用}
    \begin{ytublock}{函数调用的基本过程}
        \begin{itemize}
            \item 使用\texttt{函数名(实参列表)}的形式调用函数。
            \item 调用时,实参的值会传递给函数的形参。
            \item 函数执行完毕后,将结果(如果有)返回给调用处。
        \end{itemize}
    \end{ytublock}
    \inputminted[firstline=11,lastline=12]{cpp}{code/c_function_simple.c}
\end{frame}

\begin{frame}{C语言参数传递方式}
    \begin{ytublock}{值传递(pass by value)}
        \begin{itemize}
            \item 调用函数时,将实参的值复制一份传递给形参。
            \item 在函数内部对形参的修改,不会影响到外部的实参变量。
            \item 适用于基本数据类型(如\texttt{int}、\texttt{float}等)。
        \end{itemize}
        \begin{columns}
            \begin{column}{0.48\textwidth}
                \textbf{值传递定义:}
                \inputminted[firstline=12,lastline=15]{cpp}{code/c_function_example.c}
            \end{column}
            \begin{column}{0.48\textwidth}
                \textbf{值传递示例:}
                \inputminted[firstline=46,lastline=47]{cpp}{code/c_function_example.c}
            \end{column}
        \end{columns}
    \end{ytublock}
\end{frame}

\begin{frame}{C语言参数传递方式}
    \begin{ytublock}{指针传递(地址传递,pass by pointer/address)}
        \begin{itemize}
            \item 通过传递变量的地址(指针)给函数,实现对外部变量的直接修改。
            \item 在函数内部通过解引用指针,可以改变外部变量的值。
            \item 适用于需要在函数内修改外部变量,或传递大型数据结构(如数组)。
        \end{itemize}
        \begin{columns}
            \begin{column}{0.48\textwidth}
                \textbf{指针传递定义:}
                \inputminted[firstline=17,lastline=22]{cpp}{code/c_function_example.c}
            \end{column}
            \begin{column}{0.48\textwidth}
                \textbf{指针传递示例:}
                \inputminted[firstline=50,lastline=54]{cpp}{code/c_function_example.c}
            \end{column}
        \end{columns}
    \end{ytublock}
\end{frame}

\begin{frame}{C语言函数的返回值与void类型}
    \begin{ytublock}{返回值}
        \begin{itemize}
            \item C语言函数可以有返回值,也可以没有返回值。返回值用于将函数内部计算的结果传递给调用者。
            \item 返回值类型在函数声明和定义时指定。例如,\texttt{int add(int a, int b)} 表示返回类型为\texttt{int}。
            \item 使用\texttt{return}语句返回结果。例如:\texttt{return a + b;}。
            \item 如果函数有返回值,必须保证所有可能的执行路径都能返回一个与声明类型一致的值。
            \item 调用有返回值的函数时,通常用变量接收返回结果。
            \item 如果函数声明为非\texttt{void}类型但未返回值,编译器会发出警告或错误。
        \end{itemize}
    \end{ytublock}
    \begin{ytublock}{void类型}
        \begin{itemize}
            \item \texttt{void}类型表示函数没有返回值。此类函数只执行操作,不向调用者返回数据。
            \item \texttt{void}函数可以使用\texttt{return;}语句提前结束函数,但不能带返回值。
        \end{itemize}
    \end{ytublock}
\end{frame}

\begin{frame}{C语言函数的作用与注意事项}
    \begin{columns}
        \begin{column}{0.48\textwidth}
    \begin{ytublock}{函数的优点}
        \begin{itemize}
            \item \textbf{可读性}
            \item \textbf{可维护性}
            \item \textbf{可测试性}
            \item \textbf{可扩展性}
            \item \textbf{可重用性}
            \item \textbf{可移植性}
            \item \textbf{模块化}
        \end{itemize}
    \end{ytublock}
    \end{column}
    \begin{column}{0.48\textwidth}
    \begin{ytublock}{注意事项}
        \begin{itemize}
            \item \textbf{函数名不能重复}
            \item \textbf{函数声明与定义要一致}
            \item \textbf{参数类型和个数要与声明一致}
            \item \textbf{注意变量作用域和生命周期}
            \item \textbf{返回值类型要匹配}
            \item \textbf{避免递归陷入死循环}
            \item \textbf{参数传递方式要明确}
            \item \textbf{防止未初始化变量参与运算}
        \end{itemize}
    \end{ytublock}
    \end{column}
    \end{columns}
\end{frame}

\begin{frame}{C语言指针基础}
    \begin{ytublock}{什么是指针}
        \begin{itemize}
            \item 指针是存储变量内存地址的变量。通过指针可以间接访问和操作内存中的数据。
            \item 声明指针:使用\texttt{*}声明指针类型,如\texttt{int *p;}表示\texttt{p}是一个指向\texttt{int}类型的指针。
            \item 取地址操作符\texttt{\&}:\texttt{p = \&x;}将变量\texttt{x}的地址赋值给指针\texttt{p}。
            \item 解引用操作符\texttt{*}:\texttt{*p}访问指针所指向的内存单元的值。
        \end{itemize}
    \end{ytublock}
    \inputminted[firstline=9,lastline=16]{cpp}{code/c_pointer_example.c}
\end{frame}

\begin{frame}{指针与变量}
    \begin{ytublock}{指针与变量}
        \begin{itemize}
            \item 通过指针可以修改变量的值。
            \item 指针本身也是一个变量,存储的是地址。
        \end{itemize}
    \end{ytublock}
    \begin{ytublock}{代码示例}
        \inputminted[firstline=9,lastline=20]{cpp}{code/c_pointer_example.c}
    \end{ytublock}
\end{frame}

\begin{frame}{指针与数组}
    \begin{ytublock}{指针与数组}
        \begin{itemize}
            \item 数组名本质上是首元素的指针,指针可以遍历数组元素。
            \item 指针可以进行加减运算(如\texttt{p++}),步长为所指类型的字节数,常用于数组遍历。
        \end{itemize}
    \end{ytublock}
    \begin{columns}
        \begin{column}{0.48\textwidth}
    \begin{ytublock}{指针访问数组代码示例}
        \inputminted[firstline=22,lastline=30]{cpp}{code/c_pointer_example.c}
    \end{ytublock}
    \end{column}
    \begin{column}{0.48\textwidth}
        \begin{ytublock}{指针修改数组代码示例}
            \inputminted[firstline=32,lastline=38]{cpp}{code/c_pointer_example.c}
        \end{ytublock}
    \end{column}
    \end{columns}
\end{frame}

\begin{frame}{C语言动态内存管理}
    \begin{ytublock}{内存分区与动态分配}
        \begin{itemize}
            \item C程序运行时内存主要分为代码区、全局/静态区、栈区和堆区。
            \item 栈内存:局部变量和函数参数存储在栈区,由系统自动分配和释放,空间有限,生命周期随函数调用结束而结束。
            \item 堆内存:通过\texttt{malloc()}等函数动态分配,需手动用\texttt{free()}释放,适合存储生命周期较长或大小不确定的数据。
        \end{itemize}
        \inputminted[firstline=8,lastline=12]{cpp}{code/c_pointer_memory_example.c}
    \end{ytublock}
\end{frame}

\begin{frame}{指针常见错误与良好习惯}
    \begin{ytublock}{常见错误}
        \begin{itemize}
            \item 未初始化指针,访问已释放内存,指针越界等都可能导致程序崩溃或不可预期行为。
            \item 内存泄漏:动态分配的内存未及时释放会造成内存泄漏,长期运行会耗尽系统资源。
            \item 野指针:指针指向的内存已被释放或未初始化,访问野指针会导致不可预期的错误。
            \item 悬空指针:释放内存后未将指针置为\texttt{NULL},指针仍然指向原地址,极易出错。
            \item 内存越界:访问数组或内存块时超出分配范围,可能破坏数据或引发崩溃。
        \end{itemize}
    \end{ytublock}
    \begin{ytublock}{良好习惯}
        \begin{itemize}
            \item 动态分配内存后及时检查返回值是否为\texttt{NULL},用完后立即\texttt{free()}并将指针赋为\texttt{NULL}。
            \item 初始化指针为\texttt{NULL},避免野指针。
        \end{itemize}
    \end{ytublock}
\end{frame}

\begin{frame}{C语言结构体}
    \begin{ytublock}{结构体概念}
        \begin{itemize}
            \item \textbf{复合数据类型}:将不同类型的数据组合在一起
            \item \textbf{成员访问}:使用 \texttt{.} 运算符
            \item \textbf{指针访问}:使用 \texttt{->} 运算符
            \item \textbf{内存对齐}:结构体成员的内存布局
        \end{itemize}
    \end{ytublock}

    \begin{ytublock}{结构体应用}
        \begin{itemize}
            \item \textbf{数据封装}:将相关数据组织在一起
            \item \textbf{函数参数}:传递复杂数据结构
            \item \textbf{链表实现}:自引用结构体
            \item \textbf{面向对象基础}:C++类的雏形
        \end{itemize}
    \end{ytublock}
\end{frame}

\begin{frame}[fragile]{C语言结构体示例}
    \begin{columns}
        \begin{column}{0.48\textwidth}
            \inputminted[firstline=1,lastline=14]{cpp}{code/c_struct_example.c}
        \end{column}
        \begin{column}{0.48\textwidth}
            \inputminted[firstline=15,lastline=28]{cpp}{code/c_struct_example.c}
        \end{column}
    \end{columns}
\end{frame}

\begin{frame}{C语言与C++的关系}
    \begin{ytublock}{兼容性}
        \begin{itemize}
            \item \textbf{完全兼容}:C++是C的超集,几乎所有的C代码都是有效的C++代码
            \item \textbf{语法兼容}:C的基本语法在C++中完全支持
            \item \textbf{库兼容}:C标准库在C++中可用(需要适当的头文件)
            \item \textbf{编译兼容}:C++编译器可以编译C代码
        \end{itemize}
    \end{ytublock}

    \begin{ytublock}{C++对C的扩展}
        \begin{itemize}
            \item \textbf{面向对象}:类、继承、多态
            \item \textbf{函数重载}:同名函数不同参数
            \item \textbf{引用}:变量的别名
            \item \textbf{模板}:泛型编程
            \item \textbf{异常处理}:try-catch机制
            \item \textbf{命名空间}:避免名称冲突
        \end{itemize}
    \end{ytublock}
\end{frame}

\begin{frame}{底层语言的特点}
    \begin{ytublock}{强类型语言的优势}
        \begin{itemize}
            \item \textbf{编译时检查}:类型错误在编译时发现,减少运行时错误
            \item \textbf{性能优化}:编译器可以根据类型信息进行优化
            \item \textbf{内存安全}:类型系统帮助防止内存访问错误
            \item \textbf{代码可读性}:类型信息使代码意图更清晰
        \end{itemize}
    \end{ytublock}

    \begin{ytublock}{底层语言的优势}
        \begin{itemize}
            \item \textbf{直接内存操作}:指针和位操作,精确控制内存
            \item \textbf{高效执行}:编译为机器码,执行效率高
            \item \textbf{硬件抽象}:提供硬件抽象但不隐藏底层细节
            \item \textbf{系统编程}:适合操作系统、驱动程序开发
            \item \textbf{资源控制}:精确控制CPU、内存等资源
        \end{itemize}
    \end{ytublock}
\end{frame}

\section{C++基本语法}
\begin{frame}{目录}
    \begin{multicols}{2}
        \tableofcontents[currentsection]
    \end{multicols}
\end{frame}

\begin{frame}{C++语言发展历程}
    \begin{ytublock}{C++的起源与设计理念}
        \begin{itemize}
            \item \textbf{1980年代}:Bjarne Stroustrup在贝尔实验室开发
            \item \textbf{设计目标}:结合C的高效性和Simula的面向对象特性
            \item \textbf{核心理念}:零开销抽象原则(Zero-Overhead Principle)
            \item \textbf{应用领域}:系统编程、应用开发、嵌入式系统
        \end{itemize}
    \end{ytublock}

    \begin{ytublock}{C++标准版本演进}
        \begin{itemize}
            \item \textbf{C++98} (1998年) - 第一个国际标准,确立基础语法
            \item \textbf{C++03} (2003年) - 技术修正版本,修复缺陷
            \item \textbf{C++11} (2011年) - 现代C++开始,重大更新
            \item \textbf{C++14} (2014年) - 功能完善和优化
            \item \textbf{C++17} (2017年) - 标准库增强
            \item \textbf{C++20} (2020年) - 最新标准,重大特性
        \end{itemize}
    \end{ytublock}
\end{frame}

\begin{frame}{C++11:现代C++的转折点}
    \begin{ytublock}{C++11的重大创新}
        \begin{itemize}
            \item \textbf{智能指针}:\texttt{std::unique\_ptr}, \texttt{std::shared\_ptr}
            \item \textbf{Lambda表达式}:函数式编程支持
            \item \textbf{移动语义}:性能优化的革命性改进
            \item \textbf{auto关键字}:类型推导,简化代码
            \item \textbf{范围for循环}:更简洁的迭代语法
            \item \textbf{nullptr}:类型安全的空指针
        \end{itemize}
    \end{ytublock}

    \begin{ytublock}{对编程范式的影响}
        \begin{itemize}
            \item \textbf{从OOP到多范式}:支持面向对象、泛型、函数式编程
            \item \textbf{性能优先}:移动语义显著提升性能
            \item \textbf{安全性提升}:智能指针减少内存泄漏
            \item \textbf{代码简化}:auto和Lambda减少样板代码
        \end{itemize}
    \end{ytublock}
\end{frame}

\begin{frame}{C++14/17/20:持续演进}
    \begin{columns}
        \begin{column}{0.48\textwidth}
            \begin{ytublock}{C++14增强}
                \begin{itemize}
                    \item \textbf{泛型Lambda}:支持auto参数
                    \item \textbf{变量模板}:模板变量声明
                    \item \textbf{数字分隔符}:提高可读性
                    \item \textbf{std::make\_unique}:智能指针工厂
                \end{itemize}
            \end{ytublock}
        \end{column}
        \begin{column}{0.48\textwidth}
            \begin{ytublock}{C++17新特性}
                \begin{itemize}
                    \item \textbf{结构化绑定}:多返回值处理
                    \item \textbf{std::optional}:可选值类型
                    \item \textbf{std::variant}:类型安全联合
                    \item \textbf{并行算法}:标准库并行化
                \end{itemize}
            \end{ytublock}
        \end{column}
    \end{columns}

    \begin{ytublock}{C++20重大更新}
        \begin{itemize}
            \item \textbf{概念(Concepts)}:模板约束系统
            \item \textbf{协程(Coroutines)}:异步编程支持
            \item \textbf{模块(Modules)}:编译时依赖管理
            \item \textbf{三向比较}:\texttt{<=>}操作符
        \end{itemize}
    \end{ytublock}
\end{frame}

\begin{frame}{C++语言特点}
    \begin{ytublock}{多范式编程语言}
        \begin{itemize}
            \item \textbf{面向对象编程}:封装、继承、多态,支持抽象和重用
            \item \textbf{泛型编程}:模板、STL,编译时多态
            \item \textbf{过程式编程}:函数、模块化设计(继承自C语言)
            \item \textbf{函数式编程}:Lambda表达式、算法库
        \end{itemize}
    \end{ytublock}

    \begin{ytublock}{C语言兼容性}
        \begin{itemize}
            \item \textbf{完全兼容}:C++是C的超集,几乎所有的C代码都是有效的C++代码
            \item \textbf{底层控制}:保留C语言的指针、内存管理、位操作等底层特性
            \item \textbf{性能优先}:零开销抽象原则,高级特性不带来性能损失
            \item \textbf{系统编程}:适合操作系统、驱动程序、嵌入式系统开发
        \end{itemize}
    \end{ytublock}
\end{frame}

\begin{frame}{C++的应用领域与优势}
    \begin{columns}
        \begin{column}{0.48\textwidth}
            \begin{ytublock}{主要应用领域}
                \begin{itemize}
                    \item \textbf{系统软件}:操作系统、驱动程序
                    \item \textbf{游戏开发}:引擎、图形渲染
                    \item \textbf{嵌入式系统}:实时控制、IoT设备
                    \item \textbf{高性能计算}:科学计算、金融交易
                    \item \textbf{桌面应用}:Qt、MFC、WPF
                \end{itemize}
            \end{ytublock}
        \end{column}
        \begin{column}{0.48\textwidth}
                \begin{ytublock}{技术优势}
        \begin{itemize}
            \item \textbf{跨平台}:一次编写,多处运行
            \item \textbf{类型安全}:编译时类型检查
            \item \textbf{向后兼容}:C语言兼容性,平滑过渡
            \item \textbf{标准化}:ISO标准,长期稳定
            \item \textbf{生态系统}:丰富的库和工具链
            \item \textbf{底层控制}:直接内存操作,高效执行
        \end{itemize}
    \end{ytublock}
        \end{column}
    \end{columns}

    \begin{ytublock}{与其他语言的比较}
        \begin{itemize}
            \item \textbf{vs C}:更强的类型安全,面向对象支持,但保持底层控制能力
            \item \textbf{vs Java}:更高的性能,更直接的内存控制,无虚拟机开销
            \item \textbf{vs Python}:编译型语言,执行效率更高,强类型检查
            \item \textbf{共同特点}:都是强类型语言,编译时检查,适合系统编程
        \end{itemize}
    \end{ytublock}
\end{frame}

\begin{frame}{C++与Qt的关系}
    \begin{ytublock}{Qt框架的C++基础}
        \begin{itemize}
            \item \textbf{原生C++框架}:Qt完全用C++编写,无虚拟机依赖
            \item \textbf{面向对象设计}:充分利用C++的封装、继承、多态特性
            \item \textbf{现代C++支持}:支持C++11及以后特性,包括智能指针、Lambda等
            \item \textbf{跨平台抽象}:统一不同操作系统的API,实现真正的跨平台
        \end{itemize}
    \end{ytublock}

    \begin{columns}
        \begin{column}{0.48\textwidth}
            \begin{ytublock}{Qt的C++特性应用}
                \begin{itemize}
                    \item \textbf{信号槽机制}:基于函数指针和回调
                    \item \textbf{元对象系统}:运行时类型信息和反射
                    \item \textbf{内存管理}:父子对象关系,自动清理
                    \item \textbf{模板应用}:容器类、算法库
                \end{itemize}
            \end{ytublock}
        \end{column}
        \begin{column}{0.48\textwidth}
            \begin{ytublock}{学习C++的优势}
                \begin{itemize}
                    \item \textbf{类型安全}:编译时错误检查,减少运行时错误
                    \item \textbf{性能优化}:C++的高效执行
                    \item \textbf{生态系统}:丰富的第三方库和工具支持
                \end{itemize}
            \end{ytublock}
        \end{column}
    \end{columns}
\end{frame}

\begin{frame}{Qt与C++的协同优势}
    \begin{columns}
        \begin{column}{0.48\textwidth}
            \begin{ytublock}{开发效率提升}
                \begin{itemize}
                    \item \textbf{快速原型}:Qt Designer可视化设计
                    \item \textbf{代码生成}:uic工具自动生成UI代码
                    \item \textbf{调试支持}:集成调试器和性能分析
                    \item \textbf{文档完善}:详细的API文档和示例
                \end{itemize}
            \end{ytublock}
        \end{column}
        \begin{column}{0.48\textwidth}
            \begin{ytublock}{性能与稳定性}
                \begin{itemize}
                    \item \textbf{编译优化}:C++编译器深度优化
                    \item \textbf{内存安全}:RAII和智能指针
                    \item \textbf{异常处理}:C++异常机制
                    \item \textbf{线程安全}:Qt的线程模型
                \end{itemize}
            \end{ytublock}
        \end{column}
    \end{columns}

    \begin{ytublock}{学习路径建议}
        \begin{itemize}
            \item \textbf{基础阶段}:掌握C++基本语法和面向对象概念
            \item \textbf{进阶阶段}:学习现代C++特性和Qt框架
            \item \textbf{实践阶段}:结合项目开发,深入理解两者结合
            \item \textbf{高级阶段}:性能优化、设计模式、架构设计
        \end{itemize}
    \end{ytublock}
\end{frame}

\begin{frame}{C++基础语法概述}
    \begin{ytublock}{C++对C的语法扩展}
        \begin{itemize}
            \item \textbf{完全兼容C}:C++是C的超集,所有C代码都是有效的C++代码
            \item \textbf{面向对象扩展}:类、对象、继承、多态
            \item \textbf{函数重载}:同名函数不同参数类型
            \item \textbf{引用类型}:变量的别名,避免指针的复杂性
            \item \textbf{命名空间}:避免名称冲突
            \item \textbf{异常处理}:try-catch机制
        \end{itemize}
    \end{ytublock}

    \begin{ytublock}{C++语法特点}
        \begin{itemize}
            \item \textbf{强类型}:编译时类型检查,类型安全
            \item \textbf{静态类型}:类型在编译时确定
            \item \textbf{编译型语言}:直接编译为机器码
            \item \textbf{多范式}:支持过程式、面向对象、泛型编程
        \end{itemize}
    \end{ytublock}
\end{frame}

\begin{frame}{C++基本数据类型}
    \begin{ytublock}{C++数据类型扩展}
        \begin{itemize}
            \item \textbf{bool类型}:\texttt{true}和\texttt{false},C++原生支持
            \item \textbf{wchar\_t}:宽字符类型,支持Unicode
            \item \textbf{引用类型}:\texttt{int\&}、\texttt{double\&}等
            \item \textbf{类类型}:用户自定义类型
            \item \textbf{模板类型}:泛型类型
        \end{itemize}
    \end{ytublock}

    \begin{ytublock}{类型安全增强}
        \begin{itemize}
            \item \textbf{类型转换}:显式类型转换,避免隐式转换错误
            \item \textbf{const修饰符}:常量类型,防止意外修改
            \item \textbf{类型推导}:auto关键字,编译器自动推导类型
            \item \textbf{nullptr}:类型安全的空指针(C++11)
        \end{itemize}
    \end{ytublock}
\end{frame}

\begin{frame}{C++变量和常量声明}
    \begin{ytublock}{C++变量声明增强}
        \begin{itemize}
            \item \textbf{引用声明}:\texttt{int\& ref = x;} 创建变量的别名
            \item \textbf{const引用}:\texttt{const int\& ref = x;} 只读引用
            \item \textbf{auto关键字}:\texttt{auto x = 42;} 自动类型推导
            \item \textbf{decltype}:\texttt{decltype(expr)} 推导表达式类型
        \end{itemize}
    \end{ytublock}

    \begin{ytublock}{常量声明}
        \begin{itemize}
            \item \textbf{const常量}:\texttt{const int MAX = 100;}
            \item \textbf{constexpr}:编译时常量(C++11)
            \item \textbf{constinit}:编译时初始化(C++20)
            \item \textbf{const成员函数}:\texttt{void func() const;}
        \end{itemize}
    \end{ytublock}
\end{frame}

\begin{frame}[fragile]{C++变量和常量示例}
    \inputminted[firstline=1,lastline=15]{cpp}{code/cpp_variables_constants.cpp}
\end{frame}

\begin{frame}{C++运算符和表达式}
    \begin{ytublock}{C++运算符扩展}
        \begin{itemize}
            \item \textbf{作用域解析运算符}:\texttt{::} 访问全局变量或类成员
            \item \textbf{成员访问运算符}:\texttt{.} 和 \texttt{->} 访问对象成员
            \item \textbf{类型转换运算符}:\texttt{static\_cast}、\texttt{dynamic\_cast}
            \item \textbf{条件运算符}:\texttt{?:} 三元运算符
        \end{itemize}
    \end{ytublock}

    \begin{ytublock}{表达式增强}
        \begin{itemize}
            \item \textbf{函数调用表达式}:支持函数重载和默认参数
            \item \textbf{成员函数调用}:\texttt{obj.func()} 或 \texttt{ptr->func()}
            \item \textbf{模板实例化}:\texttt{vector<int> v;}
            \item \textbf{Lambda表达式}:\texttt{[](int x) \{ return x * 2; \}}
        \end{itemize}
    \end{ytublock}
\end{frame}

\begin{frame}{C++控制结构}
    \begin{ytublock}{C++控制结构增强}
        \begin{itemize}
            \item \textbf{范围for循环}:\texttt{for (auto\& item : container)} (C++11)
            \item \textbf{初始化语句}:\texttt{if (auto it = find(x); it != end())}
            \item \textbf{结构化绑定}:\texttt{auto [x, y] = pair;} (C++17)
            \item \textbf{switch增强}:支持初始化语句和fallthrough
        \end{itemize}
    \end{ytublock}

    \begin{ytublock}{异常处理}
        \begin{itemize}
            \item \textbf{try-catch}:异常捕获和处理
            \item \textbf{throw}:抛出异常
            \item \textbf{noexcept}:指定函数不抛出异常
            \item \textbf{RAII}:资源获取即初始化
        \end{itemize}
    \end{ytublock}
\end{frame}

\begin{frame}{C++函数}
    \begin{ytublock}{C++函数特性}
        \begin{itemize}
            \item \textbf{函数重载}:同名函数不同参数类型或数量
            \item \textbf{默认参数}:\texttt{void func(int x = 0, int y = 0);}
            \item \textbf{内联函数}:\texttt{inline} 关键字
            \item \textbf{函数模板}:泛型函数
            \item \textbf{Lambda表达式}:匿名函数
        \end{itemize}
    \end{ytublock}

    \begin{ytublock}{函数调用约定}
        \begin{itemize}
            \item \textbf{值传递}:\texttt{void func(int x);} 传递副本
            \item \textbf{引用传递}:\texttt{void func(int\& x);} 避免拷贝
            \item \textbf{const引用}:\texttt{void func(const int\& x);} 只读
            \item \textbf{右值引用}:\texttt{void func(int\&\& x);} 移动语义
        \end{itemize}
    \end{ytublock}
\end{frame}

\begin{frame}[fragile]{C++函数示例}
    \begin{columns}
        \begin{column}{0.48\textwidth}
                \inputminted[firstline=1,lastline=18]{cpp}{code/cpp_function_basic.cpp}
        \end{column}
        \begin{column}{0.48\textwidth}
                \inputminted[firstline=20,lastline=36]{cpp}{code/cpp_function_basic.cpp}
        \end{column}
    \end{columns}
\end{frame}

\begin{frame}[fragile]{C++函数重载}
    \begin{columns}
        \begin{column}{0.48\textwidth}
            \inputminted[firstline=1,lastline=20]{cpp}{code/cpp_function_overload.cpp}
        \end{column}
        \begin{column}{0.48\textwidth}
                \inputminted[firstline=21,lastline=37]{cpp}{code/cpp_function_overload.cpp}
        \end{column}
    \end{columns}
\end{frame}

\begin{frame}{Lambda表达式}
    \begin{ytublock}{Lambda表达式语法}
        \begin{itemize}
            \item \texttt{[capture](parameters) -> return\_type \{ body \}}
            \item capture:捕获外部变量
            \item parameters:参数
            \item return\_type:返回类型
            \item body:函数体
        \end{itemize}
    \end{ytublock}

    \begin{ytublock}{Lambda表达式捕获}
        \begin{itemize}
            \item \texttt{[]}:不捕获任何外部变量。若在Lambda体内使用未捕获的外部变量会导致编译错误。
            \item \texttt{[x, \&y]}:x以值捕获,y以引用捕获。
            \item \texttt{[\&]}:所有被用到的外部变量都以引用方式捕获。
            \item \texttt{[=]}:所有被用到的外部变量都以值方式捕获。
            \item \texttt{[\&, x]}:x以值方式捕获,其余变量以引用方式捕获。
            \item \texttt{[=, \&z]}:z以引用方式捕获,其余变量以值方式捕获。
        \end{itemize}
    \end{ytublock}
\end{frame}

\begin{frame}[fragile]{C++ Lambda表达式示例}
    \inputminted[firstline=1,lastline=18]{cpp}{code/cpp_lambda_example.cpp}
\end{frame}

\begin{frame}{C++数组}
    \begin{ytublock}{C++数组增强与细节}
        \begin{itemize}
            \item \textbf{原生数组}:如 \texttt{int arr[5];},大小固定,不能自动推断长度,越界不安全。
            \item \textbf{std::array}:C++11引入,固定大小、类型安全,支持标准库算法,如 \texttt{std::array<int, 5> arr;}。
            \item \textbf{std::vector}:动态数组,自动管理内存,可动态扩容,常用操作有 \texttt{push\_back}、\texttt{size}、\texttt{at} 等。
            \item \textbf{初始化列表}:可用于原生数组、\texttt{std::array}、\texttt{std::vector},如 \texttt{int arr[] = \{1, 2, 3\};} 或 \texttt{std::vector<int> v = \{1,2,3\};}。
            \item \textbf{范围for循环}:C++11起支持,简化遍历,如 \texttt{for (auto x : arr) \{\}},适用于原生数组、\texttt{std::array}、\texttt{std::vector}。
            \item \textbf{迭代器遍历}:\texttt{std::vector} 和 \texttt{std::array} 支持迭代器,可用 \texttt{begin()} 和 \texttt{end()} 进行灵活遍历。
            \item \textbf{内存安全}:\texttt{std::vector} 的 \texttt{at()} 方法有越界检查,原生数组无越界保护。
            \item \textbf{多维数组}:原生数组、\texttt{std::array}、\texttt{std::vector} 均可实现多维数组。
        \end{itemize}
    \end{ytublock}
\end{frame}

\begin{frame}[fragile]{C++原生数组示例}
    \inputminted[firstline=1,lastline=20]{cpp}{code/cpp_array_example.cpp}
\end{frame}

\begin{frame}[fragile]{C++std::array示例}
    \inputminted[firstline=1,lastline=20]{cpp}{code/cpp_stdarray_example.cpp}
\end{frame}

\begin{frame}[fragile]{C++std::vector示例}
        \begin{columns}
        \begin{column}{0.48\textwidth}
            \inputminted[firstline=1,lastline=17]{cpp}{code/cpp_vector_example.cpp}
        \end{column}
        \begin{column}{0.48\textwidth}
            \inputminted[firstline=18,lastline=38]{cpp}{code/cpp_vector_example.cpp}
        \end{column}
    \end{columns}
\end{frame}

\begin{frame}{C++指针和引用}
    \begin{ytublock}{C++指针增强}
        \begin{itemize}
            \item \textbf{nullptr}:类型安全的空指针(C++11)
            \item \textbf{智能指针}:\texttt{std::unique\_ptr}、\texttt{std::shared\_ptr}
            \item \textbf{void指针}:\texttt{void*} 通用指针
            \item \textbf{函数指针}:指向函数的指针
        \end{itemize}
    \end{ytublock}

    \begin{ytublock}{C++引用}
        \begin{itemize}
            \item \textbf{左值引用}:\texttt{int\& ref = x;}
            \item \textbf{常量引用}:\texttt{const int\& ref = x;}
            \item \textbf{右值引用}:\texttt{int\&\& ref = 42;} (C++11)
            \item \textbf{引用vs指针}:引用更安全,不能为空
        \end{itemize}
    \end{ytublock}
\end{frame}

\begin{frame}[fragile]{C++指针示例}
    \inputminted[firstline=1,lastline=18]{cpp}{code/cpp_pointer_example.cpp}
\end{frame}

\begin{frame}[fragile]{C++引用示例}
    \inputminted[firstline=1,lastline=15]{cpp}{code/cpp_reference_example.cpp}
\end{frame}

\section{C++面向对象编程}
\begin{frame}{目录}
    \begin{multicols}{2}
        \tableofcontents[currentsection]
    \end{multicols}
\end{frame}

\begin{frame}{C++面向对象编程}
    \begin{columns}
        \begin{column}{0.48\textwidth}
    \begin{ytublock}{什么是类?}
        \begin{itemize}
            \item 类是对象的蓝图或模板
            \item 类定义了对象的结构和行为
            \item 类包含数据成员和成员函数
            \item 类是抽象的,对象是具体的
        \end{itemize}
    \end{ytublock}
    \end{column}
    \begin{column}{0.48\textwidth}
    \begin{ytublock}{什么是对象?}
        \begin{itemize}
            \item 对象是类的实例
            \item 对象具有类定义的结构和行为
            \item 对象是具体的,有自己的状态和行为
        \end{itemize}
    \end{ytublock}
    \end{column}
    \end{columns}
    \begin{ytublock}{面向对象的优点}
        \begin{itemize}
            \item 封装:将数据和方法封装在类中,隐藏实现细节,只暴露接口
            \item 继承:继承父类的方法和属性,实现代码复用
            \item 多态:实现接口的统一,不同实现方式
        \end{itemize}
    \end{ytublock}
\end{frame}

\begin{frame}[fragile]{C++类的定义、对象的创建与使用}
        \begin{columns}
            \begin{column}{0.48\textwidth}
                \inputminted[firstline=1,lastline=16]{cpp}{code/cpp_class_example.cpp}
            \end{column}
            \begin{column}{0.48\textwidth}
                \inputminted[firstline=18,lastline=30]{cpp}{code/cpp_class_example.cpp}
            \end{column}
        \end{columns}
\end{frame}

\begin{frame}[fragile]{C++继承}
    \begin{columns}
        \begin{column}{0.48\textwidth}
        \begin{itemize}
            \item 基类:父类,被继承的类,基类中定义的属性和方法可以被派生类继承
            \item 派生类:子类,继承父类的类,在基类的基础上扩展新的属性和方法
            \item 继承:派生类继承基类的属性和方法,派生类可以重写基类的方法
        \end{itemize}
        \end{column}
        \begin{column}{0.48\textwidth}
            \inputminted[firstline=1,lastline=16]{cpp}{code/cpp_inherit_example.cpp}
        \end{column}
        \begin{column}{0.48\textwidth}
            \inputminted[firstline=18,lastline=28]{cpp}{code/cpp_inherit_example.cpp}
        \end{column}
    \end{columns}
\end{frame}

\begin{frame}[fragile]{C++多继承}
    \begin{columns}
        \begin{column}{0.48\textwidth}
            \inputminted[firstline=1,lastline=17]{cpp}{code/cpp_multi_inherit_example.cpp}
        \end{column}
        \begin{column}{0.48\textwidth}
            \inputminted[firstline=19,lastline=33]{cpp}{code/cpp_multi_inherit_example.cpp}
        \end{column}
    \end{columns}
\end{frame}

\begin{frame}[fragile]{C++继承的规则}
    \begin{columns}
        \begin{column}{0.48\textwidth}
            \begin{itemize}
                \item 公有继承:基类的public和protected成员在派生类中保持原样,private成员不可访问
                \item 保护继承:基类的public和protected成员在派生类中变为protected,private成员不可访问
                \item 私有继承:基类的public和protected成员在派生类中变为private,private成员不可访问
            \end{itemize}
        \end{column}
        \begin{column}{0.48\textwidth}
            \inputminted[firstline=1,lastline=15]{cpp}{code/cpp_inherit_rule.cpp}
        \end{column}
    \end{columns}
\end{frame}

\begin{frame}[fragile]{C++继承的规则}
    \begin{columns}
        \begin{column}{0.48\textwidth}
            \inputminted[firstline=17,lastline=37]{cpp}{code/cpp_inherit_rule.cpp}
        \end{column}
        \begin{column}{0.48\textwidth}
            \inputminted[firstline=38,lastline=57]{cpp}{code/cpp_inherit_rule.cpp}
        \end{column}
    \end{columns}
\end{frame}

\begin{frame}[fragile]{C++多态}
    \begin{columns}
        \begin{column}{0.48\textwidth}
            \begin{ytublock}{多态}
                \begin{itemize}
                    \item 多态:派生类可以重写基类的方法,实现不同的行为
                    \item 虚函数:基类中定义的虚函数,派生类可以重写
                    \item 纯虚函数:基类中定义的纯虚函数,派生类必须重写
                \end{itemize}
            \end{ytublock}
        \end{column}
        \begin{column}{0.48\textwidth}
            \inputminted[firstline=1,lastline=17]{cpp}{code/cpp_polymorphism_example.cpp}
        \end{column}
    \end{columns}
\end{frame}

\begin{frame}[fragile]{C++多态}
    \begin{columns}
        \begin{column}{0.48\textwidth}
            \begin{ytublock}{多态的作用}
                \begin{itemize}
                    \item 代码复用:派生类直接使用基类的属性和方法,不需要重新实现
                    \item 接口统一:派生类和基类具有相同的接口,可以相互替换
                    \item 扩展性:新增子类时,不需要修改父类,只需要重写子类的方法
                \end{itemize}
            \end{ytublock}
        \end{column}
        \begin{column}{0.48\textwidth}
            \inputminted[firstline=19,lastline=33]{cpp}{code/cpp_polymorphism_example.cpp}
        \end{column}
    \end{columns}
\end{frame}

\begin{frame}{智能指针}
    \begin{columns}
        \begin{column}{0.58\textwidth}
        \begin{ytublock}{智能指针的类型}
        \begin{itemize}
            \item \textbf{std::unique\_ptr} - 独占所有权,不能复制
            \item \textbf{std::shared\_ptr} - 共享所有权,引用计数
            \item \textbf{std::weak\_ptr} - 弱引用,不增加引用计数
            \item \textbf{std::auto\_ptr} - 已废弃,C++17移除
        \end{itemize}
        \end{ytublock}
        \end{column}
        \begin{column}{0.38\textwidth}
        \begin{ytublock}{智能指针的优势}
            \begin{itemize}
                \item \textbf{自动内存管理} - 避免内存泄漏
                \item \textbf{异常安全} - 异常时自动清理
                \item \textbf{RAII} - 资源获取即初始化
                \item \textbf{线程安全} - shared\_ptr线程安全
            \end{itemize}
        \end{ytublock}
        \end{column}
    \end{columns}
    \begin{ytublock}{什么是引用计数?}
        \begin{itemize}
            \item 创建对象时,引用计数为1
            \item 当新的指针或者引用指向该对象时,引用计数加1
            \item 当指针或者引用不再指向该对象时,引用计数减1
            \item 当引用计数为0时,对象被销毁,资源被释放
        \end{itemize}
    \end{ytublock}
\end{frame}

\begin{frame}[fragile]{C++智能指针示例}
    \begin{columns}
        \begin{column}{0.48\textwidth}
            \inputminted[firstline=1,lastline=18]{cpp}{code/cpp_smart_pointer_clean.cpp}
        \end{column}
        \begin{column}{0.48\textwidth}
            \inputminted[firstline=19,lastline=35]{cpp}{code/cpp_smart_pointer_clean.cpp}
        \end{column}
    \end{columns}
\end{frame}

\begin{frame}{总结}
    \begin{itemize}
        \item 本章介绍了C的基本语法、C++的基本语法、Lambda表达式、数组、指针和引用、面向对象编程、智能指针等概念,通过示例代码,我们了解了这些概念的用法和实现方式
        \item 在实际开发中,我们可以根据需要选择合适的数据结构和算法,提高代码的效率和可读性
        \item C++的语法还有很多,比如模板、异常处理、多线程、网络编程、数据库编程、图形编程等,这些内容可以参考其他书籍或者网络资源
    \end{itemize}
\end{frame}
\end{document}